\documentclass[11pt, oneside]{article} 
\usepackage{geometry}
\geometry{letterpaper} 
\usepackage{graphicx}
	
\usepackage{amssymb}
\usepackage{amsmath}
\usepackage{parskip}
\usepackage{color}
\usepackage{hyperref}

\graphicspath{{/Users/telliott/Dropbox/Github-Math/figures/}}
% \begin{center} \includegraphics [scale=0.4] {gauss3.png} \end{center}

\title{Zero}
\date{}

\begin{document}
\maketitle
\Large

%[my-super-duper-separator]

Consider the equation $xy = 1$.  

Now suppose $y = 0$.  What is $x$?  The answer is that since $0$ times \emph{anything} is zero, there is no $x$ that satisfies the equation when $y = 0$.  

If we multiply both sides of the equation by $1/y$ 
\[ x = \frac{1}{y} \]

It's less obvious now, but still true that if $y = 0$, then there is no real number $x$ to satisfy the equation.

Thus, we learn in algebra not to do this
\[ \frac{1}{0} \stackrel{?}{=} \]

We say that the result of division by zero is \emph{undefined}. 

Let's suppose the denominator is not $0$ but instead is small and then ask what happens if it becomes smaller.  So for example $1/0.1 = 10$, and using exponents
\[ \frac{1}{10^{-1}} = 10^{1} \]

Then
\[ \frac{1}{10^{-1000}} = 10^{1000} \]

The inverse of a very very very small number is a very very very large number.  What's the largest number?  

Infinity, symbolized $\infty$, is a candidate.  Suppose we try treating $\infty$ as a number.  Maybe
\[ \frac{1}{0} \stackrel{?}{=} \infty \]

However, infinity is a strange beast.  Consider
\[ \infty + 1 \stackrel{?}{=} \]

Certainly some number one more than infinity is also infinite?  $\infty$ is supposed to be the largest number.  But then if
\[  \infty + 1 = \infty \]

Subtracting $\infty$ on both sides (assuming that $\infty$ is just a number) we obtain
\[ 1 = 0 \]

That's problematic, to say the least.  

Continuing our exploration, suppose we try

\[ \frac{1}{0} = \infty \]

What do we mean by the division operation?  We mean that result of $a/b$ is a number $c$ such that
\[ a = c \cdot b \]

So, applying the same logic to our problem
\[ 1 = \infty \cdot 0 \]

But $0$ times any number is defined to be equal to zero.  Oops.  We must also ask about $2/0$.  It must be that
\[ \frac{2}{0} = 2 \cdot \frac{1}{0} = 2 \cdot \infty =  \infty \]

So, working with the left-hand and very right-hand sides:
\[ 2 = \infty \cdot 0 \]

and since $1 = \infty \cdot 0$ it follows that

\[ 2 = \infty \cdot 0 = 1 \]

These are huge contradictions.  We conclude that

$\bullet$ \ \ $\infty$ is not a number

$\bullet$ \ \ Division by $0$ is not defined

\subsection*{another demonstration}
Suppose we accept for a moment that $\frac{1}{0}$ is \emph{something}, without specifying exactly what it is.  Then consider
\[ 0 \cdot \frac{1}{0} \]

On one hand, if we use the rule that $0 \cdot a = 0$, where $a$ is any number, then
\[ 0 \cdot \frac{1}{0} = 0 \]

On the other hand, if we use the rule that any number divided by itself is $1$
\[ 0 \cdot \frac{1}{0} = 1 \cdot \frac{0}{0} = 1 \cdot 1 = 1  \]

We have to break at least one rule, and that's not desireable.



\end{document}
