\documentclass[11pt, oneside]{article} 
\usepackage{geometry}
\geometry{letterpaper} 
\usepackage{graphicx}
	
\usepackage{amssymb}
\usepackage{amsmath}
\usepackage{parskip}
\usepackage{color}
\usepackage{hyperref}

\graphicspath{{/Users/telliott/Dropbox/Github-Math/figures/}}
% \begin{center} \includegraphics [scale=0.4] {gauss3.png} \end{center}

\title{adding fractions}
\date{}

\begin{document}
\maketitle
\Large

%[my-super-duper-separator]
\subsection*{Denominators are different and prime.}
\[ \frac{1}{2} + \frac{1}{3}, \ \ \ \ \ \  \frac{1}{3} + \frac{1}{5}, \ \ \ \ \ \   \frac{1}{2} + \frac{1}{13} , \ \ \ \ \ \    \frac{1}{a} + \frac{1}{b} = \frac{b + a}{ab} \]
\subsection*{One or both denominators not prime but have no common factors.}
\[ \frac{1}{9} + \frac{1}{10}, \ \ \ \ \ \  \frac{1}{11} + \frac{1}{15}, \ \ \ \ \ \   \frac{1}{15} + \frac{1}{16} , \ \ \ \ \ \   \text{same as above} \]
\subsection*{One is a multiple of the other.}
\[ \frac{1}{2} + \frac{1}{4}, \ \ \ \ \ \  \frac{1}{20} + \frac{1}{5}, \ \ \ \ \ \   \frac{1}{7} + \frac{1}{77} , \ \ \ \ \ \    \frac{1}{a} + \frac{1}{ab} = \frac{b+1}{ab} \]
\subsection*{They share a common factor.}
\[ \frac{1}{6} + \frac{1}{9}, \ \ \ \ \ \  \frac{1}{20} + \frac{1}{15}, \ \ \ \ \ \   \frac{1}{33} + \frac{1}{77} \]
\[ \frac{1}{ab} + \frac{1}{bc} = \frac{bc+ab}{abc} \]
Similar to the last, but a common factor is found in the numerator afterward.
\[ \frac{1}{10} + \frac{1}{15} = \frac{6 + 4}{60} = \frac{1}{6} \]

\subsection*{finding a common factor}
$\circ$ \ Method 1:  write the \emph{multiples} of each denominator.
\begin{verbatim}
 6 12 18 ..
 9 18 ..
 
20 40 60 ..
15 30 45 60 ..

33  66  99 132 165 198 231 ..
77 154 231 ..
\end{verbatim}
$\circ$ \ Method 2:  find the \emph{prime factors} of each denominator.
\begin{verbatim}
6 =   2.3    
9 =     3.3
      2.3.3 = 18
 
20 =  2.2. .5     
15 =      3.5
      2.2.3.5 = 60

33 =  3. .11
77 =    7.11
      3.7.11 = 231
\end{verbatim}
$\circ$ \ 3: Euclid's algorithm for the \emph{greatest common divisor} (gcd):
\begin{verbatim}
20 = 1.15 + 5
15 = 3.5 +  0      5 is the gcd
\end{verbatim}
Stop when the remainder is zero.  $5$ is the gcd of $20$ and $15$.  Divide $20/5 = 4$ and then multiply $4 \cdot 15 = 60$.
\begin{verbatim}
77 = 2.33 + 11
33 = 3.11 +  0     11 is the gcd
\end{verbatim}
Stop when the remainder is zero.  $11$ is the gcd of $33$ and $77$.  Divide $77/11 = 7$ and then multiply $7 \cdot 33 = 231$.

\end{document}
