\documentclass[11pt, oneside]{article} 
\usepackage{geometry}
\geometry{letterpaper} 
\usepackage{graphicx}
	
\usepackage{amssymb}
\usepackage{amsmath}
\usepackage{parskip}
\usepackage{color}
\usepackage{hyperref}

\graphicspath{{/Users/telliott/Dropbox/Github-math/figures/}}
% \begin{center} \includegraphics [scale=0.4] {gauss3.png} \end{center}

\title{Vertex and roots}
\date{}

\begin{document}
\maketitle
\Large

%[my-super-duper-separator]

\subsection*{min/max value}
We start with the standard form for a quadratic
\[ y = ax^2 + bx + c \]

We have been told that the vertex of the graph lies at the point with
\[ x = - \frac{b}{2a} \]
We might call that point $x_m$ for min/max but I prefer to just call it $m$: 
\[ x = m = - \frac{b}{2a} \]
It's the $x$-value at the vertex.  If you like, you can also figure out the value of $y$ at the vertex by plugging in to the equation, but we won't do that.

Instead, we want to focus on the zeros or roots of the equation, those values of $x$ that give $y = 0$.  (We ignore the complication that for some equations, some graphs, they may not cross the $x$-axis so there are no such values in the normal way of thinking.  We'll explain later).

\[ 0 = ax^2 + bx + c \]
And having that $0$ immediately suggests 
\[ 0 = x^2 + \frac{b}{a}x + \frac{c}{a} \]

The same $x$-values that give zero in the first equation also give zero in the second one, but the second is simpler to solve.

\subsection*{average of $s$ and $t$}
Here is the equation we wrote previously using the zeros $s$ and $t$
\[ 0 = (x - s)(x - t) \]
\[ 0 = x^2 - (s + t)x + st \]
Compare with
\[ 0 = x^2 + \frac{b}{a}x + \frac{c}{a} \]

If these are two forms of the \emph{same} equation, then the cofactors of $x$ must match:
\[ - (s + t) = \frac{b}{a} \]
\[ s + t =  - \frac{b}{a} \]
And if, as we said, the vertex $x = m$ is the average of $s$ and $t$
\[ m = \frac{1}{2} (s + t)\]
\[  = \frac{1}{2} (- \frac{b}{a}) = - \frac{b}{2a} \]
So that explains where the formula for the vertex $m$ comes from.

The other part $c/a$ must also match.  That is
\[ \frac{c}{a} = st \]
we'll come back to that.

\subsection*{completing the square}
We have 
\[ m = - \frac{b}{2a} \]
Re-arranging:
\[ -2m = \frac{b}{a} \]
Let us plug that into the standard form
\[ x^2 + \frac{b}{a} x + \frac{c}{a} = 0 \]
\[ x - 2mx = - \frac{c}{a} \]
We want the left-hand side be a perfect square.
\[ (x + \text{ something })^2 \]

The trick is to see that if we add $m^2$ it will work with $m$ as the something
\[ x - 2mx + m^2 = (x - m)^2 \]
Of course, we must add $m^2$ on both sides of the original equation so from
\[ x - 2mx = - \frac{c}{a} \]
we get
\[ x - 2mx + m^2 = m^2 - \frac{c}{a} \]
\[ (x - m)^2 = m^2 - \frac{c}{a} \]
Now it's just a little algebra.  We take the square root, which means we can have either the positive or the negative branch.
\[ x - m = \pm \ \sqrt{m^2 - c/a} \]
\[ x = m \pm \ \sqrt{m^2 - c/a} \]

This is a second formula to memorize.  If you plug in the value of $m$ (i.e. $m = -b/2a$) and also $c/a$, then it will give us the roots of the equation that we started with.

\subsection*{problems}
In the previous write-up we had this problem
\[ y = -16t^2 + 64t + 144 \]
and we want to know the time $t$ when $y = 0$ so
\[ 0 = -16t^2 + 64t + 144 \]
Divide both sides by $16$
\[ 0 = -t^2 + 4t + 9 \]
$m = -b/2a = 2$ and the roots are
\[ x = 2 \pm \ \sqrt{4 - 4(-1)(9)} = 2 \pm \sqrt{40} \]
It's not a round number but $2 + \sqrt{40}$ is about $8.3$ seconds.  Notice that we pick the branch with a positive square root because we are interested only in positive values of $t$.

The sum of two numbers is 18 and their product is 56.  What are they?

\[ u + v = 18 \]
\[ uv = 56 \]

\[ u(18 - u) = 56 \]
\[ u^2 - 18u + 56 = 0 \]

The factors of $56$ are $2 \cdot 28$, $4 \cdot 14$, and $7 \cdot 8$.
\[ 4 + 14 = 18 \]
so
\[ (u - 4)(u - 14) = 0 \]

The numbers are $4$ and $14$.

The product of two consecutive positive odd numbers is $255$.  Find the numbers.

Odd numbers can be described as $2n+1$ for some $n$, since that is one more than the even number $2n$.  So we have
\[ (2n + 1)(2n + 3) = 255 \]
\[ 4n^2 + 8n + 3 = 255 \]
\[ 4n^2 + 8n - 252 = 0 \]
We are looking for zeros.  So we can divide by $4$:
\[ n^2 + 2n - 63 = 0 \]
Two numbers multiply to give $-63$ and add to give $2$:
\[ (n + 9)(n - 7) = 0 \]
The roots are $-9, 7$.  

But we specified positive numbers which means $n = 7$.  The numbers are $2 \cdot 7 + 1 = 15$ and $17$, easily confirmed.

Alternatively we can write the numbers as $2n + 1$ and $2n - 1$ and then
\[ (2n + 1)(2n - 1) = 255 \]
This gives a simpler solution.
\[ 4n^2 - 1 = 255 \]
\[ n^2 = 64 \]
\[ n = \pm 8 \]
Again $n > 0$ so $n = 8$ and the numbers are $2n - 1 = 15$ and $2n + 1 = 17$.

\subsection*{quadratic formula}
I use the formula
\[ x = m \pm \ \sqrt{m^2 - c/a} \]
because it's relatively simple, but the one you will find in your book doesn't use $m$, it uses $a,b$ and $c$.

Let us plug in $m = -b/2a$ and see what we get.
\[ x = - \frac{b}{2a} \ \pm \ \sqrt{(\frac{-b}{2a})^2 - \frac{c}{a}} \]

The trick is to put$c$ over the same denominator as everything else
\[ x = - \frac{b}{2a} \ \pm \ \sqrt{(\frac{-b}{2a})^2 - \frac{4ac}{4a^2}} \]
We multiplied the last term by $4a$ on top and bottom.

So then
\[ x = - \frac{b}{2a} \ \pm \ \sqrt{\frac{b^2}{(2a)^2} - \frac{4ac}{(2a)^2}} \]
Factor out the square root of $(2a)^2$
\[ x = - \frac{b}{2a} \ \pm \ \frac{1}{2a} \sqrt{(-b)^2 - 4ac} \]
And then combine everything over the common denominator so finally
\[ x = \frac{- b \ \pm \ \sqrt{b^2 - 4ac}}{2a} \]

This is the quadratic formula which has been memorized by generations of algebra students.  I really prefer
\[ x = m \pm \ \sqrt{m^2 - c/a} \]
but it's the same thing.

\subsection*{alternative approach}
There is a different way to get the quadratic formula.  We said that $m$ lies exactly halfway between $s$ and $t$.  Let the distance from $m$ to $s$ (or to $t$) be $d$.  Then
\[ m - d = s \ \ \ \ \ \ \ m + d = t \]
\[ (m - d)(m + d) = st \]
\[ m^2 - d^2 = st \]
So we have a formula for $d^2$ in terms of $m$, $s$ and $t$.
\[ d^2 = m^2 - st \]

But remember we know that
\[ \frac{c}{a} = st \]
and said we'd come back?  So
\[ d^2 = m^2 - c/a \]
\[ d = \pm \ \sqrt{m^2 - c/a} \]
and then the roots are
\[ s = m - d = m - \sqrt{m^2 - c/a} \]
\[ t = m + d = m + \sqrt{m^2 - c/a} \]
And that, I hope, shows the meaning of the quadratic formula.

The roots lie an equal distance $d$ on either side of the vertex at $m$, and that distance is given by what's under the square root.

\subsection*{complex numbers}
Let us take a look at what is under the square root.  Here, it is probably simpler to use $a,b$ and $c$.  

The expression we have is called the \emph{discriminant}
\[ D = b^2 - 4ac \]

Depending on the particular example, $D$ may be positive, negative, or even zero.  And if it is the case that $D < 0$, then we'll have the square root of a negative number.

There are no real numbers with that property.  And this corresponds to the case where the graph does not cross the $x$-axis.  Then, there are no roots.

Notice that if $a > 0$, so the graph opens up, then by making $c$ more and more positive we can eventually make $4ac > b^2$.  We can always shift the graph up above the $x$-axis by adding more to $c$.

The way we explain these weird roots is to define $i$ as a special number with the property $i = \sqrt{-1}$, or equivalently $i^2 = -1$, so then if $D < 0$, let's say
\[ D = -p^2 \]
where $p^2$ is a positive real number, then
\[ \sqrt{D} = ip \]
So we have that the roots are
\[ s = m - ip \]
\[ t = m + ip \]
Take a look at $s$ times $t$
\[ st = (m + ip)(m - ip) \]
\[ = m^2 - i^2 p^2 \]
but remember $i^2 = -1$ so
\[ st = m^2 + p^2 \]

and we get the standard form as
\[ 0 =  (x - s)(x - t) \]
\[ = x^2 - (s + t)x + st \]
$m$ is the average of $s$ and $t$, meaning that $2m = s + t$, and we computed just now $st = m^2 + p^2$ so
\[ 0 = x^2 - 2m + m^2 + p^2 \]
\[ = (x - m)^2 + p^2 \]
The $i$ has gone away.

This is a perfectly valid equation for $y$
\[ y = (x - m)^2 + p^2 \]
but it only has solutions as long as $y >= 0$!  It is possible to have $m$ and $p$ both zero, so then $y = x^2$.  But there is no way to have $y < 0$ because it is the sum of two squares, which are both either $0$ or positive but never negative.

Which is another way of saying that the graph does not cross the $x$-axis.  There is no $x$ such that $y < 0$.

\subsection*{shifting the vertex}

We won't prove it, but just say that any quadratic (in fact, any formula) can be re-written in a form like:
\[ (y - k) = a(x - h)^2 \]
where $a$ is the shape factor, as usual, and $(h,k)$ is the point where the vertex lies.  (We abandon our favorite symbol $m$ for this part).

Multiplying out
\[ y - k = ax^2 - 2ahx + ah^2 \]
\[ y = ax^2 - 2ahx + ah^2 + k \]

by comparison with the standard form
\[ y = ax^2 + bx + c \]
we have that 
\[ b = -2ah \]
\[ h = - \frac{b}{2a} \]
Which matches what we said before.  The $x$-value at the vertex is 
\[ x = h = - \frac{b}{2a} \]

\subsection*{intersection}

You have a parabola and a line and want to know where (or if) they intersect.  Write the equation of the line as
\[ y = kx + y_0 \]
To make life easier, we will consider a parabola that is translated to have its vertex at the origin so
\[ y = ax^2 \]

We are interested in the $x$-values that give equal $y$.  So
\[ kx + y_0 = ax^2 \]

Gather like terms
\[ 0 = ax^2 + - kx - y_0 \]
We have a quadratic.  So 
\[ m = - \frac{-k}{2a} = \frac{k}{2a} \]

And the roots are
\[ x = \frac{k}{2a}  \ \pm \ \sqrt{(\frac{k}{2a} )^2} \]
\[ = \frac{k}{2a}  \pm \frac{k}{2a}  \]
\[ = 0, \frac{k}{a} \]

The interesting thing is when there is only a single root, a single intersection.  That happens when the line is the tangent to the parabola at the point of intersection.

There
\[ x = \frac{k}{2a} \]
\[ k = 2ax \]
The slope of the tangent line is $2ax$, which matches the result given by calculus.




\end{document}