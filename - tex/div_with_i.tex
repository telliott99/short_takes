\documentclass[11pt, oneside]{article} 
\usepackage{geometry}
\geometry{letterpaper} 
\usepackage{graphicx}
	
\usepackage{amssymb}
\usepackage{amsmath}
\usepackage{parskip}
\usepackage{color}
\usepackage{hyperref}

\graphicspath{{/Users/telliott/Dropbox/Github-math/figures/}}
% \begin{center} \includegraphics [scale=0.4] {gauss3.png} \end{center}

\title{Long division}
\date{}

\begin{document}
\maketitle
\Large

%[my-super-duper-separator]

\[ \frac{1 + i}{2 + i} = \ \text{?} \]

\subsection*{method 1}
Multiply by the fraction formed from the complex conjugate of the denominator:

\[ \frac{1 + i}{2 + i} \cdot \frac{2 - i}{2 - i} = \frac{3 + i}{5} \]

The length of the resulting vector is $\sqrt{10}/5 = \sqrt{2/5}$.

A slightly different approach is to use the inverse.  $z^{-1}$ is the inverse of $z$ if and only if
\[ z \cdot z^{-1} = 1, \ \ \ \ \ \ z^{-1} = \frac{1}{z} \]
so
\[ \frac{w}{z} = w \cdot z^{-1} \]

What is the inverse of $2 + i$?  It is almost $2 - i$ since
\[ (2 + i)(2 - i) = 5 \]
The inverse of $2 + i$ is $(2 - i)/5$.  And you can see that in the calculation that we did on the first line.

\subsection*{method 2}
Write the numbers in $r, \theta$ notation as
\[ \sqrt{2} \ e^{\pi/4}, \ \ \ \ \ \sqrt{5} \ e^{\theta}, \text{ with } \theta = \tan^{-1} 1/2 \]

Dividing $r$-values gives us the same length as before.  Subtracting the angles
\[ \tan^{-1} 1 -  \tan^{-1} 1/2 \stackrel{?}{=}  \tan^{-1} 1/3 \]

This is one of Gardner's problems.
\begin{center} \includegraphics [scale=0.35] {gardner12.png} \end{center}
Focusing on point $O$, it is clear that the central angle is $45^{\circ}$ (because $\triangle OBC$ is isosceles and $\angle OBC$ is a right angle), so the sum of the other two angles is also $45^{\circ}$.

\subsection*{method 3}

The last approach is long division.  I'm not going to try to typeset this.

\url{https://twitter.com/MrHonner/status/1587593196606980097}

But let's just think about it.  We're trying to divide $2 + i$ \emph{into} $1 + i$.  So the first factor is $1/2$ and we would write below $1 + i/2$ as what we need to subtract.  The remainder is $i - i/2 = i/2$.  

The next step is to divide $2 + i$ into $i/2$.  The factor is $i/4$ and what we're subtracting is $-1/4$.

Next, divide $2 + i$ into $1/4$.  The factor is $1/8$ and what we're subtracting is $i/8$.

If you continue this, you will obtain two infinite series (see the web page)
\[ S_r = \frac{1}{2} + \frac{1}{8} - \frac{1}{32} + \frac{1}{128} + \dots \]
\[ S_i = i ( \frac{1}{4} - \frac{1}{16} + \frac{1}{64} - \frac{1}{256} + \dots ) \]

The sign of the first term of $S_r$ is not quite right, but after that we have geometric series with the same common ratio $r = -1/4$.

Since $-1 < r < 1$, the series converge.  

There are various ways to remember the formula for the sum.  What I think of is to make the first term of the series equal to $1$.  

Here we must adjust the sign of the first term before doing the rest.  

\[ S_r = \frac{1}{2} + \frac{1}{8} - \frac{1}{32} + \frac{1}{128} + \dots \]
Add $-1$ on both sides:
\[ S_r - 1 = - \frac{1}{2} + \frac{1}{8} - \frac{1}{32} + \frac{1}{128} + \dots \]

Make the first term $1$
\[ (-2) \cdot (S_r - 1) = 1 - \frac{1}{4} + \frac{1}{16} - \frac{1}{64} \dots \]
Now, the right-hand side is just 
\[ \frac{1}{1 - r} = \frac{1}{5/4} = \frac{4}{5} \]
so, putting things together
\[ (-2) \cdot (S_r - 1) = \frac{4}{5} \]
\[ S_r - 1 = - \frac{2}{5} \]
I get $S_r = 3/5$, as expected.

For the second series we start with
\[ S_i = i ( \frac{1}{4} - \frac{1}{16} + \frac{1}{64} - \frac{1}{256} + \dots ) \]
Remembering the equality $-i \cdot i = 1$
\[ (-4i) \cdot S_i = 1 - \frac{1}{4} + \frac{1}{16} - \frac{1}{64} \dots \]
\[ = \frac{4}{5} \]
Applying the equality again, we obtain:
\[ S_i = \frac{i}{5} \]

Putting the two results together we have that the result of the long division is $(3 + i)/5$, which matches the other two methods.

\end{document}