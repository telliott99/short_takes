\documentclass[11pt, oneside]{article} 
\usepackage{geometry}
\geometry{letterpaper} 
\usepackage{graphicx}
	
\usepackage{amssymb}
\usepackage{amsmath}
\usepackage{parskip}
\usepackage{color}
\usepackage{hyperref}

\graphicspath{{/Users/telliott/Github-Math/figures/}}
% \begin{center} \includegraphics [scale=0.4] {gauss3.png} \end{center}

\title{Heron's formula}
\date{}

\begin{document}
\maketitle
\Large

%[my-super-duper-separator]

In this short write-up we will derive Heron's formula for the area of a triangle in terms of the \emph{semi-perimeter}.  Start by drawing the altitude in any triangle (if obtuse, use the vertex with the obtuse angle).
\begin{center} \includegraphics [scale=0.2] {heron3.png} \end{center}

\[ x^2 + h^2 = b^2 \]
\[ (a - x)^2 + h^2 = c^2 \]
Subtract the first from the second
\[  (a - x)^2  - x^2 = c^2 - b^2 \]
\[ a^2 - 2ax = c^2 - b^2 \]
\[ c^2 = a^2 + b^2 - 2ax \]
Straightforward, this is the Pythagorean theorem with a correction term.

Let $\angle C$ be the angle opposite side $c$.  Since $x = b \cos C$:
\[ c^2 = a^2 + b^2 - 2ab \cos C \]
The law of cosines.  

However, we will leave our result as $2ax$ and move on to write an expression for the area $A$.  Going back to the right triangle:
\[ h^2 = b^2 - x^2 \]
\[ h = \sqrt{b^2 - x^2} \]
\[ = \sqrt{b^2 - (\frac{a^2 + b^2 - c^2}{2a})^2} \]
Now find the area (or twice that):
\[ 2A = ah \]
So
\[ 2A = a \sqrt{b^2 - (\frac{a^2 + b^2 - c^2}{2a})^2} \]

This seems pretty complicated, but notice, we have $a^2$ on the bottom under a square root, so we can cancel the leading factor of $a$.

Put the $b^2$ term on top of a common denominator:
\[ 2A = a \sqrt{\frac{4a^2b^2}{4a^2} - (\frac{a^2 + b^2 - c^2}{2a})^2} \]
\[ 4A = \sqrt{4a^2b^2 - (a^2 + b^2 - c^2)^2} \]
\[ 16A^2 = 4a^2b^2 - (a^2 + b^2 - c^2)^2 \]

Now comes the part that makes this derivation beautiful.  We will use the difference of squares:
\[ 16A^2 = \ [ \ 2ab + (a^2 + b^2 - c^2) \ ] \ [ \ 2ab - (a^2 + b^2 - c^2) \ ]  \]
\[  = \ [ \ (a + b)^2 - c^2 \ ] \ [ \ c^2 - (a - b)^2 \ ]  \]
and then again
\[ 16A^2 = (a + b + c)(a + b - c)(c - (a - b))(c + a - b) \]
\[ = (a + b + c)(a + b - c)(c - a + b)(c + a - b) \]

We're basically done.  The semi-perimeter $s$ is
\[ s = \frac{a + b + c}{2} \]
\[ 2s = a + b + c \]
Thus
\[ 2(s-c) = a + b - c \]
and so on.

We had
\[ 16A^2 = (a + b + c)(a + b - c)(c - a + b)(c + a - b) \]
So
\[ A^2 = s (s - a)(s - b)(s - c) \]
\[ A = \sqrt{s (s - a)(s - b)(s - c)} \]
$\square$

This formula for the area of a triangle in terms of the three sides is ascribed to Hero of Alexandria.

\end{document}