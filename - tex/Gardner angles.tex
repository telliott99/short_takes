\documentclass[11pt, oneside]{article} 
\usepackage{geometry}
\geometry{letterpaper} 
\usepackage{graphicx}
	
\usepackage{amssymb}
\usepackage{amsmath}
\usepackage{parskip}
\usepackage{color}
\usepackage{hyperref}

\graphicspath{{/Users/telliott/Dropbox/Github-math/figures/}}
% \begin{center} \includegraphics [scale=0.4] {gauss3.png} \end{center}

\title{Gardner's angles}
\date{}

\begin{document}
\maketitle
\Large

%[my-super-duper-separator]

Here is a puzzle I came across on the internet.

\url{https://www.theguardian.com/science/alexs-adventures-in-numberland/2014/oct/21/martin-gardner-mathematical-puzzles-birthday}

It is attributed to Martin Gardner.

\begin{center} \includegraphics [scale=0.25] {gardner1.png} \end{center}

\emph{To prove}.

Use elementary geometry to show that $\angle A + \angle B = \angle C$.

\subsection*{geometric solutions}
I found another version of this problem on the web.  Given three identical squares arranged as follows:
\begin{center} \includegraphics [scale=0.6] {gardner7.png} \end{center}
\[ \angle AOC + \angle BOC = \ \text{?} \]

The solution is to be obtained without measurements or using trigonometry.

As in so many problems, the key is to draw an inspired diagram, one that extends the figure somehow.  In this one, a major hint was provided, namely, a grid of squares.
\begin{center} \includegraphics [scale=0.45] {gardner6b.png} \end{center}

So let's draw the same two angles on that grid and form a triangle.  The right triangle with $\angle A$ (black) has a base of $3$ and opposite side of $1$, while that with $\angle B$ (red) has a base of $2$ and opposite side of $1$.
\begin{center} \includegraphics [scale=0.45] {gardner8.png} \end{center}
Connecting the distant points, we form a second triangle with $\angle A$, congruent to the first.

We can fill in many of the other angles using the theorem on alternate interior angles, as well as the fact that $AC$ and $CB$ are the diagonals of congruent rectangles.
\begin{center} \includegraphics [scale=0.45] {gardner9.png} \end{center}

Notice that the angle at $C$ is basically a rotated version of one vertex of a square, that is, a right angle.  

$\circ$ \ State this proof using sums of angles.

$\circ$ \ Furthermore $\angle CAB = \angle CBA$.  Why?

$\circ$ \ Thus, two copies of $\angle A + \angle B$ sum to a right angle.  Why?

So, the sum of one of each is one-half of a right angle.

$\square$

For a second geometric approach we draw a different rectangle.
\begin{center} \includegraphics [scale=0.25] {gardner3.png} \end{center}

The long side of the rectangle is equal to $2 \sqrt{2}$ and is twice the short side, which is just the diagonal of the original unit square.

Therefore, the right triangle containing $\angle D$ is similar to the right triangle containing $\angle B$.  We conclude that $\angle B = \angle D$.

Since $\angle A + \angle D = \angle C$, it follows that $\angle A + \angle B = \angle C$.

$\square$

\subsection*{algebra}

Let's label some other angles.

\begin{center} \includegraphics [scale=0.30] {gardner2.png} \end{center}

I see two equations we can write (using alternate interior angles):
\[ B = A + x \]
\[ C = A + x + y = B + y \]

If we can show that $A = y$, then it follows that $B = x + y$ so $C = A + B$.

Notice that the sides of a right triangle containing $\angle A$ (blue) are in the ratio $1:3$.

\begin{center} \includegraphics [scale=0.5] {gardner10.png} \end{center}

Draw additional squares to extend the two adjacent sides of $\angle y$ (red).  With the appropriate scaling, we can draw a right triangle containing $\angle y$, where the opposite side has a length equal to the diagonal of a square, and the base has a length $3$ times that.  
\begin{center} \includegraphics [scale=0.5] {gardner11.png} \end{center}

That is, the ratio is $1:3$.  This is the same as for the right triangle containing $\angle A$.  Since the two triangles are similar, the angles are equal.

$\square$

One can carry out the same proof with a simpler diagram.  Divide the unit square into four small squares with side lengths equal to one-half.  Draw the diagonals in one of those smaller squares.

\begin{center} \includegraphics [scale=0.4] {gardner5.png} \end{center}

The resulting small triangles, formed by crossing diagonals in a square, are all congruent.  We don't need to do any algebra.

Thus, $a$ is one-quarter of the length of the long diagonal.  If we were doing algebra we would say that $4a = \sqrt{2}$ so $a = 1/2 \sqrt{2}$.

The angle which we called $y$ before, formed here by line segments in red, lies in a right triangle with opposite side $a$ and adjacent side $3a$, so its tangent is $1/3$.  

This is the same as the tangent of the angle we called $A$ before, which is formed by the blue line segment and the base of the large square.  

The two marked angles are part of similar right triangles, so they are equal.

\subsection*{sum of angles:  tangents}

Let each square be a unit square.  Clearly $\tan C = 1, \tan B = 1/2$ and $\tan A = 1/3$.

Recall the sum of angles formulas:
\[ \sin s + t = \sin s \cos t + \cos s \sin t \]
\[ \cos s + t = \cos s \cos t - \sin s \sin t \]

So
\[ \tan s + t = \frac{\sin s \cos t + \cos s \sin t}{\cos s \cos t - \sin s \sin t} \]
Multiply through by $1/\cos s \cos t$ on top and bottom to obtain
\[ \tan s + t = \frac{\tan s + \tan t}{1 - \tan s \tan t} \]

\[ \tan A = \frac{1/2 + 1/3}{1 - 1/2 \cdot 1/3} \]
which does indeed equal $1$.  

Since the inverse tangent is a single-valued function over the interval $[0,\pi/2]$, and $\tan A = \tan B + C$, it follows that $\angle A = \angle B + \angle C$.

\end{document}