\documentclass[11pt, oneside]{article} 
\usepackage{geometry}
\geometry{letterpaper} 
\usepackage{graphicx}
	
\usepackage{amssymb}
\usepackage{amsmath}
\usepackage{parskip}
\usepackage{color}
\usepackage{hyperref}

\graphicspath{{/Users/telliott/Dropbox/Github-math/figures/}}
% \begin{center} \includegraphics [scale=0.4] {gauss3.png} \end{center}

\title{Similar triangles}
\date{}

\begin{document}
\maketitle
\Large

%[my-super-duper-separator]

\subsection*{similarity and scaling}

From Jacobs, chapter 10.  

\begin{center} \includegraphics [scale=0.35] {Jacobs10b.png} \end{center}

If two triangles are similar, all three angles are the same.  Also, the three corresponding pairs of sides are in the same proportions, each re-scaled by the same constant factor.

\begin{center} \includegraphics [scale=0.4] {similar2.png} \end{center}

From the above diagram of two similar triangles, similarity implies that (for example)
\[ \frac{a}{A} = \frac{b}{B} \]

For any pair of similar triangles, there is a constant $k$ such that

\[ k = \frac{a}{A} = \frac{b}{B} = \frac{c}{C} \]

A slight rearrangement gives:

\[ \frac{a}{b} = \frac{A}{B} \]

The second set of ratios is different and has its own constant, say, $k'$.

\subsection*{similar right triangles}

For similar right triangles, all angles equal implies equal ratios of sides.  Our approach to proving this result is from Acheson and is based on an observation about area.

\begin{quote}Draw a rectangle ABCD, and a diagonal AC.  Then pick a point E on the diagonal and draw lines through it parallel to both sides.\end{quote}

\begin{center} \includegraphics [scale=0.6] {Acheson_G42.png} \end{center}

All of the right triangles in the figure are similar.  (\emph{Proof}:  use some combination of the alternate interior angles theorem, complementary angles in a right triangle and vertical angles).  

By changing the height of the figure, we can obtain any two complementary angles we wish.  And by changing the placement of $E$ we can get any ratio we like.

The two shaded rectangles are bisected by the diagonal $AEC$ (\emph{Proof}:  this is a basic property of rectangles;  we have congruent $\triangle$ by SSS).  So the two light-gray triangles have equal area, and the two dark gray ones do as well.

But $\triangle ABC$ and $\triangle ADC$ also have equal area.

Therefore, we just subtract equal areas to find that the two unshaded rectangles above and below the diagonal are equal in area.  The one on top has area $bc$ and the one below has area $ad$.  We have

\[ bc = ad \]
\[ \frac{a}{c} = \frac{b}{d} \]

$\square$

A bit of algebra gives:
\[ \frac{a}{c} + \frac{c}{c} = \frac{b}{d} + \frac{d}{d} \]
\[ \frac{a + c}{c} = \frac{b + d}{d} \]
\[ \frac{a + c}{b + d} = \frac{c}{d} = \frac{a}{b} \]

\begin{center} \includegraphics [scale=0.6] {Acheson_G42.png} \end{center}

\subsection*{hypotenuse in proportion}

It is natural to ask, what about the hypotenuse?

\begin{center} \includegraphics [scale=0.5] {similar18.png} \end{center}

By the Pythagorean theorem, for a right triangle with sides $a$ and $b$ and hypotenuse $g$:
\[ a^2 + b^2 = g^2 \]
and
\[ c^2 + d^2 = h^2 \]

We will show that
\[ \frac{a}{c} = \frac{b}{d} = \frac{g}{h} \]

\emph{All} of the sides of two similar right triangles have the same ratio.  

We must be careful, however.  There is a deep connection between similarity, area and the Pythagorean theorem.  It is important that we will have Euclid's proof of the theorem, which uses SAS and does not depend on similarity.

\emph{Proof}.

Start with 
\[ \frac{a}{c} = \frac{b}{d} = k \]
\[ a = kc, \ \ \ \ \ \ b = kd \]
\[ a^2 + b^2 = k^2c^2 + k^2d^2 \]
\[ g^2 = k^2h^2 \]

Since these are lengths, we can take the positive square root and obtain

\[ \frac{g}{h} = k \]
\[ = \frac{a}{c} = \frac{b}{d} \]

$\square$

Thus, AAA similarity is established for right triangles.  If either of the smaller angles matches between two right triangles, then they are not only similar but all the side lengths are in the same ratio as well.

\subsection*{general case}

For the general case, we first look at an abbreviated version of Euclid's proof (from VI.2).

In $\triangle ABC$ (below), let $DE$ be drawn parallel to $BC$, with $AD$ not necessarily equal to $DB$.  Then, we claim that $AD$ is to $DB$ as $AE$ is to $EC$.

\begin{center} \includegraphics [scale=0.6] {Euclid_VI_2.png} \end{center}

\emph{Proof.}

$\triangle BDE$ and $\triangle CDE$ have the same area, because they share the base $DE$ and the vertices $B$ and $C$ lie on the same line $BC$, which is parallel to $DE$.  Therefore, the altitudes are equal.  
\[ (\triangle BDE) = (\triangle CDE) \]

\begin{center} \includegraphics [scale=0.35] {similar21.png} \end{center}

When added, separately, to $\triangle ADE$, the resulting triangles also have the same area
\[ (\triangle ABE) = (\triangle ACD) \]

Next, we compute the areas in terms of the side lengths.

\begin{center} \includegraphics [scale=0.35] {similar22.png} \end{center}

If the altitude from $E$ to base $AB$ has length $h$, then twice $(\triangle ABE) = h \cdot AB$.  And if the altitude from $D$ to base $AC$ has length $k$, then twice $(\triangle ACD) = k \cdot AC$.  

But these areas are equal, which means that
\[ h \cdot AB = k \cdot AC \]

The area of $\triangle ADE$ can be figured in two different ways as
\[ h \cdot AD = k \cdot AE \]

Dividing, we obtain
\[ \frac{AB}{AD} = \frac{AC}{AE} \]

$\square$

\subsection*{all triangles}

An alternative proof for the general case builds on the result for right triangles.

Any triangle can be decomposed into two right triangles.  So, combining the results for the two sub-triangles, we will have the result for the general case.  

Start with two triangles similar because the angles are the same (left panel).  

\begin{center} \includegraphics [scale=0.35] {similar13.png} \end{center}

Make a copy of the smaller triangle and rotate it and then attach at the top, forming a parallelogram.  The original small triangle and the rotated version are congruent by our construction.

Now, draw the two altitudes, label the sides, and suppress the labels for the angles but just mark them with colored circles.

\begin{center} \includegraphics [scale=0.35] {similar14.png} \end{center}

We have two different pairs of similar right triangles.  

We have
\[ \frac{a}{h} = \frac{a'}{h'} \]
\[ \frac{b}{h} = \frac{b'}{h'} \]

So then
\[ \frac{h}{h'} = \frac{a}{a'} =  \frac{b}{b'} \]

There is nothing special about this pair of sides, we could have chosen any other pair, either $a$ and $c$ or $b$ and $c$, and have the same result.

Therefore if any two triangles have three angles the same, the side lengths are all in the same proportion.

$\square$

\end{document}