\documentclass[11pt, oneside]{article} 
\usepackage{geometry}
\geometry{letterpaper} 
\usepackage{graphicx}
	
\usepackage{amssymb}
\usepackage{amsmath}
\usepackage{parskip}
\usepackage{color}
\usepackage{hyperref}

\graphicspath{{/Users/telliott/Dropbox/Github-math/figures/}}

\title{Cycloid}
\date{}

\begin{document}
\maketitle
\Large

%[my-super-duper-separator]

We imagine a bicycle with one tire marked at a particular point on the rim, say with fluorescent paint or a small light.  As the tire rotates our fixed point $P$ on the rim traces a curve called the cycloid.
\begin{center} \includegraphics [scale=0.6] {cycloid.png} \end{center}

In calculus, we will find equations that give the position of the point $P$ as a function of time, or the angle of a wheel of radius $a$ turning at constant speed.

\begin{center} \includegraphics [scale=0.5] {cycloid2.png} \end{center}

It turns out that there is a way to calculate the area under the curve of the cycloid without using calculus.  

It may help to give the answer from calculus first.  The area is 

\[ A = 3 \pi a^2 \]

where $a$ is the radius of the circle.  This by itself, suggests there may be a simple solution.  

We note that the area of a rectangle that just contains the circle, rolled its whole length, has height $2a$ and length $2 \pi a$.  The area of the enclosing box is then $4 \pi a^2$.  

Where shall we find $\pi a^2$ to subtract to give the answer?

\begin{center} \includegraphics [scale=0.6] {Apostol1_3.png} \end{center}

I think you can see the idea in the figure below.

\begin{center} \includegraphics [scale=0.25] {Mamikon_cycloid.png} \end{center}

Tangents are constructed to the curve on a regular interval.  These are then arranged in the inner circle.  The area of the inner circle is just $\pi a^2$.

There more at wikipedia

\url{https://en.wikipedia.org/wiki/Visual_calculus}

\subsection*{Mamikon}

\includegraphics [scale=0.5] {Mamikon1_5.png} 
\includegraphics [scale=0.5] {Mamikon1_6.png} 

According to the book \emph{New Horizons in Geometry} by Tom Apostol and Mamikon, the latter's original insight came from the problem of the area of an annulus or ring.

If the tangent to the inner disk has length $a$, then the area of the ring is $\pi (a/2)^2$.  That is, it does not depend explicitly on either radius.

But of course, the area can also be calculated as the difference:
\[ A = \pi (R^2 - r^2) = \pi (a/2)^2 \]

The somewhat surprising result is that the areas of these two rings are the same, since they depend only on the length of the tangent.

\begin{center} \includegraphics [scale=0.35] {Mamikon_wiki_1.png} \end{center}
\begin{center} \includegraphics [scale=0.35] {Mamikon_wiki_2.png} \end{center}

So then, the idea is that this area is also equal to the area of a disk with radius equal to the length of the tangent, where the angular measure for each sector on the right is equal to the measure of the outer ring on the left, divided by the total $2 \pi$.

\begin{center} \includegraphics [scale=0.7] {Mamikon1_7.png} \end{center}

There is a lot more in the book, which I saw the first chapter of by using Google Books.  For example Apostol and Mamikon extend the theorem to figures with unequal tangent lengths. 

Here is a proof that the area under $y = x^2$ between $0$ and $x$ is equal to $x^3/3$.

\begin{center} \includegraphics [scale=0.4] {Apostol1_19.png} \end{center}

Compare these tangents to those we drew in the chapter on parabolas.

They develop a theorem that the area goes like the square of the length of the tangent.  

So the region $S + T$ under the curve and above the horizontal line between $0$ and $x$ is equal in area to the region covered by tangents of twice the original length, $4S$, which means that the area of $S+T$ is quadruple that of $S$.

We have

\[ 4S = S + T \]
\[ S = \frac{1}{3} T \]

The area of the rectangular box $R$ is $4$ times $T$, while the region under the whole curve is $4$ times $S$, hence the fraction of $R$ that is under the curve is 
\[ \frac{4 S}{4 T} = \frac{S}{T} = \frac{1}{3} \].  Of course, $R = x^3$.

\begin{center} \includegraphics [scale=0.4] {Apostol1_19.png} \end{center}

\end{document}