\documentclass[11pt, oneside]{article} 
\usepackage{geometry}
\geometry{letterpaper} 
\usepackage{graphicx}
	
\usepackage{amssymb}
\usepackage{amsmath}
\usepackage{parskip}
\usepackage{color}
\usepackage{hyperref}

\graphicspath{{/Users/telliott/Dropbox/Github-math/figures/}}
% \begin{center} \includegraphics [scale=0.4] {gauss3.png} \end{center}

\title{Bertrand's paradox}
\date{}

\begin{document}
\maketitle
\Large

%[my-super-duper-separator]

Grinstead and Snell's wonderful \emph{Introduction to Probability} has this problem (example 2.6).  It's called Bertrand's paradox.  We are asked to draw a chord of a unit circle randomly.

\begin{center} \includegraphics [scale=0.6] {Bertrand1.png} \end{center}

Here we might say, let's choose each of three angles $\alpha$, $\beta$ and $\theta$ randomly (uniform density) from $[0, 2 \pi]$.  

But there is no reason why the radius to $B$ cannot lie along the $x$-axis, so there are really only two choices.  

The question is posed:  what is the probability that the length of this random chord is $> \sqrt{3}$.

However, there are several different approaches to parametrize the problem, and randomizing the different parameters leads to different results.

\subsection*{equilateral triangles}

Let's review briefly some properties of equilateral triangles.  We drop an altitude and observe the ratio of side lengths.  It is convenient to start with a side length of 2 for the original triangle, then in the bisected copies the sides are in the ratio 1-2-$\sqrt{3}$ (the $1$ by bisection, and $\sqrt{3}$ by the Pythagorean theorem).

The angle at each vertex of the original equilateral triangle is $\pi/3$, so the new triangles have angles of $\pi/6$, both by bisection or because the altitude forms an angle of $\pi/2$ at the base, so the sum of angles theorem gives us the last angle.

In the right panel, the equilateral triangle is inscribed in a unit circle, so $OR = OP = OQ = 1$.  We claim that the line segment $OM$ has a length of $1/2$.

\begin{center} \includegraphics [scale=0.5] {Bertrand2.png} \end{center}

\emph{Proof}.

$\angle PQR$ is a right angle, by Thales' circle theorem, and $\angle MRQ$ is shared, so $\triangle PQR$ is similar to $\triangle RMQ$.  Therefore, $\angle RPQ = \angle MQR = \pi/3$.

Therefore the sides of $\triangle PQR$ are also in the ratio 1-2-$\sqrt{3}$, with $PQ/PR = 1/2$ and so $PQ = OP = OQ$.  Thus, $\triangle OPQ$ is isosceles.

$QM \perp OP$ so $MQ$ is the bisector of both $\angle PQO$ and the base $OP$.  

Therefore, $OM$ is one-half of $OP$ and has a length of $1/2$.

$\square$

\subsection*{first parametrization}

We have just shown that the altitude of the inscribed equilateral triangle in a unit circle has length $3/2$.  This means that the ratio of the inscribed triangle to the standard one is $\sqrt{3}/2$.

And that means the side length of the inscribed equilateral triangle is $\sqrt{3}$.

That explains the length chosen for the chord in this problem.  We see that if $M$ is chosen at random anywhere along $OP$, one-half of the time the chord formed will be larger than $\sqrt{3}$.

\begin{center} \includegraphics [scale=0.5] {Bertrand2.png} \end{center}

So the probability we are asked to give is just $1/2$.

\subsection*{second parametrization}

\begin{center} \includegraphics [scale=0.5] {Bertrand3.png} \end{center}

The second parametrization has the same triangle we just saw, $\triangle PQR$.  The angle at vertex $R$ is $\theta$.

$\theta$ can lie in the interval $[0, \pi/2]$, and in the event that $\theta < \pi/6$, the chord length $RQ > \sqrt{3}$.

The probability that the chord is greater than $\sqrt{3}$ in length is $1/3$, since $\pi/6$ is one-third of $\pi/2$.

\subsection*{third parametrization}

Finally, we imagine picking two coordinates $(x,y)$ at random from the interior of the circle.  We place the midpoint of the chord $M$ at $(x,y)$.

If $M$ is such that $r = \sqrt{x^2 + y^2} < 1/2$, then $M$ will be closer to the center of the circle than $1/2$ and so the chord length will be $> \sqrt{3}$.

The number of points that have this property is proportional to the relative areas of the inside small circle, and the larger circle around is.

\[ \frac{\pi (1/2)^2}{\pi} = \frac{1}{4} \]

We see that, depending on which parameter is randomized, we obtain a probability of $1/2$, $1/3$ or $1/4$.

In Jaynes' words, the problem is not well-formed.

\end{document}