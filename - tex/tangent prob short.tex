\documentclass[11pt, oneside]{article} 
\usepackage{geometry}
\geometry{letterpaper} 
\usepackage{graphicx}
	
\usepackage{amssymb}
\usepackage{amsmath}
\usepackage{parskip}
\usepackage{color}
\usepackage{hyperref}

\graphicspath{{/Users/telliott/Dropbox/Github-Math/figures/}}
% \begin{center} \includegraphics [scale=0.4] {gauss3.png} \end{center}

\title{Tangents}
\date{}

\begin{document}
\maketitle
\large
Let $T=(x,y)$ be the point where the tangent line meets a unit circle centered at the origin.  Write the equation of the tangent line in terms of its slope $-x/y$ as
\[ -\frac{x}{y} =  \frac{y - (-1)}{x - 1/2} \]
\[ y^2 + y = -x^2 + \frac{1}{2} x \]
Since $x^2 + y^2 = 1$ everywhere on the circle,
\[ y + 1 = \frac{1}{2} x  \]
This is a relationship between the $x$- and $y$- values at both tangent points, but it is clearly not the equation of the tangent line, confirmed since $(1/2,-1)$ is not a solution.  The trick here is to use the information from the circle \emph{again}:
\[ \sqrt{1 - x^2} = \frac{1}{2} x - 1 \]
\[ 1 - x^2 = \frac{1}{4}x^2 - x + 1 \]
\[ \frac{5}{4}x^2 - x = 0 \]
\[ (\frac{5}{4}x - 1)x = 0 \]
This has two solutions, $x = 0$ and $x = 4/5$, which are indeed the $x-$coordinates of the points of tangency as a more sophisticated analysis will confirm.  Then $y = \sqrt{1 - x^2} = -3/5$ (the minus sign because this is the fourth quadrant).  So the slope is 
\[ \frac{-x}{y} = \frac{-4/5}{-3/5} = \frac{4}{3} \]
The height of the triangle is then $2/3$ and the area is easily found to be $1/6$.
\end{document}
