\documentclass[11pt, oneside]{article} 
\usepackage{geometry}
\geometry{letterpaper} 
\usepackage{graphicx}
	
\usepackage{amssymb}
\usepackage{amsmath}
\usepackage{parskip}
\usepackage{color}
\usepackage{hyperref}

\graphicspath{{/Users/telliott/Dropbox/Github-Math/figures/}}

\title{Pentagon}
\date{}

\begin{document}
\maketitle
\Large

%[my-super-duper-separator]

In this chapter we explore some properties of a regular pentagon.  A pentagon has 5 sides, and a regular polygon has all sides equal.

The regular pentagon has five-fold rotational symmetry.  Draw all of the internal chords of the figure and label a few angles.

By rotational symmetry each of the five vertices of the pentagon has the same three components, the central one labeled $s$, and two flanking ones $r$ and $t$.  
\begin{center}
\includegraphics [scale=0.35] {pent1.png}
\includegraphics [scale=0.35] {pent2.png}
\end{center}

Mirror image symmetry shows that $r = t$, by reflection across the midline.  

Alternatively, we may invoke the properties of isosceles triangles, using two sides of the pentagon.  Hence we relabel the diagram (right panel).

Next, we compute two triangle sums:
\[ 3s + 2t = 4t + s \]
\[ 2s = 2t \]

Hence, $s = t$.  Relabel the diagram with black dots for the angles previously marked $s$, $r$ or $t$.

\begin{center}
\includegraphics [scale=0.35] {pent3b.png}
\end{center}

We observe that five copies of $s$ add up to one triangle (i.e. $\pi$).

Alternatively we might note that isosceles $\triangle BDO$ is similar to isosceles $\triangle AOE$ (vertical angles plus repeated angles equal to repeated angles).  Hence the base angles such as $\angle AEO$ are equal to $\angle s$.

From here, we might proceed by extending the base $ED$ past point $D$.  The newly formed angle will be supplementary to three copies of $s$, and therefore we have alternate interior angles when comparing it with $\angle ACD$.  

It follows that $AC \parallel ED$.

Therefore the angle with the magenta dot is worth $2s$ and the angle with the blue dot is worth $3s$.

Alternatively, compute two triangle sums:
\[ 5s = v + 2s, \ \ \ \ \ v = 3s \]
\[ 5s = 2u + s, \ \ \ \ \ u = 2s \]

Since $v = 3s$, its measure is the same as the vertex angle of the pentagon.  Thus the inner figure is also a regular pentagon.

\begin{center} \includegraphics [scale=0.3] {pent4.png} \end{center}

Our drawing is filled with regular parallelograms, with four sides equal (\emph{proof}:  use ASA to show that, e.g. $\triangle CEP \cong \triangle CDE$.  So we have two pairs of opposing sides equal).

One can draw two types of isosceles triangles using the chords and sides of the pentagon.  One is tall and skinny, with $2s$ at each of the base angles, while the other type is short and fat, with $s$ for the base angles.  

Here are three examples of the tall skinny type:
\begin{center} \includegraphics [scale=0.4] {three_triangles_2.png} \end{center}

If we look at the medium-sized blue (tall, skinny) triangle, we can scale them so that the base is equal to $1$ and then label the long sides as $x$.  The ratio of the long side to the base is $x:1$ or just $x$

But these triangles are all similar.  In particular, the large red ones have long sides equal to $x + 1$ and base equal to $x$.  The ratio of side lengths is the same for similar triangles, so form the ratio $\phi$ of the base to the long side (red on the left, blue on the right):

\[ \frac{x + 1}{x} = \frac{x}{1} \]
Does that look familiar?  Here is a picture from wikipedia.

\begin{center} \includegraphics [scale=0.3] {goldenratioab.png} \end{center}

We draw a square with sides $a$ and then extend two parallel sides to make a large rectangle and a small one at the same time.  We require the rectangles to have the same ratio of sides.
\[ \frac{a + b}{a} = \frac{a}{b} \]

Scale so that $b = 1$ and change notation, using $x$ instead of $a$:
\[ \frac{x + 1}{x} = \frac{x}{1} \]
This is our equation from above.

To solve it, rearrange:
\[ x^2 - x - 1 = 0 \]

Substitute into the quadratic equation:
\[ x = \frac{1 \pm \ \sqrt{5}}{2} \]

We are talking about a length.  Noticing that one of the solutions is negative, we ignore it for the moment, and take the positive branch of the square root.

$x$ is usually called $\phi$, the famous \emph{golden ratio}:
\[ \phi = \frac{1 + \sqrt{5}}{2} \]

Check that $\phi$ really does solve the equation:
\[ \phi^2 =  \frac{1 + \sqrt{5}}{2} \cdot  \frac{1 + \sqrt{5}}{2} \]
\[ = \frac{1}{4} (1 + 2 \sqrt{5} + 5) \]
\[ = 1 + \frac{2 + 2 \sqrt{5}}{4} = 1 + \phi \]
That checks.  

\subsection*{Aside on $\phi$ and the Fibonacci sequence}

The Fibonacci sequence is defined as $F_{n+2} = F_{n+1} + F_n$, starting with $1$.  

The first ten numbers in the sequence are:
\begin{verbatim}
1 1 2 3 5 8 13 21 34 55 ...
\end{verbatim}

Recall that $\phi^2 = 1 + \phi$.  The powers of $\phi$ generate an interesting pattern:
\[ \phi^2 = \phi \cdot \phi = 1 + \phi \]
\[ \phi^3 = \phi \cdot \phi^2 = \phi + \phi^2 = 1 + 2 \phi \]
\[ \phi^4 = \phi \cdot \phi^3 = \phi + 2 \phi^2 = 2 + 3 \phi \]
\[ \phi^5 = \phi \cdot \phi^4 = 2 \phi + 3 \phi^2 = 3 + 5 \phi \]

Both the first term and the cofactors generate the elements of the Fibonacci sequence from the powers of $\phi$

The reason is that $\phi^n + \phi^{n+1} = \phi^{n+2}$, which is the same as the definition for the Fibonacci numbers.

Going back to the solution that we left behind, take the negative branch of the square root, and let us call that other solution $\psi$.  

\[ \psi = \frac{1 - \sqrt{5}}{2} \]

If you look closely, you can easily see that
\[ \psi + \phi = 1 \]

Since $\psi$ is also a solution of the original equation:
\[ \psi^2 = 1 + \psi \]

Furthermore
\[ (\phi + \psi)^2 = \phi^2 + 2 \phi \cdot \psi + \psi^2 \]

Now, the left-hand side is just $1$, since $\phi + \psi = 1$.  Furthermore $\phi^2 = 1 + \phi$ and $\psi^2 = 1 + \psi$ so
\[ 1 = 1 + \phi + 2 \phi \cdot \psi + 1 + \psi \]
\[ 1 = 3 + 2  \phi \cdot \psi  \]
\[ \phi \cdot \psi = -1 \]

Thus, $\psi$ is the negative inverse of $\phi$.  

$\psi$ comes in handy in the following way.  Since $\psi$ solves our original equation, that means the powers of $\psi$ are just like the powers of $\phi$
\[ \psi^2 = 1 + \psi \]
\[ \psi^3 = 1 + 2 \psi \]
\[ \psi^4 = 2 + 3 \psi \]
\[ \psi^5 = 3 + 5 \psi \]

So
\[ \phi^5 - \psi^5 = 5(\phi - \psi) \]
\[ \frac{\phi^5 - \psi^5}{\phi - \psi} = 5 \]

$5$ is the fifth Fibonacci number.  If $F_n$ is the nth Fibonacci number
\[ \frac{\phi^n - \psi^n}{\phi - \psi} = F_n \]

This is called Binet's formula.  If you work out the denominator you find that it is just $\sqrt{5}$.  

The general equation is
\[ F_n = \frac{1}{\sqrt{5}} \cdot  (\phi^n - \psi^n) \]

This formula is quite surprising, because the Fibonacci numbers $F_n$ on the left-hand side are \emph{integers}, and yet the first factor on the right-hand side is the inverse of a square root, which is definitely not an integer or even a rational number.  

But it turns out that the differences $\phi^n - \psi^n$ contain only odd powers of $\sqrt{5}$.  So after multiplying by $1/\sqrt{5}$, we end up only with even powers, which are whole numbers.  

In fact, there is a connection between the Fibonacci sequence and Pascal's triangle.  
\begin{center} \includegraphics [scale=0.4] {fib_triangle.png} \end{center}

If we let $f = \sqrt{5}$ then Binet's formula says the nth Fibonacci number is equal to 
\[ (\frac{1}{2})^n \cdot \frac{1}{f} \cdot \ [ \ (1 + f)^n - (1 - f)^n \ ] \]

If you do the two binomial expansions, they are the same except that each term of the second one has a factor of $(-1)^n$.  As a result, the odd powers survive, as twice the value.  In the case of $n = 5$ we would have
\[ (\frac{1}{2})^5 \cdot \frac{1}{f} \cdot \ [ \ 10f + 20f^3 + 2f^5 \ ] \]
\[ (\frac{1}{2})^5 \cdot \ [ \ 10 + 20f^2 + 2f^4 \ ] \]

The coefficients of the powers of $f$ are twice the alternate coefficients in the binomial expansion for $(1 + f)^5$:  $5, 10$ and $1$.  If you work through more examples you'll see there is a cancellation that happens with $1/2^n$ so that this always results in an integer.  But this is probably a good place to stop.

\subsection*{note}

If you're trying to impress your friend, or a student, and reconstruct the above argument from the beginning, beware.  It has happened to me that I switched the roles of $a$ and $b$, or $x$ and $1$.  Then, you'll get a slightly different equation whose solution is the inverse of $\phi$.

\end{document}