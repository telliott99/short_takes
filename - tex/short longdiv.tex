\documentclass[11pt, oneside]{article} 
\usepackage{geometry}
\geometry{letterpaper} 
\usepackage{graphicx}
	
\usepackage{amssymb}
\usepackage{amsmath}
\usepackage{parskip}
\usepackage{color}
\usepackage{hyperref}

\graphicspath{{/Users/telliott/Dropbox/Github-Math/figures/}}
% \begin{center} \includegraphics [scale=0.4] {gauss3.png} \end{center}

\title{Decimals to fractions}
\date{}

\begin{document}
\maketitle
\Large

%[my-super-duper-separator]

For the integers between $2$ and $10$, finding decimal equivalents for the reciprocals, fractions like $1/2$ and $1/5$ and so on, is pretty simple ($0.5$ and $0.2$, respectively).

Even $1/4$ and $1/8$ ($0.25$ and $0.125$) aren't bad.  

$1/3$ and its multiples have the issue of repeating $3$'s.

\begin{verbatim}
     0.333333
 3 / 1.000000
\end{verbatim}

and we are left to ponder how to interpret the result that $3 \cdot 1/3 = 1$, but $3 \cdot 0.3333 = 0.9999$ repeating, if they are the same.  Of course the answer gets to the problem of how to define real numbers, not just rationals but also $\pi$ or $\sqrt{2}$.

The only moderately complicated result is for $7$.  Computing $1/7$ by long division:

\begin{verbatim}
     0.142857
 7 / 1.000000
       7
     ---
       30
       28
       --
        20
        14
        --
         60
         56
         --
          40
          35
          --
           50
           49
           --
           10
\end{verbatim}

For the next step, we have $10$ as the dividend so the result is just $0.142857$ repeating.

How to convert that decimal back to $1/7$?

The trick is to multiply by 10 to whatever power is needed to bring the multiple into alignment with the original.

\begin{verbatim}
          f =      0.142857
1000000 . f = 142857.142857
 999999 . f = 142857
 \end{verbatim}

 The repeat length has $6$ places and starts right after the decimal point so we need $10^6$.

\begin{verbatim}
f = 142857/999999
\end{verbatim}

The last step is to check our work.  We factor both the numerator and the denominator.  I get

\begin{verbatim}
142857 = 3.3.3. .11.13.37
------   ------------------
999999   3.3.3.7.11.13.37
\end{verbatim}

which works out correctly.  It is equal to $1/7$ as expected.

Here is a short Python function I will use to find factors (not very efficient since it tests more than primes):

\begin{verbatim}
def f(n):
    for i in range(2,n+1):
        while n % i == 0:
            print(i)
            n = n/i
        if n == 1:
            return
\end{verbatim}

When I do \begin{tt}{f(999999)}\end{tt} I get those factors.

An interesting thing about this problem is that the successive dividends are:  $10,30,20,60,40,50$ followed by $10$ again.  So if we compute say, decimal ($3/7$), we obtain a cyclically permuted version of the same pattern, just starting with \begin{tt}{428571}\end{tt} .  

All the possible numerators are there, $1$ through $6$.

If we have a multiple of $7$ in the denominator, like 

\begin{verbatim}
>>> 1/14
0.07142857142857142
\end{verbatim}

we have the same pattern with something extra in front.  Using the same trick with many fewer zeros

\begin{verbatim}
      f = 0.07142857
100 . f = 7.142857 = 7 + 1/7
 \end{verbatim}
 
No subtraction, just do $7 + 1/7 = 50/7$, then multiply by $1/100$ to obtain $f = 1/(2 \cdot 7) = 1/14$.

Some simple fractions have quite complicated decimals.  Consider $1/23$.  I use the Python decimal module to get enough precision:

\begin{verbatim}
import decimal
decimal.getcontext().prec = 50
d = 1/decimal.Decimal(23)
d
\end{verbatim}

output:

\begin{verbatim}
Decimal('0.043478260869565217391304347826086956521739130434783')
\end{verbatim}

22 digits in the repeat.  Can you see it?

\begin{verbatim}
>>> (int(1e22) * d) - d
Decimal('434782608695652173913.00000000000000000000000000000')
>>> int(_)
434782608695652173913
\end{verbatim}

So then we just check the result:

\begin{verbatim}
>>> x = 434782608695652173913
>>> y = 9999999999999999999999
>>> y/x
23.0
>>> y % x
0
\end{verbatim}

These two integers must have all the same factors except for an extra factor of 23 on the bottom.  But those are two pretty large numbers to factor.  

Based on what we saw above, we're going to have at least two factors of $3$ and then one of $11$, and of course there'll be at least one $23$ if we work with the denominator, so dividing I get 

\begin{verbatim}
>>> n = decimal.Decimal(9999999999999999999999)
>>> 3*3*11*23
2277
>>> n/_
Decimal('4391743522178304787')
\end{verbatim}

Working with this, I find another factor of $11$ and what's left is $399249411107118617$. 

However, then it gets hard!  I use the script from above to find that the next factor is $4093$.

All the rest will be larger than that.  The next is $8779$ and then what's left is $11111111111$.  The next factor is $21649$, and it leaves a remainder of $513239$, which is the last prime.  That's fairly exhausting.

To recap

\begin{verbatim}
>>> 3*3*11*11*23*4093*8779*21649*513239
9999999999999999999999
\end{verbatim}

The numerator is the same except it's missing the factor of $23$.

To be sure there's no mistake, I look at a table of primes.  They're all there.

\url{https://primes.utm.edu/lists/small/millions/}

It's interesting that the repeat length for $1/7$ is $6$ and that for $1/23$ is $22$.  

The length of the repeat for $1/17$ is also $16$, that for $1/19$ is $18$ and so on.  Yet the repeat for $1/11$ is only $2$ and for $1/13$ it's only $6$.  

There's an interesting discussion in the second answer here

\url{https://math.stackexchange.com/questions/377683/length-of-period-of-decimal-expansion-of-a-fraction}

In general, the repeat length of $1/n$ cannot be longer than $n-1$.  Often it is equal to that but sometimes not.  The reason is related to Euler's totient function.

 


\end{document}
