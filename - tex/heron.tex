\documentclass[11pt, oneside]{article} 
\usepackage{geometry}
\geometry{letterpaper} 
\usepackage{graphicx}
	
\usepackage{amssymb}
\usepackage{amsmath}
\usepackage{parskip}
\usepackage{color}
\usepackage{hyperref}

\graphicspath{{figures/}{/Users/telliott/Github-Math/figures/}}
% \begin{center} \includegraphics [scale=0.4] {gauss3.png} \end{center}

\title{Heron's proof}
\date{}

\begin{document}
\maketitle
\Large

%[my-super-duper-separator]

Here we go through Heron's proof of the eponymous theorem (as given by Paul Yiu).  I've stolen the diagram, but switched labels on a few of the points.  The triangle is $ABC$ with incircle radii $OD$, $OE$ and $OF$.
\begin{center} \includegraphics [scale=0.3] {heron2.png} \end{center}
$BT$ is drawn equal in length to $AD$.  Two right triangles are drawn on the diameter of a circle $LC$, so $B$ and $O$ lie on the circle.

As usual, the whole triangle is the sum of three pairs of equal triangles whose total base is the semiperimeter and altitude is equal to the radius $EO$.  $CT$ is drawn to be equal to the semiperimeter, hence the area of the triangle is $CT \cdot OE$.

The total angle at $L$ is equal to $\angle AOD$.  \emph{Proof}.  Because $L$ and $\angle BOC$ are opposing angles in a quadrilateral inscribed into a circle, they are supplementary.  But the components of $\angle BOC$ ($\angle BOE$ and $\angle COE$) are, together with $\angle AOD$, one-half the total angle at $O$.  

Thus $\angle BOC + \angle AOD$ are together equal to two right angles.  Hence $\angle L = \angle AOD$.

As a result, we have two pairs of similar right triangles:  $\triangle CBL \sim \triangle AOD$ and $\triangle LBK \sim EKO$ (since $OE \parallel BL$).
\begin{center} \includegraphics [scale=0.3] {heron2.png} \end{center}
We have to follow quite a long chain of ratios.

From the first pair of similar triangles, we have the ratio of bases as
\[ \frac{BC}{BL} = \frac{AD}{DO} \]
Rearranging:
\[ \frac{BC}{AD} = \frac{BL}{DO} \]
Substituting $BT = AD$ and $EO = DO$
\[ \frac{BC}{BT} = \frac{BL}{EO} \]

From the second pair we have
\[ \frac{BL}{BK} = \frac{OE}{KE} \]
\[ \frac{BL}{OE} = \frac{BK}{KE} \]
Hence
\[ \frac{BC}{BT} = \frac{BK}{KE} \]
The rest is relatively straightforward.

Adding $1$ to both sides
\[ \frac{BC+BT}{BT} = \frac{BK+KE}{KE} \]
\[ \frac{CT}{BT} = \frac{BE}{KE} \]
\begin{center} \includegraphics [scale=0.3] {heron2.png} \end{center}

Rearranging
\[ CT = BT \cdot \frac{BE}{KE} \]
Multiply top and bottom by $CE$:
\[ CT = BT \cdot \frac{BE}{KE} \cdot \frac{CE}{CE} \]
$\triangle COK$ is right, thus $OE^2 = CE \cdot KE$ and 
\[ OE^2 \cdot CT = BT \cdot BE \cdot CE \]
\[ OE^2 \cdot CT^2 = BT \cdot BE \cdot CE \cdot CT \]
\[ OE^2 \cdot CT^2 = AD \cdot BE \cdot CE \cdot CT \]
$\square$
This is Heron's theorem.

\end{document}