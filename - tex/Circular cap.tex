\documentclass[11pt, oneside]{article} 
\usepackage{geometry}
\geometry{letterpaper} 
\usepackage{graphicx}
	
\usepackage{amssymb}
\usepackage{amsmath}
\usepackage{parskip}
\usepackage{color}
\usepackage{hyperref}

\graphicspath{{/Users/telliott/Dropbox/Github-Math/figures/}}
% \begin{center} \includegraphics [scale=0.4] {gauss3.png} \end{center}

\title{Circular cap}
\date{}

\begin{document}
\maketitle
\Large

%[my-super-duper-separator]

I found a nice problem on Twitter.  The points on the circle are "evenly spaced."  The secant has the length shown, and we're asked for the shadedd area.
\begin{center} \includegraphics [scale=0.4] {circular_cap.png} \end{center}

I always start by recognizing that any problem like this can be re-scaled to the unit circle.  It will make the arithmetic less complicated, decreasing the chances of a mistake.

Let us consider the areas of an angular sector, followed by a circular cap.

For a sector of the circle swept out by a central angle $\theta$, the sector area or slice of the pie is the same fraction of the total area of the circle ($\pi$), as the ratio of $\theta$ to the whole circumference, $2 \pi$.
\[ \frac{A_{\text{sector}}}{\pi} = \frac{\theta}{2 \pi} \]
\[ A_{\text{sector}} = \frac{\theta}{2} \]

Next, for what we're calling a circular cap (the area between a secant of the circle and the circle itself), we have to subtract a triangle.  

That triangle is composed of two right triangles, each with a central angle of $\phi = \theta/2$ and sides $\sin \theta/2$, and $\cos \theta/2$.  So the area of the entire triangle for a sector of angle $\theta = 2 \phi$ is simply the product of the sine and cosine of $\phi$.

The area of the cap is the difference 

\[ A_{\text{cap}} = \frac{\theta}{2} - \sin \theta/2 \cos \theta/2 \]

We will rewrite this in terms of the half-angle $\phi = \theta/2$.  The area of the cap for angle $2 \phi$ is
\[ A = \phi - \sin \phi \cos \phi \]

In the problem we're given, one of the areas in question (the larger blue region) is not the area of a sector or a cap alone, but the difference between the area of the semi-circle and the cap.

\begin{center} \includegraphics [scale=0.4] {circular_cap.png} \end{center}

The points are equally spaced, with $12$ of them for the whole circle, so the angle corresponding to two adjacent points is $2 \pi/12 = \pi/6$.  The blue cap at the top has $\theta = \pi/3$, so $\phi = \pi/6$ and then

\[ A = \frac{\pi}{6} - \sin \frac{\pi}{6} \cdot \cos \frac{\pi}{6} \]

We recognize that $\sin \pi/6 \cdot \cos \pi/6 = \sqrt{3}/4$, but stay with what we have for a minute.

The cap consisting of the white portion plus the blue above it has twice the angle, so that cap is 

\[ A = \frac{\pi}{3} - \sin \frac{\pi}{3} \cdot \cos \frac{\pi}{3} \]

The blue region on the bottom is the difference between the area of a semicircle and the previous result
\[ A = \frac{\pi}{2} - \frac{\pi}{3} + \sin \frac{\pi}{3} \cdot \cos \frac{\pi}{3} \]

and the total blue area is

\[ A = \frac{\pi}{2} - \frac{\pi}{3} + \sin \frac{\pi}{3} \cdot \cos \frac{\pi}{3} + \frac{\pi}{6} - \sin \frac{\pi}{6} \cdot \cos \frac{\pi}{6} \]

In this problem, the angles are complementary, which means that $\sin \pi/3 = \cos \pi/6$ and the reverse, hence those terms cancel.  We are left with simply

\[ A = \frac{\pi}{2} - \frac{\pi}{3} + \frac{\pi}{6} =  \frac{\pi}{3} \]

The area is $1/3$ the area of the complete circle.

\subsection*{calculus}

It's interesting that the arithmetic above mirrors what is seen in the calculus solution.  This latter method provides us with a way to determine the area bounded by horizontal secants between any two angles.  The unshaded area would be 
\[ A = \int_{\pi/6}^{\pi/3} \cos^2 \theta \ d \theta \]

We justify this integral with the following picture.  Turn the previous view sideways by a quarter-turn to the right.

\begin{center} \includegraphics [scale=0.4] {circle_integral.png} \end{center}

The area under the curve is $\int y \ dx$.  For this parametrization of the circle (notice which angle is labeled in the figure):
\[ x = \sin \theta \]
\[ y = \cos \theta \]

So $dx = \cos \theta \ d \theta$ and then
\[ \int y \ dx = \int \cos^2 \theta \ d \theta \]

This integral is not one we can solve immediately.  But it turns out, when playing with different expressions, we may notice that the derivative of the product $\sin x \cos x$ is, by the product rule:

\[ [ \ \sin x \cos x \ ]' \ = \sin^2 x - \cos^2 x \]

and applying our favorite trigonometric identity we get

\[ \sin^2 x - \cos^2 x = 1 - 2 \cos^2 x \]
\[ \cos^2 x =  \frac{1}{2} - \frac{[\sin x \cos x]'}{2}  \]

Integrating
\[ \int \cos^2 x \ dx =  \frac{x}{2} - \frac{\sin x \cos x}{2}  \]

We have discovered that the integral of $\cos^2 x$ involves both $x$ and the product of the sine and cosine of $x$.  This is what we saw before in the geometrical approach.

The factor of $2$ comes in because this integral is actually the area between the curve of the circle and the axis, which is half the result we need.

Since $\phi_1$ and $\phi_2$ are complementary angles, the terms with $\sin x \cos x$ cancel, just as before, and we have

\[ A = x \ \bigg |_{\phi_2}^{\phi_1} = \phi_2 - \phi_1 = \frac{\pi}{6} \]

I am missing a factor of $2$.  The reason is that we have found the area that is unshaded and what we need is the area in blue.

That area is 

\[ A = \frac{\pi}{2} - \frac{\pi}{6}  = \frac{\pi}{3}   \]

\subsection*{re-scaling}

And now I notice that the problem does not contain a radius of $1$ but rather a particular secant of length $3$.

\begin{center} \includegraphics [scale=0.4] {circular_cap.png} \end{center}

We calculate the radius of this circle using the Pythagorean theorem.  Draw a 30-60-90 right triangle with one side equal to $3/2$ and the hypotenuse equal to $r$.  The other side is $3/2$ divided by $\sqrt{3}$, which means that

\[ r^2 = (3/2)^2 + (\sqrt{3}/2)^2 = \frac{12}{4} = 3 \]

The total area of the circle is $\pi$ times this and the shaded part is $1/3$ of it.  The final answer is just $\pi$.


\end{document}