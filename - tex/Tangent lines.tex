\documentclass[11pt, oneside]{article} 
\usepackage{geometry}
\geometry{letterpaper} 
\usepackage{graphicx}
	
\usepackage{amssymb}
\usepackage{amsmath}
\usepackage{parskip}
\usepackage{color}
\usepackage{hyperref}

\graphicspath{{/Users/telliott/Dropbox/Github-math/figures/}}
% \begin{center} \includegraphics [scale=0.4] {gauss3.png} \end{center}

\title{Tangents}
\date{}

\begin{document}
\maketitle
\Large

%[my-super-duper-separator]

Consider the graph of the curve $y = f(x)$.  Suppose we draw the tangent to the curve at two points, $x_1,y_1$ and $x_2,y_2$.  What are the coordinates of the point where they meet, if they do so?
\begin{center} \includegraphics [scale=0.4] {two_lines.png} \end{center}

If we draw the tangent line at a point $(x_0,y_0)$ on the curve, the slope is the derivative at that point, and the equation of the tangent line is
\[ \frac{y - y_0}{x - x_0} = m = f'(x_0) \]
\[ y = f'(x_0) (x - x_0) + y_0 \]
for every point $(x,y)$ on the line.

So for these two points and a function $f(x)$, we have two lines
\[ y = f'(x_1) (x - x_1) + y_1 \]
\[ y = f'(x_2) (x - x_2) + y_2 \]

The point where the two lines cross has the same coordinates $(x,y)$.  So
\[ f'(x_1) (x - x_1) + y_1 = f'(x_2) (x - x_2) + y_2 \]
Solving for $x$
\[ [ \ f'(x_1) - f'(x_2) \ ] \ x = y_2 - y_1 + f'(x_1)x_1 - f'(x_2) x_2 \]

\subsection*{parabola}
Suppose the function is $y = f(x) = ax^2$ so $f'(x) = 2ax$.  Performing the substitutions we obtain
\[ 2a(x_1 - x_2) x = a(x_2^2 - x_1^2) + 2ax_1^2 - 2ax_2^2 \]

We can cancel the $a$, divide by $2$ and factor the difference of squares:
\[ (x_1 - x_2) x = \frac{(x_2 - x_1)(x_2 + x_1)}{2} + (x_1 - x_2)(x_1 + x_2) \]
Factor out the common term $(x_2 - x_1)$ (one has a minus sign):
\[ x = - \frac{(x_2 + x_1)}{2} + (x_1 + x_2) \]
\[ = \frac{x_1 + x_2}{2} \]
A remarkably simple answer!

\subsection*{square root}
The function is $y = \sqrt{x}$ so $f'(x) = 1/2\sqrt{x}$.  We have
\[ [ \ \frac{1}{2 \sqrt{x_1}} - \frac{1}{2 \sqrt{x_2}} \ ] \ x = \sqrt{x_2} - \sqrt{x_1} + \frac{x_1}{2 \sqrt{x_1}} - \frac{x_2}{2 \sqrt{x_2}} \]
Multiply by $2$ and simplify the last two terms
\[ [ \ \frac{\sqrt{x_2} - \sqrt{x_1}}{\sqrt{x_1}\sqrt{x_2}} \ ] \ x =  2 \sqrt{x_2} - 2 \sqrt{x_1} + \sqrt{x_1} - \sqrt{x_2}  \]
\[ [ \ \frac{\sqrt{x_2} - \sqrt{x_1}}{\sqrt{x_1x_2}} \ ] \ x =  \sqrt{x_2} - \sqrt{x_1}  \]
\[ x = \sqrt{x_1 x_2} \]

The first one was the arithmetic mean, this is the geometric mean!

Restating the general result
\[ [ \ f'(x_1) - f'(x_2) \ ] \ x = y_2 - y_1 + f'(x_1)x_1 - f'(x_2) x_2 \]

\subsection*{inverse}
The function is $y = f(x) = 1/x$ so $f'(x) = -1/x^2$.  We have
\[ [ \ \frac{1}{x_2^2} - \frac{1}{x_1^2} \ ] \ x = \frac{1}{x_2} - \frac{1}{x_1} + \frac{1}{x_2} - \frac{1}{x_1} \]
\[ (\frac{1}{x_2} - \frac{1}{x_1})(\frac{1}{x_2} + \frac{1}{x_1}) \ x = 2 (\frac{1}{x_2} - \frac{1}{x_1})  \]
\[ (\frac{1}{x_2} + \frac{1}{x_1}) \ x = 2 \]
\[ \frac{1}{x} = \frac{1}{2} \cdot (\frac{1}{x_1} + \frac{1}{x_2}) \]


\end{document}