\documentclass[11pt, oneside]{article} 
\usepackage{geometry}
\geometry{letterpaper} 
\usepackage{graphicx}
	
\usepackage{amssymb}
\usepackage{amsmath}
\usepackage{parskip}
\usepackage{color}
\usepackage{hyperref}

\graphicspath{{/Users/telliott/Dropbox/Github-Math/figures/}}

\title{Polynomial factoring}
\date{}

\begin{document}
\maketitle
\Large

%[my-super-duper-separator]

Algebra and even pre-algebra textbooks spend an inordinate amount of time on factoring higher order polynomials.  One motivation might be to prepare for the integration technique called partial fractions in calculus, but this is pre-algebra and that's a long way away.

A much better reason is to drive home the point that all polynomials have a number of roots equal to the degree.  For example, any cubic equation has 3 roots.  By the roots of a polynomial, we mean the values of $x$ that make the expression equal to zero.  

We can also define roots graphically, they are the $x$-values at the points where the graph of the polynomial crosses the $x$-axis, i.e. $y = 0$.  We're plotting the graph of
\[ y = x^3 + 4x^2 - x - 8 \]
\begin{center} \includegraphics [scale=0.5] {cubic12.png} \end{center}
We can see from the graph that there are clearly $3$ such places.

For any cubic, there is either one real root or three.  It is easy to see graphically why there must be at least one.  As $|x|$ gets large, the cubic term $x^3$ dominates, but the sign of the result is the same as the sign of $x$.  So ultimately, the graph must cross the $x$-axis at least once.

Since there is (at least) one real root, any cubic has the form $(x - a)$ times a quadratic, where $a$ is a real number.  

We can modify any equation so it has only one real root.  Just slide the graph up, in this case by adding $10$ to the equation.  Then the part which reverses directions lies entirely above the $x$-axis.  
\[ y = x^3 + 4x^2 - x + 2  \]
\begin{center} \includegraphics [scale=0.5] {cubic13.png} \end{center}
We can also see this by factoring out $(x - a)$.  The resulting quadratic ($x^2$), like any quadratic, may or may not have real roots.  Thus, two of the roots of a cubic may not be real either.

Having three roots means that any cubic can also be written as
\[ y = (x - a)(x - z)(x - \bar{z})  \]
where, as we said, $a$ is real, and $z$ and $\bar{z}$ may be both complex or both real.  (If the coefficient of $x^3$ is not $1$, then there will be an additional, constant factor).

Any complex roots will be complex conjugates.  This will give something like the following for the quadratic part:
\[ z = 1 + 2i , \ \ \ \ \ \bar{z} = 1 - 2i\]
\[ (x - z)(x - \bar{z}) = (x - 1 + 2i)(x -1 - 2i) \]
\[ = (x - 1)^2 - (2i)^2 \]
\[ = x^2 - 2x + 3 \]
If we compare 
\[ (x-z)(x-\bar{z}) = x^2 - (z + \bar{z})x + z \cdot \bar{z} \]
with $x^2 + bx + c$, it's important that both $z + \bar{z}$ and $z \cdot \bar{z}$ are real.

Of course, there is no guarantee that $a$ is an integer, which makes the focus on cubics with an integer root even more of a puzzle.

\subsection*{problem}
\begin{center} \includegraphics [scale=0.4] {cubic14.png} \end{center}

Here is the problem I saw in an algebra textbook that got me started on this write-up. 
\[ y = 4x^3 + 10x^2 - 6x - 20 \]

We are assured that $x = -2$ is one root of this equation, and that seems reasonable.  In other words $(x + 2)$ should be a factor, since if $x = -2$ the result is zero.

If you make the graph big enough, we clearly see that one root is $\approx -1.8$ and the other \emph{might} be exactly $-2$.  Either take it on faith or plug in to check:  $4(-8) + 10(4) - 6(-2) - 20 = 0$.

To divide the expression above by $(x + 2)$, we do long division:
\begin{verbatim}
        4x^2
       ___________________
x + 2 | 4x^3 + 10x^2 - 6x - 20
        4x^3 +  8x^2
        ------------
                2x^2 - 6x
\end{verbatim}

Here is the rest
\begin{verbatim}
        4x^2 +  2x   - 10
       ___________________
x + 2 | 4x^3 + 10x^2 - 6x - 20
        4x^3 +  8x^2
        ------------
                2x^2 - 6x
                2x^2 + 4x
                ---------
                     - 10x - 20
                     - 10x - 20
                     __________
                              0
\end{verbatim}

We conclude that, since the remainder is zero, $x + 2$ does evenly divide the original expression and
\[ (x + 2)(4x^2 + 2x - 10) = 4x^3 + 10x^2 - 6x + 20 \]
which means that $x = -2$ is a root.

In the text, there is no attempt to factor the quadratic.  The quadratic formula gives
\[ x = \frac{-2 \pm \sqrt{4 + 160}}{8} = \frac{-1 \pm \sqrt{41}}{4}  \]
which is clearly not an integer since $\sqrt{41} = 6.403..$.

Next, a method is introduced which is called synthetic division.  The variable $x$ and its powers are suppressed and we are given:
\begin{center} \includegraphics [scale=0.5] {synth_div.png} \end{center}
You can figure out where this came from, I'm sure.  Sign switching is employed to change subtraction into addition.

But this is just bonkers.  The only real justification for trying to divide the original cubic by $x + 2$ is that (i) the graphing software suggests it may work, and plugging in proves it, (ii) $2$ is an integer, and (iii) it reinforces the concept that equations have roots.

I object because synthetic division is more abstract than long division, in the nature of a cute trick, and it's a special technique applicable only for integer values when the coefficient of $x^3$ is $1$.

Real problems do not often have integer roots!  Many real world problems have some roots that are not even real.  Much better to understand how non-real roots can nevertheless give real values for $f(x)$.

\end{document}