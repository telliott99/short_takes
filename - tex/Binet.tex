\documentclass[11pt, oneside]{article} 
\usepackage{geometry}
\geometry{letterpaper} 
\usepackage{graphicx}
	
\usepackage{amssymb}
\usepackage{amsmath}
\usepackage{parskip}
\usepackage{color}
\usepackage{hyperref}

\graphicspath{{/Users/telliott/Github/figures/}}
% \begin{center} \includegraphics [scale=0.4] {gauss3.png} \end{center}

\title{Phi and Binet's formula}
\date{}

\begin{document}
\maketitle
\Large

%[my-super-duper-separator]

The classic derivation of $\phi$ (often written as the capital letter $\Phi$) is to start with a square inside a rectangle.  Drawing the fourth side of the square also forms a second rectangle, one whose long side is the same length as the square.  
\begin{center} \includegraphics [scale=0.8] {phi.png} \end{center}

We can re-size the entire drawing so that the short side of the smaller rectangle has unit length.  The ratio of long to short side for that rectangle is $\phi$.  We want to find the value(s) of $\phi$ such that the ratio of side lengths is the same for both rectangles.
\[ \frac{\phi}{1} = \frac{\phi + 1}{\phi} \]
\[ \phi^2 = 1 + \phi \]
\[ \phi^2 - \phi - 1 = 0 \]
This quadratic has two solutions.  The vertex of the graph is at $-b/2a = 1/2$.  We just look up the quadratic formula and fill in:
\[ \phi, \psi = \frac{1}{2}  \pm \ \frac{\sqrt{5} }{2} \]

We take $\phi$ to be the positive branch and $\psi$ to be the negative one, and note that since $\psi < 0$ it doesn't satisfy the physical constraint that lengths are always positive, so their ratio is also positive.

For future use, we also compute
\[ \phi - \psi = \sqrt{5} \]

We can also compute the powers of $\phi$:
\[ \phi^2 = 1 + \phi \]
\[ \phi^3 = \phi + \phi^2 = 1 + 2 \phi \]
\[ \phi^4 = \phi + 2 \phi^2 = 2 + 3 \phi \]
\[ \phi^5 = 2 \phi + 3 \phi^2 = 3 + 5 \phi \]
\[ \phi^6 = 3 \phi + 5 \phi^2 = 5 + 8 \phi \]

The powers of $\psi$ are exactly the same, since $\psi^2 = 1 + \psi$.

It looks like we're getting the Fibonacci numbers.  That sequence is defined by the recurrence:
\[ F_{n-2} + F_{n-1} = F_n \]
starting with 
\[ F_1 = 1, \ \ \ \ \ F_2 = 1 \]
so
\[ F_3 = 2, \ \ \ \ \ F_4 = 3, \ \ \ \ \ F_5 = 5, \ \ \ \ \ F_6 = 8 \]
and so on.

Since $\psi$ is a solution of the original quadratic equation, it generates exactly the same sequence.  So then, for example
\[ \phi^5 - \psi^5 = 3 + 5 \phi - (3 + 5 \psi) \]
\[ = 5 (\phi - \psi) \]
\[ = 5 \sqrt{5} \]

If you do the same thing with 
\[ \phi^4 - \psi^4 = 2 + 3 \phi - (2 + 3 \psi) \]
\[ = 3 \sqrt{5} \]

So it seems that
\[ \phi^n - \psi^n = \sqrt{5} \cdot F_n \]
\[ F_n = \frac{1}{\sqrt{5}} \cdot (\phi^n - \psi^n) \]

This is called Binet's formula.  

We can prove the formula by induction.  First, the  base case.  As we said $\phi - \psi = \sqrt{5}$ so $F_1 = 1$, which is certainly correct.  

Since we need both $F_{n-2}$ and $F_{n-1}$ to compute $F_n$ using the recurrence, we should do the next one also as part of the base case.  Namely

\[ \frac{1}{\sqrt{5}} \cdot (\phi^2 - \psi^2) \]
\[ = \frac{1}{\sqrt{5}} \cdot \ [ \ 1 + \phi - (1 + \psi) \] 
\[ =  \frac{1}{\sqrt{5}} \cdot (\phi - \psi) \]
\[ = 1 \]

So then we assume that the following two equations are correct
\[ F_{n-2} = \frac{1}{\sqrt{5}} \cdot (\phi^{n-2} - \psi^{n-2}) \]
\[ F_{n-1} = \frac{1}{\sqrt{5}} \cdot (\phi^{n-1} - \psi^{n-1}) \]
and now, compute $F_n$:

\[ F_{n-2} + F_{n-1} = \frac{1}{\sqrt{5}} \cdot (\phi^{n-2} + \phi^{n-1} -  \psi^{n-2} - \psi^{n-1}  ) \]

This will match the formula, provided that
\[ \phi^{n-2} + \phi^{n-1} = \phi^n \]
Factoring, we obtain
\[ \phi^n = \phi^2(\phi^{n-2}) \]
\[ = (1 + \phi)(\phi^{n-2}) \]
\[ = \phi^{n-2} + \phi^{n-1} \]

And the same applies to $\psi$.

This completes the proof.  $\square$

\subsection*{cancellation}
Our formula is
\[ F_n = \frac{1}{\sqrt{5}} \cdot (\phi^n - \psi^n) \]

It is strange to see that factor of $\sqrt{5}$, since it is irrational, and the formula is supposed to produce integers.  If you just see the formula and haven't been through the derivation it seems weird.

There must be some kind of cancellation.  We look at the powers of $\phi$ and $\psi$ again (say, $n = 5$).

\[ \phi^5 = 3 + 5 \phi \]
\[ \psi^5 = 3 + 5 \psi \]

So the difference is then
\[ \phi^5 - \psi^5 = 5(\phi - \psi) = 5 \ \sqrt{5} \]
This is true for every power of $\phi^n - \psi^n$.

\subsection*{generating functions}

I came across a couple of posts talking about the generating function for the Fibonacci sequence (references at the end).

A generating function is a function like this:
\[ F(x) = \sum_{n=0}^{\infty} f(n) \ x^n \]

It is an infinite series whose cofactors are the terms of the sequence:  $f(0), f(1) \dots$.  We do not yet know what form $f$ will take, so we don't know what $F(x)$ really looks like either, yet.

Generating functions are one of those places where math gets too complicated for me (in terms of justifying results), but let's just accept the definition and see where things go.

For the Fibonacci sequence, we say that $f(0) = 0$ (we start counting from 1).   And we want the values of $f(1), f(2), f(3)$ to be equal to $1,1,2$ and so on.

Then we do the following trick.
\[ F(x) =  \sum_{n=0}^{\infty} f(n) \ x^n \]
\[  = f(0) + \sum_{n=1}^{\infty} f(n) \ x^n \]
As we said, $f(0)$ is just zero so
\[ F(x) = \sum_{n=1}^{\infty} f(n) \ x^n \]

And, since $f(1) = 1$, the first term is just $x$ and
\[ F(x) = x + \sum_{n=2}^{\infty} f(n) \ x^n \]

Now
\[ f(n) = f(n-2) + f(n-1) \]
 (this is our recurrence relationship), so we can substitute
\[ F(x) = x + \sum_{n=2}^{\infty} \ [ \ f(n-1) +  f(n-2) \ ] \   x^n \]
and break it up into two sums
\[ = x +  \sum_{n=2}^{\infty} f(n-1) \ x^n +  \sum_{n=2}^{\infty} f(n-2) \ x^n \]

Focusing on the first term on the right-hand side we factor out one $x$
\[ \sum_{n=2}^{\infty} f(n-1) \ x^n = x \ \sum_{n=2}^{\infty} f(n-1) \ x^{n-1}\]
Substitute $m = n-1$:
\[ =  x \sum_{m=1}^{\infty} f(m) \ x^{m} = x F(x) \]

(Notice the changed lower bound of the summation).  But $m$ is just a dummy variable, so as a shortcut you can think of substituting $n$ for $n-1$ above.

For the last term on the right we have
\[  \sum_{n=2}^{\infty} F_{n-2} \ x^n = x^2  \sum_{n=2}^{\infty} F_{n-2} \ x^{n-2}\]
Substitute $n = n - 2$:
\[ =  x^2 \sum_{n=0}^{\infty} F_n \ x^n = x^2 F(x) \]

which gives, finally
\[ F(x) = x + x F(x) + x^2 F(x) \]
solve for $F(x)$:
\[ (1 - x - x^2) \ F(x) = x \]
\[ F(x) = \frac{x}{1 - x  - x^2} \]

This completes the first part.

\subsection*{quadratic}

Now, we need to solve (i.e. factor)
\[ -x^2 - x + 1 \]
This is (almost) the same as
\[ \phi^2 - \phi - 1 = 0 \]
Both equations have the same discriminant ($\sqrt{5}$) and also the entire numerator is the same.  But there is a minus sign in the solutions to the new one because the cofactor of $x^2$ is negative.

The quadratic equation gives the values of $x$ which make this equation equal to zero.
\[ x = \frac{1 \pm \ \sqrt{5}}{-2} \]
\[ = - ( \frac{1}{2} \pm \ \frac{\sqrt{5}}{2} ) \]
\[ = - \frac{1}{2} \pm \ \frac{\sqrt{5}}{2}  \]
Now 
\[ - \frac{1}{2} - \ \frac{\sqrt{5}}{2} = - \phi \]
\[ - \frac{1}{2} + \ \frac{\sqrt{5}}{2} = - \psi \]

That is, $x = - \phi, x = - \psi$ are solutions. 

We re-write the quadratic as
\[ (x + \phi)(x + \psi) \]
and so
\[ F(x) = \frac{x}{(x + \phi)(x + \psi)} \]

Note:  it is equally true that $-\phi$ and $-\psi$ are solutions for
\[ (-1)(x + \phi)(x + \psi)  = 0 \]

The second reference below adopts the latter as the correct factorization, without comment.  

We adopt their solution and explain why, at the end.

\subsection*{partial fractions}

We can make further progress more by the method of partial fractions.  The product in the denominator is just what we would have if we were trying to add two fractions:
\[ \frac{A}{x + \phi} + \frac{B}{x + \psi} = - \frac{x}{(x + \phi)(x +\psi)} \]
where the terms $A$ and $B$ yet to be determined.

Multiply both sides by $(x + \phi)(x + \psi)$ to obtain
\[ A(x + \psi) + B(x + \phi) = - x \]

Now, use the solutions we just found.  If $x = - \phi$ then
\[ A(-\phi + \psi)  = \phi \]
\[ - A(\phi - \psi)  = \phi  \]
Recall that $\phi - \psi = \sqrt{5}$ so
\[ A = - \frac{1}{\sqrt{5}} \ \phi \]

similarly, if $x = - \psi$ then
\[ \psi = B (- \psi + \phi)  \]
\[ = \sqrt{5} B \]
\[ B = \frac{1}{\sqrt{5}} \ \psi \]

We have then
\[ F(x) = \frac{1}{\sqrt{5}} \ ( - \frac{\phi}{x + \phi} + \frac{\psi}{x + \psi}  )\]

One of the properties of $\phi$ and $\psi$ is that
\[ \phi \cdot \psi = (\frac{1}{2} + \frac{\sqrt{5}}{2}) \cdot  (\frac{1}{2} - \frac{\sqrt{5}}{2}) = - 1 \]

Multiply top and bottom on the left by $\psi$ and on the right by $\phi$:
\[ F(x) = \frac{1}{\sqrt{5}} \ ( - \frac{\phi}{x + \phi} + \frac{\psi}{x + \psi}  )\]
\[ = \frac{1}{\sqrt{5}} \ ( - \frac{-1}{\psi x - 1} + \frac{-1}{\phi x - 1}  )\]
\[  = \frac{1}{\sqrt{5}} \ ( - \frac{1}{1 - \psi x} + \frac{1}{1 - \phi x}  )\]
Switch the order of terms
\[ F(x)  = \frac{1}{\sqrt{5}} \ (\frac{1}{1 - \phi x}  - \frac{1}{1 - \psi x} )\]

And then, finally, recall that 
\[ \frac{1}{1 - r} = 1 + r + r^2 + \dots \]
for $-1 < r < 1$.

So our equation for $F(x)$ is \emph{really} the sum of two power series.
\[ F(x) = \frac{1}{\sqrt{5}} \ [ \  ( 1 + \phi x + \phi^2x^2 + \dots)  - (1 + \psi x + \psi^2 x^2 + \dots)  )\]
\[ = \frac{1}{\sqrt{5}} \sum_{n=0}^{\infty} \ (\phi^n - \psi^n) x^n \ ] \]

We are to obtain the Fibonacci sequence from the cofactors of $x$.  As you can see, the nth Fibonacci number will be
\[ \frac{1}{\sqrt{5}} \ (\phi^n - \psi^n)  \]
This is Binet's formula, again.

I think guessing the pattern and the proof by induction was a lot simpler.

\subsection*{discussion}

When I first worked this problem, I obtained $\psi^n - \phi^n$ in parentheses, above.  I looked for an error causing the change in sign without success.  The references I looked at include:

\url{https://towardsdatascience.com/why-is-the-closed-form-of-the-fibonacci-sequence-not-used-in-competitive-programming-674b805da341}

\url{https://austinrochford.com/posts/2013-11-01-generating-functions-and-fibonacci-numbers.html}

The second reference gives the factorization of $-x^2 - x + 1$ as $-(x + \phi)(x + \psi)$, without comment.  If we could make sense of that choice, it would solve the sign problem.

Naturally, \emph{any} quadratic could have additional constant factors.  So
\[ (x + \phi)(x + \psi) = 0 \]
and
\[ (-1)(x + \phi)(x + \psi) = 0 \]
have exactly the same solutions.

Choosing the $+$ branch of the factorization leads to solutions that are "correct" but give negative integers.  By definition, the Fibonacci sequence is positive.  This is what justifies the choice of the $(-1)$ times the factorization given by the quadratic formula.

\end{document}