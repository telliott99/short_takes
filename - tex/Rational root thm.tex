\documentclass[11pt, oneside]{article} 
\usepackage{geometry}
\geometry{letterpaper} 
\usepackage{graphicx}
	
\usepackage{amssymb}
\usepackage{amsmath}
\usepackage{parskip}
\usepackage{color}
\usepackage{hyperref}

\graphicspath{{/Users/telliott/Github/figures/}}
% \begin{center} \includegraphics [scale=0.4] {gauss3.png} \end{center}

\title{Rational root theorem}
\date{}

\begin{document}
\maketitle
\Large

%[my-super-duper-separator]

Consider this polynomial with integer coefficients:
\[ a_0 + a_1 x + \dots + a_n x^n = 0 \]

The theorem says that \emph{if} there is a rational root $p/q$ of this equation (a solution), then it must be true that $p|a_0$ ($p$ is a divisor of $a_0$) and $q|a_n$.  

As usual, we require that $p/q$ be in "lowest" terms, so that there is no remaining common factor.  $p$ and $q$ are coprime.

\emph{Proof}.

The proposed solution can be written as
\[ a_0 + a_1 (p/q) + \dots + a_n (p/q)^n = 0 \]

Multiply by $q^n$
\[ a_0 q^n + a_1 p q^{n-1} + \dots + a_n p^n = 0 \]

Now simply isolate the first term
\[ a_1 p q^{n-1}  + \dots + a_n p^n = - a_0 q^n \]

Factor out $p$ on the left-hand side:
\[ p(a_1 q^{n-1} + \dots + a_n p^{n-1}) = - a_0 q^n \]

Since $p$ divides the product on the left-hand side, it must divide either $a_0$ or $q$.  But $p$ is co-prime to $q$.  Therefore $p|a_0$.

An analogous argument shows that $q|a_n$.

\subsection*{caution}
I have written the polynomial starting with $a_0$ so that the index will increase from left-to-right.

However, we usually write simple polynomials with the higher powers first:
\[ a_n x^n + \dots + a_1 x + a_0 = 0 \]

Our result says that the numerator of any rational solution must evenly divide $a_0$, the constant at the end, and further that the denominator of any rational solution must evenly divide the cofactor of the highest power of $x$, i.e. $a_n$

\subsection*{applications}

Consider
\[ x^2 - 2 = 0 \]

The solution is $\sqrt{2}$.  

For a rational solution $p/q$, it must be that $p$ divides $-2$ and $q$ divides $1$.   This means that $p \in \{ 1,-1,2,-2 \}$ and $q = 1$.

None of the four forms $p/q$ solves the equation.  Thus, there is no solution in the rational numbers.

The condition $q= 1$ means that, \emph{if} there is a solution, it must be an integer.  And this is true for any square root.  Therefore, the only rational square roots are those of the perfect squares.

Consider the cube root:
\[ x^3 - 2 = 0 \]

The same analysis holds.  None of the possible solutions is an actual one.  The cube root of $2$ is also irrational.

Now consider
\[ 7x^3 - 3x^2 + 14x - 6 = 0 \]

We have that $q$ is either $1$ or $7$ and $p$ is either $1,2$ or $3$  or one of their negatives, so the possibilities are

\[ 1/1, 2/1, 3/1, 6/1, 1/7, 2/7, 3/7, 6/7 \]
together with their negatives.  

By trial and error, we discover that $x = 3/7$ is a solution since
\[ 7 (3/7)^3 - 3(3/7)^2 + 14(3/7) - 6 = 0 \]
\[ \frac{27}{49} - \frac{27}{49} + \frac{42}{7} - 6 = 0 \ \ \ \checkmark \]

Which means that $7x - 3$ must factor out of the equation, so that $x = 3/7$ gives zero for that term.  

By "long division" we factor to obtain:

\[ = (7x - 3)(x^2 + 2) \]

and recognize that the term $x^2 + 2$ does not have any rational roots (it cannot be equal to zero for any value of $x$).

\url{https://www.quora.com/Whats-your-favorite-mathematical-theorem}

\url{https://en.wikipedia.org/wiki/Rational_root_theorem}

\end{document}