\documentclass[11pt, oneside]{article} 
\usepackage{geometry}
\geometry{letterpaper} 
\usepackage{graphicx}
	
\usepackage{amssymb}
\usepackage{amsmath}
\usepackage{parskip}
\usepackage{color}
\usepackage{hyperref}

\graphicspath{{/Users/telliott/Dropbox/Github-Math/figures/}}
% \begin{center} \includegraphics [scale=0.4] {gauss3.png} \end{center}

\title{Phi, Fibonacci and the Pentagon}
\date{}

\begin{document}
\maketitle
\Large

It is easy to show that the chords of the pentagon form a rhombus, a parallelogram with all sides equal.
\begin{center} \includegraphics [scale=0.3] {pent5.png} \end{center}

We start with an isosceles triangle showing opposing angles at the base are equal, then counting up the angles in two types of triangles shows the central angle at each vertex is equal to the two flanking ones.  The magenta dotted angles are $1/5$ of a right angle.  Finally, this gives the equal angles which lead to parallel lines.

As a result, we see that there are two classes of similar triangles.  One is tall and skinny and has a base angle of $72^{\circ}$.  The other is short and squat and has a base angle of $36^{\circ}$.

Hence the smaller red triangle is similar to the light red one.  If we label the sides of the larger as $x$ and scale the small triangle to have equal sides of length $1$ then by similar triangles we have
\[ \frac{1}{x} = \frac{x}{1 + x} \]
\[ x^2 = x + 1 \]

You should recognize $\phi$.  

A similar thing can be done with the tall skinny triangles.  If the side of the pentagon is re-scaled to have length $1$, which is also the base of these triangles, then their equal sides are of length $\phi$.

We can use this to draw a right triangle with angle $18^{\circ}$, side opposite of length $1/2$ and hypotenuse of length $\phi$.  Hence the sine of $18^{\circ}$ turns out to be equal to $1/2$ divided by $\phi$, or $1/2 \phi$.

Recall that $1/\phi = \phi - 1$, so the sine of $18^{\circ}$ is also $(\phi - 1)/2$.

We will also need $(\phi - 1)^2 = 2 - \phi$.  

The cosine of $2A$ is easy if we have the sine of $A$.
\[ \cos 36^{\circ} = 1 - 2 \sin^2 18^{\circ} \]

Since $\sin 18^{\circ} = 1/2 \phi = (\phi - 1)/2$ we have
\[ \cos 36^{\circ} = 1 - 2 (\frac{ \phi - 1}{2})^2 \]
\[ = 1 - \frac{1}{2} (2 - \phi) = \frac{\phi}{2}  \]
which can be confirmed with a calculator.

\begin{center} \includegraphics [scale=0.6] {sin18.png} \end{center}

But the situation we're most interested in is the relationship between the radius of the circle that contains the vertices of the pentagon, and the side length.  Suppose the side length is $1$.

Since half of the central angle is $\cos 36^{\circ}$ we have that the radius $r$ is the hypotenuse of a right triangle with that angle.

What we know is the side opposite, which is $1/2$.  We have that
\[ \sin 36^{\circ} = \frac{1/2}{r} \]

Here's one way:
\[ \sin^2 36^{\circ} = 1 - (\frac{\phi}{2} )^2 \]
\[ = 1 - \frac{1}{4} (\phi + 1) \]
\[ = \frac{3 - \phi}{4} \]

so
\[ \sin 36^{\circ} = \frac{\sqrt{3 - \phi}}{2}  \]

$r = 1/2$ divided by $\sin 36^{\circ}$, so we can cancel the $2$ and use the inverse:
\[ r = \frac{1}{\sqrt{3 - \phi}} \]

For a pentagon with side length $1$, the radius of the circumscribing circle is as given.

The apothem is
\[ \sqrt{\frac{1}{3 - \phi} - \frac{1}{4}} = \sqrt{\frac{4 - (3 - \phi)}{4(3 - \phi)}}  \]
\[ = \frac{\phi}{2} \frac{1}{\sqrt{3 - \phi}} \]

If the radius is scaled to be length $1$, then the apothem is just $\phi/2$ or the cosine of $36^{\circ}$.

\subsection*{irrationality}
Usually, the first irrational number that one encounters in mathematics is $\pi$ and the first one with a proof is $\sqrt{2}$.  However, $\phi$ is also a good candidate for the first such number to be discovered.  

The proof for $\phi$ comes from the fact that drawing the diagonals in the pentagon generates a smaller regular pentagon.  So, obviously, the construction can be continued again and again.  

But this means that the ratio of the diagonal and sides of a pentagon (i.e., $\phi$) cannot be a rational number, since it must have the same value for all regular pentagons, and yet the values the numerator and denominator can be made as small as we like.  

This is quite like a proof for the irrationality of $\sqrt{2}$ that I saw in Apostol:

A more elaborate exposition is:

\url{https://jeremykun.com/2011/08/14/the-square-root-of-2-is-irrational-geometric-proof/}

Theorem:  if there is an isosceles triangle with integer sides, then there is a smaller one with the same property.

\emph{Proof}.

Draw an isosceles triangle with side length $1$, then Pythagoras tells us that the hypotenuse is equal in length to $\sqrt{2}$ (left panel).

Our hypothesis is that this length is a rational number, and its ratio to the side is in "lowest terms".

\begin{center} \includegraphics [scale=0.4] {sqrt2e.png} \end{center}

Mark off the length of the side (length $1$) on the hypotenuse, and erect a perpendicular (middle panel).  Also draw the line segment to the opposite vertex of the original triangle.

The new small triangle that is formed containing the right angle and with side length $x$ in the middle panel is isosceles, because it is a right triangle that contains one of the complementary angles of the original right triangle.

By hypothesis, its side length $x$ is the difference of two rational numbers, so $x$ is a rational number.

Furthermore, the \emph{other} small triangle is also isosceles.  Its base angles, when added to the equal angles of an isosceles triangle, form right angles.  This allows us to mark the side along the base as having length $x$ as well.

Therefore, the hypotenuse of the new, small right triangle is a rational number, since it is equal to $1 - x$.

We are back where we started, with an isosceles right triangle that has all rational sides.  

It is clear that this process can continue forever.  The sides will never be in "lowest terms" because we can always form a new similar but smaller right triangle, which amounts to evenly dividing both the sides and the hypotenuse by a rational number.

$\square$

\subsection*{Aside on $\phi$ and the Fibonacci sequence}

The Fibonacci sequence is defined using a recurrence relation $F_{n+2} = F_{n+1} + F_n$.

Depending on how you define things, you could start with $F_0 = 0$ or ignore $0$ and let the first number be $F_1 = 1$, with $F_2 = 1$.  After that the recurrence kicks in.

The first ten numbers in the sequence are:
\begin{verbatim}
1 1 2 3 5 8 13 21 34 55 ...
\end{verbatim}

(According to wikipedia, Fibonacci himself started with 1 2 3 $\dots$, but for what we're going to do it is useful that the fifth number, $F_5$, is equal to $5$).

Recall that $\phi^2 = 1 + \phi$.  The powers of $\phi$ generate an interesting pattern:
\[ \phi^2 = \phi \cdot \phi = 1 + \phi \]
\[ \phi^3 = \phi \cdot \phi^2 = \phi + \phi^2 = 1 + 2 \phi \]
\[ \phi^4 = \phi \cdot \phi^3 = \phi + 2 \phi^2 = 2 + 3 \phi \]
\[ \phi^5 = \phi \cdot \phi^4 = 2 \phi + 3 \phi^2 = 3 + 5 \phi \]

Both the first term and the cofactors generate the elements of the Fibonacci sequence from the powers of $\phi$.

The reason is that $\phi^n + \phi^{n+1} = \phi^{n+2}$, which is the same as the definition for the Fibonacci numbers.

Going back to the solution to the original quadratic equation that we left behind, take the negative branch of the square root, and let us call that other solution $\psi$.

\[ \psi = \frac{1 - \sqrt{5}}{2} \]

If you look closely, you can easily see that
\[ \psi + \phi = 1 \]

Since $\psi$ is also a solution of the original equation:
\[ \psi^2 = 1 + \psi \]

Furthermore
\[ (\phi + \psi)^2 = \phi^2 + 2 \phi \cdot \psi + \psi^2 \]

Now, the left-hand side is just $1$, since $\phi + \psi = 1$.  Furthermore $\phi^2 = 1 + \phi$ and $\psi^2 = 1 + \psi$ so
\[ 1 = 1 + \phi + 2 \phi \cdot \psi + 1 + \psi \]
\[ 1 = 3 + 2  \phi \cdot \psi  \]
\[ \phi \cdot \psi = -1 \]

$\psi$ is the negative inverse of $\phi$.  

$\psi$ comes in handy in the following derivation.  Since $\psi$ solves our original equation, the powers of $\psi$ are just like the powers of $\phi$:
\[ \psi^2 = 1 + \psi \]
\[ \psi^3 = 1 + 2 \psi \]
\[ \psi^4 = 2 + 3 \psi \]
\[ \psi^5 = 3 + 5 \psi \]

So
\[ \phi^5 - \psi^5 = 5(\phi - \psi) \]
\[ \frac{\phi^5 - \psi^5}{\phi - \psi} = 5 \]

$5$ is the fifth Fibonacci number.  If $F_n$ is the nth Fibonacci number
\[ \frac{\phi^n - \psi^n}{\phi - \psi} = F_n \]

This is called Binet's formula.  If you work out the denominator you find that it is 
\[ \phi - \psi =  (1 + \sqrt{5})/2 - (1 - \sqrt{5})/2 = \sqrt{5} \]

The general equation is
\[ F_n = \frac{1}{\sqrt{5}} \cdot  (\phi^n - \psi^n) \]

This formula is quite surprising, because the Fibonacci numbers $F_n$ on the left-hand side are \emph{integers}, and yet the first factor on the right-hand side is the inverse of a square root, which is definitely not an integer or even a rational number.  

But it turns out that the differences $\phi^n - \psi^n$ contain only odd powers of $\sqrt{5}$.  So after multiplying by $1/\sqrt{5}$, we end up only with even powers, which are whole numbers.  

In fact, there is a connection between the Fibonacci sequence and Pascal's triangle.  
\begin{center} \includegraphics [scale=0.4] {fib_triangle.png} \end{center}

Binet's formula says the nth Fibonacci number is 
\[ (\frac{1}{2})^n \cdot \frac{1}{\sqrt{5}} \cdot \ [ \ (1 + \sqrt{5})^n - (1 - \sqrt{5})^n \ ] \]

If you do the two binomial expansions, they are the same except that each term of the second one has a factor of $(-1)^n$.  As a result, the odd powers survive, as twice the value.  In the case of $n = 5$ we would have

\[ (1 + \sqrt{5})^5 = 1 + 5 \sqrt{5} + 10 \sqrt{5}^2 + 10 \sqrt{5}^3 + 5 \sqrt{5}^4 + \sqrt{5}^5 \]
\[ (1 - \sqrt{5})^5 = 1 + (-1)^1 5 \sqrt{5} + (-1)^2 10 \sqrt{5}^2 + (-1)^3 10 \sqrt{5}^3 \dots \]
\[ \dots +(-1)^4  5 \sqrt{5}^4 + (-1)^5 \sqrt{5}^5 \]

expanding $(-1)^n$ and subtracting the second line from the first, the odd powers survive and the even powers vanish.
\[ = 10 \sqrt{5} + 20 \sqrt{5}^3 + 2 \sqrt{5}^5 \]

so the whole thing is
\[ = (\frac{1}{2})^5 \cdot \frac{1}{\sqrt{5}} \cdot \ [ \ 10 \sqrt{5} + 20 \sqrt{5}^3 + 2 \sqrt{5}^5 \ ] \]
\[ = (\frac{1}{2})^5 \cdot \ [ \ 10 + 20 \sqrt{5}^2 + 2 \sqrt{5}^4 \ ] \]
\[ = (\frac{1}{2})^5 \cdot \ [ \ 10 + 100 + 50 \ ] \]
\[ = \frac{1}{32} \cdot 160 = 5 \]

The coefficients of the powers of $\sqrt{5}$ are twice the alternate coefficients in the binomial expansion for $(1 + \sqrt{5})^5$:  $5, 10$ and $1$.  

If you work through more examples you'll see there is a cancellation that happens with $1/2^n$ so that this always results in an integer.  But this is probably a good place to stop.

\subsection*{note}

If you're trying to impress your friend, or a student, and reconstruct the above argument from the beginning, beware.  It has happened to me that I switched the roles of $a$ and $b$, or $x$ and $1$.  Then, you'll get a slightly different equation whose solution is the positive inverse of $\phi$.


\end{document}
