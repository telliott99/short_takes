\documentclass[11pt, oneside]{article} 
\usepackage{geometry}
\geometry{letterpaper} 
\usepackage{graphicx}
	
\usepackage{amssymb}
\usepackage{amsmath}
\usepackage{parskip}
\usepackage{color}
\usepackage{hyperref}

\graphicspath{{/Users/telliott/Dropbox/Github-Math/figures/}}
% \begin{center} \includegraphics [scale=0.4] {gauss3.png} \end{center}

\title{Theorems}
\date{}

\begin{document}
\maketitle
\Large

\section*{Mean Value Theorem:  MVT}

A man passes a police car at point $A$ doing 60 mph (the speed limit) and 4 minutes later passes another police car at point $B$, also doing 60 mph, yet the second officer writes him a ticket for speeding, justified by the mean value theorem.  

The reason:  point $A$ and point $B$ are 5 miles apart, hence the average speed over this interval was 75 mph, and \emph{must at least have been equaled at some point}.

\subsection*{theorem}

If $f$ is a "nice" function on the interval $(a,b)$ then there exists at least one point $c$ in that interval where
\[ f'(c) = \frac{f(b) - f(a)}{b-a} \]
\begin{center} \includegraphics [scale=0.4] {mvt.png} \end{center}

\subsection*{being nice}

What does it take to be "nice"?  The function $f$ must be continuous over the closed interval $[a,b]$ and differentiable over the open interval $(a,b)$.

Why continuous?  Suppose $f(b) = f(a)$ so the average is zero, and suppose $f$ starts out increasing, so we we're looking for the maximum $c$ where the slope is zero.  Then when the function approaches the point $c$ which would otherwise be the maximum, simply define $f(x)$ for points near $c$ as something smaller for a while.  That's a legal function which evades the "capture" at the maximum.

Why differentiable?  In a similar way the function might have a sharp point at the maximum, shaped like the inverse of the absolute value function.  Then, at the maximum there will not be a derivative (since the difference quotient has different limits in the two directions), so $f'(c)$ does not exist.

The standard definition is continuity on the closed interval $[a,b]$ and differentiability on the open interval $(a,b)$.  That's because the limit at $a$ or $b$ can exist even if the function is not defined at those points (and is thus not differentiable at those points), but it still must exist at every point close to $a$ and $b$, and so be continuous.

\subsection*{Extreme value theorem}

If a real-valued function $f$ is continuous in the closed and bounded interval $[a,b]$, then $f$ must attain a maximum and a minimum, each at least once.  Consider the maximum.

The theorem says that there exists a number $c$ in the interval such that $f(c) \ge f(x)$ for all $x$ in the interval.

This theorem is both intuitively obvious and deep.  Proving it is part of a course in analysis.  It relies on properties of the real numbers.

\subsection*{Fermat's theorem on extreme values}

Following \emph{Teaching AP Calculus}:

\url{http://wp.me/p2zQso-12S}

If a function is continuous on a closed interval, and has a maximum (or minimum) at a point $c$ in the interval, then the derivative at $x = c$ is either zero or does not exist.  

\emph{Proof}.

We will talk only about the case of a maximum.  The minimum is a symmetric argument.

We consider the difference quotient and its limit (which exists and is the same when approaching from either direction, if and only if the derivative exists):

\[ \lim_{h \rightarrow 0} \ \frac{f(c + h) - f(c)}{h} \]

For a maximum, $f(c + h) \le f(c)$ regardless of whether $h$ is positive or negative.

So, approaching from the left, we have that the numerator is negative or zero, while the denominator is negative.  Therefore the difference quotient is positive or zero.

When approaching from the right, we have that the numerator is still negative or zero, but the denominator is positive.  Therefore the difference quotient is negative or zero.

So one possibility is that the two limits are not equal (because the function does not have a derivative there).

The other possibility is that the two limits are equal and they are both zero.

$\square$

\subsection*{Rolle's theorem}

Rolle's Theorem says that for a "nice" $f$, if $f(a) = f(b)$, then there will exist at least one point $c$ in the interval $(a,b)$ such that $f'(c) = 0$.  So it's like the MVT but for a function where $f(a) = f(b)$

\emph{Proof}.

One possibility is that the function is constant.   The derivative of a constant is zero, so any point in the interval satisfies $f'(c) = 0$.

If the function is not constant, then by the Extreme Value Theorem it must have a maximum or minimum in the interval $(a,b)$.  Fermat's theorem says that the derivative there must be zero.

\subsection*{MVT}

The MVT proof basically turns Rolle's interval so that $f(a) = f(b)$.

We construct a new function utilizing the slope of the line connecting $a$ and $b$.  That slope is
\[ m = \frac{f(b) - f(a)}{b - a} \]

The equation of the line connecting the two endpoints is
\[ m(x-a) + f(a) \]

We subtract the above from $f(x)$ to construct our new function.  This is what "turns" $f(x)$ so the values at the endpoints are equal.  If we write it out in full:
\[ g(x) = f(x) - \ [ \ f(a) + \frac{f(b) - f(a)}{b - a}  \ (x-a) \ ] \]

Notice that at $x = a$ the term with $(x-a)$ is just zero so $g(a) = f(a) - f(a) = 0$.

On the other hand, at $x = b$ we have $g(b) = f(b) - f(a) - f(b) + f(a) = 0$.  Since $g(a) = g(b)$, Rolle's theorem applies.

What is the derivative of $g$?  To make life easier remember that the slope
\[ m = \frac{f(b) - f(a)}{b - a} \]
is \emph{just a number} and so is $f(a)$.  Hence

\[ g(x) = f(x) - \ [ \ f(a) + m(x-a) \ ] \]
\[ g'(x) = f'(x) - m \]
Rolle's theorem says that there is at least one value $x = c$ where this expression is zero, which means at that point:
\[ f'(c) = m = \frac{f(b) - f(a)}{b - a} \]
This completes the proof of the MVT.


\end{document}}