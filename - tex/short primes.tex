\documentclass[11pt, oneside]{article} 
\usepackage{geometry}
\geometry{letterpaper} 
\usepackage{graphicx}
	
\usepackage{amssymb}
\usepackage{amsmath}
\usepackage{parskip}
\usepackage{color}
\usepackage{hyperref}

\graphicspath{{/Users/telliott/Dropbox/Github-Math/figures/}}
% \begin{center} \includegraphics [scale=0.4] {gauss3.png} \end{center}

\title{Primes}
\date{}

\begin{document}
\maketitle
\Large

%[my-super-duper-separator]

Let's step through the "whole" numbers, what math peeps call the integers, from $2$ to $9$, one at a time. 

$4$ is the first one to have any factor other than itself (and $1$), since $2 \cdot 2 = 4$.  The next one is $6$, since $2 \cdot 3 = 6$.  And then $8$, since $2 \cdot 2 \cdot 2 = 8$.  And finally $9$, since $3 \cdot 3 = 9$.

So the first thing is, we can separate the integers into those numbers that have factors (other than themselves and $1$), and those that don't.  The latter are called primes.  We have the first four primes

\begin{verbatim} 
2 3 5 7 
\end{verbatim}

The other kind, that do have factors, are called composite

\begin{verbatim}
4 6 8 9
\end{verbatim}

There's a method for finding primes, it's called the Sieve of Eratosthenes.  Write out the first $25$ integers:

\begin{verbatim}
  1  2  3  4  5  6  7  8  9 10
 11 12 13 14 15 16 17 18 19 20
 21 22 23 24 25
 \end{verbatim}

$1$ is special, so we don't think about it for now

\begin{verbatim}
  .  2  3  4  5  6  7  8  9 10
 11 12 13 14 15 16 17 18 19 20
 21 22 23 24 25
 \end{verbatim}

Now, strike out every number that is a multiple of $2$, which means every even number after $2$.  (Anything that ends in $2,4,6,8,0$).

\begin{verbatim}
  .  2  3  .  5  .  7  .  9 ..
 11 .. 13 .. 15 .. 17 .. 19 ..
 21 .. 23 .. 25
 \end{verbatim}

Next, strike out every number that is a multiple of $3$.  Notice that the first number to get clobbered is $3 \cdot 3 = 9$:

\begin{verbatim}
  .  2  3  .  5  .  7  .  . ..
 11 .. 13 .. .. .. 17 .. 19 ..
 .. .. 23 .. 25
 \end{verbatim}
 
 What remains (except for the last one) are the primes.  The prime numbers smaller than $25$ are
 
 \begin{verbatim}
 2 3 5 7 11 13 17 19 23
 \end{verbatim}

If we were to do another round, the next number to be xx'd out would be $5 \cdot 5 = 25$
 
A couple of points:

$\circ$ \ \ to find primes $< N$, we check candidates up to $\sqrt{N}$

$\circ$ \ \ this makes it efficient.  We checked only $2$ and $3$ and obtained 9 different primes.

\subsection*{round of 5}

If we run the same algorithm up to $49$ we get

\begin{verbatim}
  .  2  3  .  5  .  7  .  9 ..
 11 .. 13 .. .. .. 17 .. 19 ..
 21 .. 23 .. .. .. .. .. 29 ..
 31 .. .. .. .. .. 37 .. .. ..
 41 .. 43 .. .. .. 47 .. 49
 \end{verbatim}
 
 Now, $49$ is not prime, it's still there because we haven't checked $7 \cdot 7 = 49$ yet.
 
 We only got $5$ new primes.  The density of primes is smaller going to larger numbers.
 
 Once again, the first number knocked out by checking multiples of $n$ is $n^2$.  What that means is that, if we are factoring numbers up to say, $120$, we only need to check $2,3,5,7$ as possibles.  That makes life a lot easier.  Similarly up to $17^2 = 289$, we need only add $11$ and $13$.
 
All primes going forward end in $1,3,7,9$.
 
 One more thing, every number that is not prime has a \emph{unique} prime factorization.  Let's look at all the non-primes from our sieve
 
 \begin{verbatim}
  4 = 2.2
  6 = 2.3
  8 = 2.2.2
  9 = 3.3
 10 = 2.5
 12 = 2.2.3
 14 = 2.7
 15 = 3.5
 16 = 2.2.2.2
 18 = 2.3.3
 20 = 2.2.5
 21 = 3.7
 22 = 2.11
 24 = 2.2.2.3
 25 = 5.5
 26 = 2.13
 27 = 3.3.3
 28 = 2.2.7
 30 = 2.3.5
 32 = 2.2.2.2.2
 33 = 3.11
 34 = 2.17
 35 = 5.7
 36 = 2.2.3.3
 38 = 2.19
 39 = 3.13
 40 = 2.2.2.2.5
 42 = 2.3.7
 44 = 2.2.11
 45 = 3.3.5
 46 = 2.23
 48 = 2.2.2.2.3.
 49 = 7.7
 \end{verbatim}
 
 It seems to be quite amazing how the gears mesh together so that the patterns from $2 \cdot 13$, $2 \cdot 19$ , $3 \cdot 13$, etc. don't clash with the ones from smaller numbers.
 


\end{document}
