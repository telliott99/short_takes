\documentclass[11pt, oneside]{article} 
\usepackage{geometry}
\geometry{letterpaper} 
\usepackage{graphicx}
	
\usepackage{amssymb}
\usepackage{amsmath}
\usepackage{parskip}
\usepackage{color}
\usepackage{hyperref}

\graphicspath{{/Users/telliott/Github-Math/figures/}}
% \begin{center} \includegraphics [scale=0.4] {gauss3.png} \end{center}

\title{Exponentials}
\date{}

\begin{document}
\maketitle
\Large

%[my-super-duper-separator]

We start to explore functions more systematically.  You already know powers like $x^2$ and $x^3$ and in general, we can write
\[ y = ax^n \]
where $a$ is some constant, $x$ is the independent variable, and $n$ is a positive integer.  So if $a = 2$ and $n = 3$ 
\[ y = 2x^3 \]
is a \emph{cubic} function because of that power of $3$.  Various powers of $x$ can be combined into what are called \emph{polynomials} like
\[ y = ax^2 + bx + c \]

The next major class of functions are the \emph{exponential} functions like $y = b^x$, where $b$ is some constant called the \emph{base} and $x$ is again, the independent variable.  A simple example is 
\[ y = 2^x \]
We can get some idea about this function by trying different values of $x$.
\begin{center} \includegraphics [scale=0.4] {exp1.png} \end{center}

We sketch the graph by filling in between the points $(x,2^x)$ obtained for integer values of $x$.  It's important to remember that $x$ doesn't have to be an integer.  For example $x = 1/2$ is perfectly legal, so then $y = 2^{1/2}$.

What does it mean to say that $y = 2^{1/2}$?  The meaning becomes clear when we square both sides.
\[ y^2 = 2^{1/2} \cdot 2^{1/2} = 2 \]
The last step follows from the basic rule for exponents of a common base.  Then
\[ y = \sqrt{y^2} = 2^{1/2} = \sqrt{2}  \]

So that's the first step in thinking about exponentials as functions:  $x$ can have values that are not integers.

The exponent be negative, as well.
\[ y = 2^{-x} \]
We can get some idea about this one by multiplying on top and bottom as follows:
\[ y = 2^{-x} \cdot \frac{2^{x}}{2^{x}} = \frac{2^0}{2^{x}} = \frac{1}{2^{x}} \]
So $2^{-x}$ is the inverse of $2^{x}$.
\begin{center} \includegraphics [scale=0.4] {exp2.png} \end{center}
In this figure the red curve is the graph of $y = 2^{-x}$ and the blue curve is $y = 2^x$.  They are mirror images.

This form of the exponential is called the negative exponential or exponential decay.




\end{document}
