\documentclass[11pt, oneside]{article} 
\usepackage{geometry}
\geometry{letterpaper} 
\usepackage{graphicx}
	
\usepackage{amssymb}
\usepackage{amsmath}
\usepackage{parskip}
\usepackage{color}
\usepackage{hyperref}

\graphicspath{{/Users/telliott/Github-Math/figures/}}
% \begin{center} \includegraphics [scale=0.4] {gauss3.png} \end{center}

\title{Standard binomial}
\date{}

\begin{document}
\maketitle
\Large

%[my-super-duper-separator]
Powers of $a + b$ can be written as
\[ (a + b)^1 = a + b \]
\[ (a + b)^2 = a^2 + 2ab + b^2 \]
\[ (a + b)^3 = a^3 + 3a^2b + 3ab^2 + b^3 \]
\[ \dots \]
\[ (a + b)^n = a^n + na^{n-1}b + \dots b^n \]
\[ (a + b)^{n+1} = a^{n+1} + (n+1)a^{n}b + \dots b^{n+1} \]

The ones on the ends are always just $a^3$ and $b^3$. All the others are the result of taking two sequential items in the previous row and multiplying the first by $b$ and the second by $a$.

For example
\[ 3a^2b = a^2 \cdot b + a \cdot 2ab \]

The standard binomial expansion or theorem is a general expression for powers $n$ of a binomial $a + b$:
\[ (a + b)^n = a^n + na^{n-1}b + \frac{n(n-1)}{2!}a^{n-2}b^2 + \dots \]
where $n$ is a positive integer.  

There will be $n+1$ terms in total.  The power of $a$ counts down from $n$ to $0$ while the power of $b$ counts up from $0$ to $n$.  For each term the sum of the powers of $a$ and $b$ is $n$.

If we observe the sequence of terms in the cofactor:
\[ 1, \ n, \ n \cdot (n-1) \dots \]
the $(n+2)$th term will have $(n-n) = 0$ as a factor so the series terminates and the previous term is the last.

The last term has $a^0 = 1$ and thus contains only $b^n$ with a cofactor that is always $1$ since $n \cdot (n-1) \dots 1 = n!$

The terms of the standard binomial can be observed in Pascal's triangle.  (The figure is from Courant and Robbins \emph{What is Mathematics?}).
\begin{center} \includegraphics [scale=0.8] {Courant_Pascal.png} \end{center}

\subsection*{n \emph{choose} k}

The formula above can be written as
\[ (a + b)^n = \sum_{k=0}^{k = n} \frac{n!}{(n-k)! \cdot k!} a^{n-k}b^k \]

The cofactor here is known as $n$ \emph{choose} $k$.  It gives the number of combinations of $k$ items from a total of $n$ objects.  

This formula is probably the reason why the factorial of $0$ is actually defined to be equal to $1$.

The \emph{choose} formula can be written symbolically.
\[  \binom{n}{k} \ = \frac{n!}{(n-k)! \cdot k!} \]

\subsection*{permutations and combinations}

The figure on the left below shows an example of a permutation of the first five natural numbers.  How many permutations are there?  

Starting with no number assigned, any one of the five might be placed into the first slot, followed by any one of the four that remain into the second slot, and so on.
\begin{center} \includegraphics [scale=0.5] {perms.png} \end{center}
\[ P = 5 \cdot 4 \cdot 3 \cdot 2 \cdot 1 = 5! \]
and in general the number of permutations $P$ of $n$ different objects is $n!$.

Now suppose that from a set of $n$ objects we have chosen $k$, that is, a combination $k$ from $n$ total.  Here, I did this by selecting dots in an order that I recorded, switching the background to white.

I might have picked any one of the positions first, then second and so on.  The number of ways to do this is:
\[ n \cdot (n-1) \dots \cdot (n-k+1) = \frac{n!}{(n-k)!}  \]

But suppose we do not care about the order in which the items were chosen.  If I pick 3 Bob Marley CD's from a total of 5 to take to a party, I don't really care \emph{which} one I picked first.

Our goal is simply to add up all the different ways to have $k$ powers of $a$ and $n-k$ powers of $b$).

The formula above over-counts.  The factor by which it over-counts is $k!$, the number of permutations of the items chosen specially.  To count the number of combinations $C$ the formula is:
\[ \binom{n}{k} \ = \frac{n!}{(n-k)! \cdot k!} \]

To pick 2 objects from a set of 5, the number of combinations is
\[ C = \frac{5 \cdot 4}{2!} = 10 \]

If they are labeled $a$ through $e$ we have
\[ ab \ ac \ ad \ ae  \]
\[ bc \ bd \ be \]
\[ \ cd \ ce \ \ \ \ \ \ \ de \]

There is another way to see that this should be the case.  I could have specified the previous problem by picking the filled squares first.  Then we would have not $k$ but $c$ where $c = n - k$. and
\[ \binom{n}{c} \ = \frac{n!}{(n-c)! \cdot c!} \]
But this is exactly the same since $c + k = n$.

$n-k$ and $k$ should contribute symmetrically to the result, and in our formula, they do.

We said earlier that any term $a^{n-k}b^k$ will be generated by taking two adjacent terms from the previous power and multiplying one by $a$ and one by $b$.  It makes it slightly easier to consider the $n+1$ row.

Then, the two multiplications are
\[ a^{n-k}b^k \cdot a \]
and
\[ a^{n-k+1}b^{k-1} \cdot b = a^{n-(k-1)}b^{k-1} \cdot b \]

The cofactors for these terms are $\binom{n}{k}$ and $\binom{n}{k-1}$.  We have
\[ \binom{n+1}{k} = \binom{n}{k} + \binom{n}{k-1} \]

The first term on the right gives the number of ways of picking $k$ individuals from $n$ possibles, and the second gives the number of ways of picking $k-1$ individuals from $n$ possibles.  

Suppose I have corralled a room full of $n$ people, and then you walk in.  I want to pick $k$ people (possibly including you) for a committee.  I can either first pick you and and then pick $k-1$ people from $n$;  alternatively I can exclude you and pick $k$ persons from $n$.  

The total number of ways to pick $k$ from $n+1$ is the sum of those committees which either have or do not have you as a member.  This is called Pascal's Lemma and it is the important part of the proof of the binomial theorem by induction.

\end{document}
