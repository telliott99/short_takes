\documentclass[11pt, oneside]{article} 
\usepackage{geometry}
\geometry{letterpaper} 
\usepackage{graphicx}
	
\usepackage{amssymb}
\usepackage{amsmath}
\usepackage{parskip}
\usepackage{color}
\usepackage{hyperref}

\graphicspath{{/Users/telliott/Dropbox/Github-Math/figures/}}
% \begin{center} \includegraphics [scale=0.4] {gauss3.png} \end{center}

\title{Infinite series}
\date{}

\begin{document}
\maketitle
\Large

%[my-super-duper-separator]

The most important elementary series is
\[ 1 + x + x^2 + \dots = \sum_{n=0}^{\infty} x^n \]
which can also be written as shown.

I saw a post by Steven Strogatz on Twitter that shows a clever way of looking at it.  Suppose we consider the distance from $0$ to $1$ and imagine that we will take multiple steps to move from $x = 0$ to $x = 1$.

\begin{center} \includegraphics [scale=0.4] {Strogatz2.png} \end{center}

We will use the rule that we always move a certain fraction of the distance remaining (like one of Zeno's paradoxes).  This means that in the first step, we leave a certain fraction of the distance yet to be covered, $r$.

In the second step, we have that the distance remaining is $r$, the fraction that we will actually move is $1-r$ and that distance corresponds to $r(1-r)$.  So the total distance moved so far is
\[ d = (1-r) + r(1-r) \]

If we continue forever we \emph{will} get to $1$
\[ 1 = (1 - r) + r(1-r) + r^2(1-r) + \dots \]

But then
\[ \frac{1}{1-r} = 1 + r + r^2 + \dots \]

And from our construction, clearly $0 < r < 1$, which is the standard restriction on $r$ so that the sum of the series is meaningful.

We might think about writing this another way, namely that at each stage we move a fraction of the distance remaining $f$ (with $f = 1 -r$).
\[ 1 = f + f(1-f) + f(1-f)^2 + \dots \]

Let $r = 1-f$ and then
\[ r = fr + fr^2 + \dots \]
\[ \frac{1}{f} = 1 + r + r^2 +  \dots \]
\[ \frac{1}{1-r} = 1 + r + r^2 + \dots \]


\end{document}
