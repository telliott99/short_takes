\documentclass[11pt, oneside]{article} 
\usepackage{geometry}
\geometry{letterpaper} 
\usepackage{graphicx}
	
\usepackage{amssymb}
\usepackage{amsmath}
\usepackage{parskip}
\usepackage{color}
\usepackage{hyperref}

\graphicspath{{/Users/telliott/Github/figures/}}
% \begin{center} \includegraphics [scale=0.4] {gauss3.png} \end{center}

\title{Irrationality of e}
\date{}

\begin{document}
\maketitle
\Large

%[my-super-duper-separator]

\subsection*{e is irrational}

I found a nice proof of the irrationality of $e$ in the calculus text by Courant and Robbins.  It is a proof by contradiction.  We start by assuming that $e$ is rational.
\[ e = \frac{p}{q}, \ \  p,q \in \mathbb{N} \]

We make use of the infinite series representation of $e$
\[ e = 1 + 1 + \frac{1}{2!}  + \frac{1}{3!} + \frac{1}{4!} + \cdots \]
From this, it is obvious that $e > 2$.

Here is a proof that $e < 3$.  Rewrite the rest of the series after the first two terms as
\[  \frac{1}{2}  + \frac{1}{6} + \frac{1}{24} + \cdots \]
But term-by-term that is smaller (after the first term) than
\[ \frac{1}{2}  + \frac{1}{4} + \frac{1}{8} + \cdots = 1 \]
So $e < 1 + 1 + 1 = 3$.  $\square$
 
Furthermore, if $e=p/q$ with $p$ and $q$ integers, and $2 < e < 3$, then $q>1$ which means that $q>=2$.

Equating the series representation to the rational fraction $p/q$:
\[ \frac{p}{q} = 1 + 1 + \frac{1}{2!}  + \frac{1}{3!} + \frac{1}{4!} + \cdots \]
Multiply both sides by $q!$.  

For the left-hand side, we have 
\[ e \ q! = \frac{p}{q} \ q! = p (q-1)! \]
We won't need to do anything more with the right-hand side, except to note that since $p (q-1)!$ is an integer, so also must $e\ q!$ be an integer.  

This is the series
\[ q! + q! + \frac{q!}{2!}  + \frac{q!}{3!}  + \cdots + \frac{q!}{(q-1)!} + \frac{q!}{q!} + \frac{q!}{(q+1)!} + \cdots \]
Now, 
$q!$ is obviously an integer. And for every integer $k < q$, $k!$ divides $q!$ evenly 
\[ \frac{q!}{k!} = q \times (q-1) \times (q-2) \cdots \times (q-k+1) \]
In our series
\[ q! + q! + \frac{q!}{2!}  + \frac{q!}{3!}  + \cdots + \frac{q!}{(q-1)!} + \frac{q!}{q!} + \frac{q!}{(q+1)!} + \cdots \]
all the terms to the left of $q!/(q-1)!$ are integers, as are $q!/(q-1)! = q$ and $q!/q! = 1$.  
\vspace{2 mm}

So now our concern is with the fractions that follow.  We will show that these sum up to something less than $1$.  We have
\[ \frac{1}{(q+1)} + \frac{1}{(q+1)(q+2)} + \frac{1}{(q+1)(q+2)(q+3)} + \cdots \]
Since $q >= 2$
\[ \frac{1}{(q+1)} <= \frac{1}{3} \]
\[ \frac{1}{(q+1)(q+2)} <= (\frac{1}{3})^2 \]
and so on, and the entire remaining series of fractions is less than or equal to
\[ \frac{1}{3} + (\frac{1}{3})^2 + (\frac{1}{3})^3 + \cdots \]
This is the geometric series with $r = 1/3$ and first term equal to r, and the sum is known to be
\[ \frac{1}{3} ( 1 / (1-\frac{1}{3}) ) = \frac{1}{2} \]
Since the right-hand side is equal to an integer plus something ``less than or equal to $\frac{1}{2}$", it is not an integer, and cannot be equal to the left-hand side, which is equal to an integer.  We have reached a contradiction.  

Therefore $e$ cannot be equal to $p/q$, for $p,q \in \mathbb{N}$.

\subsection*{proof 2}

I found another proof online

\url{https://twitter.com/TamasGorbe/status/1588200452880990210}

It is attributed to Fourier (1815).

We assume the infinite series expansion of $e$ (really, this is the definition of $e$):
\[ e = 1 + \frac{1}{1!} + \frac{1}{2!} + \dots \]

If $e$ is rational, we mean that $e = m/n$ for integers $m,n > 1$.  

Multiply both sides by $n!$
\[ n! \ e = (n-1)! \ m \]
Since the right-hand side is an integer, the left-hand side must be one also.  

Now, consider the following subtraction.
\[ n! \ e - n! \ [ \ 1 + \frac{1}{1!} + \frac{1}{2!} \dots + \frac{1}{n!} \  ] \]

The first term is an integer from what we just said.

The second term is also an integer because each term in the finite series is an integer.  

When we substitute the series form of $e$ and carry out the subtraction, what remains must be an integer also
\[ \frac{1}{n+1} + \frac{1}{(n+1)(n+2)} + \frac{1}{(n+1)(n+2)(n+3)}  + \dots \]

which is positive, but less than
\[ \frac{1}{n+1} + \frac{1}{(n+1)^2} + \frac{1}{(n+1)^3}  + \dots = \frac{1}{n} < 1 \]

That's a contradiction.  Hence, $e$ is not rational.

$\square$

The last step depends on the fact that this is a geometric series with common ratio $1/(n+1)$.  Since that term is between $-1$ and $1$, the series converges and its value is 

\[ \frac{1/(n+1)}{1 - 1/(n+1)} = \frac{1}{n} \]



\end{document}