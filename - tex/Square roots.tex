\documentclass[11pt, oneside]{article} 
\usepackage{geometry}
\geometry{letterpaper} 
\usepackage{graphicx}
	
\usepackage{amssymb}
\usepackage{amsmath}
\usepackage{parskip}
\usepackage{color}
\usepackage{hyperref}

\graphicspath{{/Users/telliott/Dropbox/Github-Math/figures/}}
% \begin{center} \includegraphics [scale=0.4] {gauss3.png} \end{center}

\title{Square roots}
\date{}

\begin{document}
\maketitle
\Large

%[my-super-duper-separator]

\emph{Theorem}:  

\textbf{If a positive integer has a square root that is a rational number, it must be a perfect square.}

Alternative formulations:  

$\circ$ \ Every positive integer that is not a perfect square has an irrational square root.  

$\circ$ \ Every rational square root is also a positive integer.

We will use the \emph{fundamental theorem of arithmetic}:  every positive integer has a unique prime factorization.  Thus
\[ a = p_1 \cdot p_2 \cdot p_3 \dots \]

The factors $p_1$ etc. are not necessarily all different.  Example:
\[ 36 = 2 \cdot 2 \cdot 3 \cdot 3 \]

\emph{Proof}.

Let $n$ be a positive integer whose square root is a rational number, i.e. the ratio of two positive integers $a$ and $b$:
\[ \sqrt{n} = \frac{a}{b} \]

Square both sides and rearrange:
\[ n \cdot b^2 = a^2 \]

By the preliminary result, $a$ has a unique prime factorization:
\[ a = p_1 \cdot p_2 \cdot p_3 \dots \]

So
\[ a^2 = p_1^2 \cdot p_2^2 \cdot p_3^2 \dots \]

The number of copies of each distinct factor of $a^2$ is even, and the total number of factors is also even.  

Because of the equality $nb^2 = a^2$, we write:
\[ n \cdot b^2 =  p_1^2 \cdot p_2^2 \cdot p_3^2 \dots \]

But $b$ is a positive integer, so
\[ b^2 = q_1^2 \cdot q_2^2 \cdot q_3^2 \dots \]
thus
\[ n \cdot q_1^2 \cdot q_2^2 \cdot q_3^2 \dots =  p_1^2 \cdot p_2^2 \cdot p_3^2 \dots \]

This means that every $q$ on the left-hand side must be one of the $p$'s on the right-hand side.

Let us factor out all the $q$'s from both sides, leaving
\[ n =  p_x^2 \cdot p_y^2 \cdot p_z^2 \dots \]

Every prime factor of $n$ occurs an even number of times on the right-hand side.  But this means that $n$ must be a perfect square.  

We cannot have 
\[ 3 =  p_x^2 \cdot p_y^2 \cdot p_z^2 \dots \]

because $3$ is not a perfect square.  Furthermore, we can't have
\[ 6 =  p_x^2 \cdot p_y^2 \cdot p_z^2 \dots \]
\[ 2 \cdot 3 =  p_x^2 \cdot p_y^2 \cdot p_z^2 \dots \]

The prime factorization of any integer is unique.  But $2$ and $3$ are present only once on the left-hand side, and each $p$ is present an even number of times on the right-hand side.  This is a contradiction.

$n$ must be of the form
\[ n = p_x^2 \cdot p_y^2 \dots = (p_x \cdot p_y \dots)^2 \]

That is, a perfect square.

$\square$

\end{document}
