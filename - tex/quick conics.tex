\documentclass[11pt, oneside]{article} 
\usepackage{geometry}
\geometry{letterpaper} 
\usepackage{graphicx}
	
\usepackage{amssymb}
\usepackage{amsmath}
\usepackage{parskip}
\usepackage{color}
\usepackage{hyperref}

\graphicspath{{/Users/telliott/Dropbox/Github-math/figures/}}
% \begin{center} \includegraphics [scale=0.4] {gauss3.png} \end{center}

\title{Equations for conics}
\date{}

\begin{document}
\maketitle
\Large

Here we derive the equations for parabola, ellipse and hyperbola in standard orientation.

\subsection*{parabola}
For a parabola in standard orientation and centered at the origin, the focus lies on the $y$-axis a distance $c$ above the origin, while the directrix is a line parallel to the $x$-axis the same distance $c$ below it.

Any parabola can be rotated if necessary, then translated to the origin in standard orientation with shape factor $a$ and equation $y = ax^2$.

For any point on the parabola, the distance to the focus and the vertical distance down to the directrix are equal.
\[ \sqrt{x^2 + (y-c)^2} = y + c \]
We can substitute for either $x$ or $y$.  Normally we choose $x$, so why not try $y$.
\[ \sqrt{\frac{y}{a} + (y-c)^2} = y + c \]
\[ \frac{y}{a} + (y - c)^2 = (y + c)^2 \]
\[ \frac{y}{a} = 4cy \]
\[ 4ac = 1 \]
The focus $c = 1/(4a)$.

\subsection*{ellipse}
For any point on the ellipse, the sum of the distances to the foci is constant.  Let us call that constant $2a$, we will find that $a$ is an important value in the standard graph, later.
\[ L_1 = \sqrt{(x - c)^2 + y^2} \]
\[ L_2 = \sqrt{(x + c)^2 + y^2} \]

So
\[ \sqrt{(x - c)^2 + y^2} + \sqrt{(x + c)^2 + y^2} = 2a \]
Rearrange
\[ \sqrt{(x - c)^2 + y^2} =  2a - \sqrt{(x + c)^2 + y^2} \]
\[ (x - c)^2 + y^2 = 4a^2 - 4a \sqrt{(x + c)^2 + y^2} + (x + c)^2 + y^2 \]
\[ -4xc = 4a^2 - 4a \sqrt{(x + c)^2 + y^2}  \]
\[ a \sqrt{(x + c)^2 + y^2} = a^2 + xc \]
\[ a^2 (x^2 + 2xc + c^2 + y^2) =  a^4 + 2a^2xc + x^2c^2 \]
The terms with a factor of $2$ cancel.
\[ a^2x^2 + a^2c^2 + a^2y^2 = a^4 + x^2c^2 \]
\[ (a^2 - c^2) x^2 + a^2 y^2 = (a^2 - c^2)a^2 \]
Define $a^2 - c^2 = b^2$
\[ b^2 x^2 + a^2 y^2 = b^2 a^2 \]
Divide by $a^2b^2$ and we're done.

We can see that if $y = 0$, then $x = \pm \ a$.  

Similarly, if $x = 0$ then $y = \pm \ b$.

\subsection*{hyperbola}
For the hyperbola, the math is very similar to the ellipse.  The lengths are the same, but it is the absolute value of the difference that is constant.

We have
\[ L_1 - L_2 = \pm \ 2a \]
\[ \sqrt{(x + c)^2 + y^2} - \sqrt{(x - c)^2 + y^2} =  \pm \ 2a \]
Rearrange
\[ \sqrt{(x + c)^2 + y^2} =  \pm \  2a + \sqrt{(x - c)^2 + y^2} \]
\[ (x + c)^2 + y^2 = 4a^2 \pm \  4c \sqrt{(x + c)^2 + y^2} + (x - c)^2 + y^2 \]
\[ 4xc = 4a^2 \pm \ 4a \sqrt{(x - c)^2 + y^2}  \]
\[ xc - a^2 = \pm \ a \sqrt{(x - c)^2 + y^2}  \]
\[ x^2c^2 - 2xca^2 + a^4 = a^2(x^2 - 2xc + c^2 + y^2) \]
The terms with a factor of $2$ cancel.
\[ x^2c^2 + a^4 = a^2x^2 + a^2c^2 + a^2y^2 \]
\[ x^2(c^2 - a^2) - a^2 y^2 = a^2(c^2 - a^2) \]
Define $b^2 = c^2 - a^2$
\[ x^2 b^2 - a^2 y^2 = a^2 b^2 \]
Divide by $a^2b^2$ and we're done.
\[ \frac{x^2}{a^2} - \frac{y^2}{b^2} = 1 \]

And we see that again, when $y = 0$, $x = \pm \ a$.  Rearranging
\[ \frac{x^2}{a^2} = 1 + \frac{y^2}{b^2} \]
When $x$ is large
\[ \frac{x^2}{a^2} \approx \frac{y^2}{b^2} \]
\[ y \approx \pm \ \frac{b}{a} x \]
This is the equation of a straight line with slope $b/a$ through the origin.

For the ellipse, we define $a^2 - c^2 = b^2$, and this makes perfect sense geometrically when considering a point at say $(0,b)$.  There is a right triangle with sides $b$ and $c$ and hypotenuse $a$.

Presumably, there is a geometrical explanation for $a^2 + b^2 = c^2$ in the case of the hyperbola.

\end{document}