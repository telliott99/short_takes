\documentclass[11pt, oneside]{article} 
\usepackage{geometry}
\geometry{letterpaper} 
\usepackage{graphicx}
	
\usepackage{amssymb}
\usepackage{amsmath}
\usepackage{parskip}
\usepackage{color}
\usepackage{hyperref}

\graphicspath{{/Users/telliott/Dropbox/Github-Math/figures/}}
% \begin{center} \includegraphics [scale=0.4] {gauss3.png} \end{center}

\title{Phi}
\date{}

\begin{document}
\maketitle
\Large
\subsection*{definition}
This is a short introduction to the "golden ratio" or golden mean.

One way to start is to take an arbitrary length and pick a point on it, with one section of length $a$ and the other of length $b$.  Let $a$ be larger:  $a > b$.
\begin{center} \includegraphics [scale=0.5] {golden_ratio.png} \end{center}
Then, the definition of the golden ratio is:
\[ \frac{a}{b} = \frac{a + b}{a} \]

We can also do a similar thing with rectangles.
\begin{center} \includegraphics [scale=0.3] {goldenratioab.png} \end{center}
Draw a square with sides of length $a$, and extend it a distance $b$ to form a rectangle of side length $a + b$.  We want the proportions of the two rectangles (length divided by width) in the figure to be the equal, and this gives the same equation as before.

Going back to that equation, since the length is arbitrary, we can scale it so $b = 1$.  Substitute $x$ for $a$:
\[ x = \frac{x + 1}{x} \]
\[ x^2 = x + 1 \]

This is a quadratic equation.  Rearranging:
\[ x^2 - x - 1 = 0 \]

When the value of the function is plotted on the $y$-axis against $x$, we have a parabola opening up.  Furthermore, for $x = 0$ or $x = 1$, the value of the function is $-1$.  (Symmetry then demands that the vertex is at $x = 1/2$, as we will see).

If we didn't yet know where the vertex lies, we would still know that the curve must pass the $x$-axis in two places going up (since the value at $x = 0$ and $x = 1$ is less than zero).  There are two real roots, values of $x$ resulting in an output of $zero$.
\begin{center} \includegraphics [scale=0.4] {phi_plot.png} \end{center}

If you know the quadratic equation this is a great place to use it.  Otherwise, there is another strategy.  Without proof, we will say that the equation can be written in another form:
\[ (x - s)(x - t) = 0 \]
since we know the highest power of $x$ is $x^2$, and there are two values of $x$ that give $0$.  

These values $s$ and $t$ are symmetrical around the axis of symmetry of the parabola, and the vertex.  Let us call the $x$-value of the vertex, $m$.  

Then, by symmetry, $m$ is the average of $s$ and $t$.  That is, $m = (s + t)/2$.  Now multiply out:

\[ (x - s)(x - t) = 0 \]
\[ x^2 - (s + t) + st = 0 \]
Comparing with
\[ x^2 - x - 1 = 0 \]
$(s + t) = 1$ so
\[ m = \frac{s + t}{2} = \frac{1}{2} \]

The points $(s,0)$ and $(t,0)$ are equidistant from $m$, let us call that distance $d$.  So $s = m - d$ and $t = m + d$ and 
\[ st = (m - d)(m + d) \]
\[ = m^2 - d^2 = -1 \]
That last value comes from the first form of the equation, what we would call $c$ if we were writing it as $ax^2 + bx + c$.

Hence 
\[ d^2 = m^2 + 1 = \frac{1}{4} + 1 \]
\[ d = \pm \sqrt{5/4} = \pm \frac{\sqrt{5}}{2} \]

We want the solution with $x > 1$, hence we take the positive root:
\[ x = t = m + d = \frac{1}{2} + \frac{\sqrt{5}}{2} \]
\[ = \frac{1}{2} (1 + \sqrt{5}) \]

\subsection*{$\phi$ or phi}

This number, which is usually written with the Greek letter $\phi$, pronounced "FIE", has a value of about 1.618.

As we said, $\phi$ really is \emph{defined} by the equation
\[ \phi^2 = \phi + 1 \]

We confirm the arithmetic:
\[ \phi^2 = (\frac{1}{2} (1 + \sqrt{5}))^2 \]
\[ = \frac{1}{4} (1 + 5 + 2 \sqrt{5}) \]
\[ =  \frac{1}{2} (3 + \sqrt{5}) = 1 +  \frac{1}{2} (1 + \sqrt{5}) \]
\[ = \phi + 1 \] 

$\phi$ is an \emph{irrational} number.  One proof depends on the fact that $\sqrt{5}$ is itself irrational.  (We have discussed this elsewhere).

Now, if $x$ is irrational and $p/q$ a rational number, then the result of adding $p/q$ to $x$ must be irrational.  

\emph{Proof}.  

Suppose instead that $x + p/q$ is rational.  Then $x$ is equal to a rational number minus $p/q$ which would also be rational.  But that's just $x$, which was supposed to be irrational.  This is a contradiction.  Therefore $\sqrt{5} + 1$ is irrational.  

$\square$

A similar argument will show that $(\sqrt{5} + 1)/2 = \phi$ is also irrational. 

However, do not make the mistake of thinking irrational times irrational is irrational, since we can easily come up with counter-examples such as $\sqrt{2} \cdot \sqrt{2} = 2$.

\subsection*{continued fraction representation}

Let us construct the continued fraction representation of $\phi$.
\[ \phi^2 = \phi + 1 \]
\[ \phi = 1 + \frac{1}{\phi} \]

The trick here is to see that the \emph{entire} right-hand side of the second equation above is equal to $\phi$ so it can be substituted into the denominator of the second term on that side.
\[ \phi = 1 + \frac{1}{1 + 1/\phi} \]

But we can play this game forever.
\[ \phi = 1 + \cfrac{1}{1+\cfrac{1}{1+\cfrac{1}{1 + \dots}}}  \]

The unending continuation means that to evaluate this, we must at some point decide to neglect the remaining terms.  If we only take a finite number of terms, there will always be some difference from the true value of $\phi$.   

Going back up the chain, inverting at each step, we start with $1$ and then obtain

\begin{verbatim}
1/1 + 1 = 2/1
1/2 + 1 = 3/2
2/3 + 1 = 5/3
3/5 + 1 = 8/5
5/8 + 1 = 13/8
\end{verbatim}

You may recognize these numbers as coming from the Fibonacci sequence.  As we continue out in the Fibonacci numbers, $F_n/F_{n-1}$ becomes a better and better approximation to the value of $\phi$.

\begin{center} \includegraphics [scale=0.3] {FibonacciBlocks.png} \end{center}

\subsection*{rational approximation}
We just looked at the continued fraction.  Another, simple-minded idea for obtaining the value of $\phi$ is to manipulate the equation:
\[ x^2 - x - 1 = 0 \]
\[ x = \frac{x + 1}{x} \]

We know that $1 < x < 2$ from the geometry of the rectangle.  If we plug $3/2$ as a guess for $x$ in on the right-hand side and use the output as the new value:
\[ x = \frac{3/2 + 1}{3/2} = \frac{5}{3} \]
\[ x = \frac{5/3 + 1}{5/3} = \frac{8}{5} \]

\begin{verbatim}
 x       x+1      x+1      x^2        d
 3/2     5/2     10/4      9/4     + 1/4
 5/3     8/3     24/9     25/9     - 1/9
 8/5    13/5     65/25    64/25    + 1/25
13/8    21/8    168/64   169/64    - 1/64
\end{verbatim}

We can see that this method gives the numbers of the Fibonacci sequence.  There is a cyclic variation where on one iteration $x+1 > x^2$ and then the next time $x+1 < x^2$.  At each step the difference ($d$) gets smaller, it goes like $1$ over the square of the denominator.

However, other rearrangements do not work as well:
\[ x = \sqrt{x + 1} \]
\[ x = x^2 - 1 = (x + 1)(x - 1) \]
There's a reason we got lucky but I'm not sure what it is.

\subsection*{Newton's method}
Our graph is
\[ y = x^2 - x - 1 \]
In calculus, we learn to compute the slope of the curve at any point $(x,f(x))$
\[ f'(x) = 2x - 1 \]

\begin{center}
\includegraphics [scale=0.5] 
{phi_plot2.png} 
\includegraphics [scale=0.3] 
{phi_plot3.png} 
\includegraphics [scale=0.5] 
{phi_plot4.png} 
\end{center}

We want to write the equation of the line extending from the point $x,y$ back down to the $x$-axis at $(x_0,0)$. $x_0$ should be a much better guess as to the value of $\phi$, which is where $y = 0$.

\[ f'(x) = 2x - 1 = \frac{f(x) - 0}{x - x_0} \]
\[ x_0 = x - \frac{x^2 - x - 1}{2x - 1} \]
\[ = \frac{2x^2 - x - x^2 + x + 1}{2x - 1} \]
\[ = \frac{x^2 + 1}{2x - 1} \]

For $3/2$ we have
\[ x_0 =  \frac{13/4}{2} = \frac{13}{8} \]
which is three steps along the sequence we had before.

To put this another way, if the slope is $2x - 1$ and $x = 3/2$ then the slope is $2$ and the equation of the line is
\[ y = 2x + y_0 \]
where $x = 3/2$ and $y = -1/4$ so
\[ -\frac{1}{4} = 2 \frac{3}{2} + y_0 \]
\[ y_0 = -3 \ \frac{1}{4} = -\frac{13}{4} \]

For $5/3$ we have
\[ x_0 =  \frac{25/9 + 9/9}{7/3} = \frac{34}{21} \]
which is \emph{four} steps along the sequence we had before.

It gets better and better!

\subsection*{note}

If you're trying to impress your friend, or a student, and reconstruct the above argument from the beginning, beware.  It has happened to me that I switched the roles of $a$ and $b$, or $x$ and $1$.  Then, you'll get a slightly different equation whose solution is the positive inverse of $\phi$.


\end{document}
