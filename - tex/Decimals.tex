\documentclass[11pt, oneside]{article} 
\usepackage{geometry}
\geometry{letterpaper} 
\usepackage{graphicx}
	
\usepackage{amssymb}
\usepackage{amsmath}
\usepackage{parskip}
\usepackage{color}
\usepackage{hyperref}

\graphicspath{{/Users/telliott/Dropbox/Github-math/figures/}}
% \begin{center} \includegraphics [scale=0.4] {gauss3.png} \end{center}

\title{Decimals}
\date{}

\begin{document}
\maketitle
\Large

%[my-super-duper-separator]

In pre-algebra, we start to think about the decimal system in a serious way.  Decimal is a method for writing any number.  For example, $1/10$ becomes $0.1$, $1/4$ becomes $0.25$ and their product, $1/40 = 0.025$.

You object, saying you already know how to write numbers, as integers and fractions.  Why add another way?

Rational is a fancy name for the numbers we've been calling fractions.  We don't mean 'sane'.  We mean rational as in ratio.  And integers are also rational, with $1$ as the denominator.  But it turns out that some numbers cannot be written as fractions or ratios.  These are called irrational numbers.  

One example is the square root of $2$, written $\sqrt{2}$, and defined to be the number which, when multiplied by itself, gives $2$.
\[ \sqrt{2} \cdot \sqrt{2} = 2 \]
For the perfect squares, the square root is what you expect.  For example $2 = \sqrt{4}$.

Another irrational number is $\pi$, the ratio of the circumference of a circle to its diameter, $C = \pi d$.

A second reason is that using decimals makes adding (and subtracting) fractions a lot easier, because it gets rid of the need to find a common denominator.

\subsection*{decimals}

Of course, decimal notation is introduced pretty early for whole numbers, or integers.  It's a positional notation, the value of a symbol depends on where it occurs in the number.
\[ 2042 = 2(1000) + 0(100) + 4(10) + 2 \]

The $2$ on the left is worth $2000$, but the other one is just $2$.  Our example uses the digit $0$, which acts as a place-holder showing there are no hundreds to add.  Without $0$, we would not know that the leading $2$ is $2000$ rather than $200$.

All of this is pretty familiar.  You likely also know about powers, or exponents.  Rather than write $1000 = 10 \cdot 10 \cdot 10$, we write $1000 = 10^3$.  The $3$ indicates how many $10$s must be multiplied together to give the product $1000$.

Rewriting the previous expression, and proceeding from smaller numbers to larger ones:
\[ 2042 = 2 + 4(10^1) + 0(10^2) +  2(10^3) \]

\subsection*{decimal point}

As we said at the beginning, $1/10$ becomes $0.1$.  So adding to our example
\[ 2042 + 0.13 = 2042.13 \]
\[ = 3(10^{-2}) + 1(10^{-1}) + 2 + 4(10^1) + 0(10^2) +  2(10^3) \]
The pattern of the exponents is interesting.  It looks a lot like the number line for integers.
\begin{center} \includegraphics [scale=0.4] {number_line2.png} \end{center}

There is one place that doesn't follow the pattern.  In the middle we wrote $+\ 2\ +$.  Before long we will write $+ \ 2(10^0) \ +$, with the definition that $10^0 = 1$.  If you haven't seen that yet don't worry too much about it yet.

One last thing about the decimal point.  When a number written in decimal is less than $1$ but still positive, it is customary to write a $0$ before the decimal point.  As in
\[ 2042.13 - 2042 = 0.13 \]

\subsection*{fractions}

Fractions are converted to decimals by long division.  I presume you've seen this.

\begin{verbatim}
    0.25
    ______
4 | 1.000
    0.8
    ---
    0.20
    0.20
    -----
       00
\end{verbatim}

Some fractions convert to decimals that end neatly in zero.  Other fractions do not.  $1/3 = 0.{\bar{3}}$, where the bar mean that the threes continue forever.  This may sometimes be written as $0.3333 \dots$, where the dots mean something else is coming, but we haven't specified what it is.

If you carry out long division for other fractions, you'll find that the repeats are longer.  For example 
\[ \frac{1}{7} = 0.142857142857 \dots \]
A repeat can never be as long as the denominator.  Here, the length of the repeat is $6$.

You may have seen this:
\[ 3 \cdot \frac{1}{3} = 3 \cdot 0.3333 \dots = 0.9999 \dots = \stackrel{?}{1} \]
How can $0.9999 \dots$ be \emph{equal} to $1$?  One answer is that we can make the difference between $0.9999 \dots$ and $1$ as small as we like, by taking as many nines as necessary.
\[ 1 - 0.99 > 1 - 0.9999999999 \]

Another approach is to convert $0.\bar{9}$ back into a fraction.  It goes like this:

\[ 10 (0.9999 \dots) = 9.999 \dots \]
Subtract $0.9999 \dots$ from both sides:
\[ 9 (0.9999 \dots) = 9 \]
\[ 0.9999 \dots = \frac{9}{9} = 1 \]

\subsection*{truncation}

Let's go back to $\sqrt{2}$.  What is the value of this expression in decimal?

One approach is trial multiplication.  Like this
\[ 1^2 = 1, \ \ \ \ \ 2^2 = 4 \]
We know that $1 < \sqrt{2} < 2$.  Then
\[ 1.4^2 = 1.96, \ \ \ \ \ 1.5^2 = 2.25 \]
We know that $1.4 < \sqrt{2} < 1.5$.  Then
\[ 1.41^2 = 1.9881, \ \ \ \ \ 1.42^2 = 2.0164 \]
We know that $1.41 < \sqrt{2} < 1.42$

This can be continued forever.  You will never find a decimal that will give \emph{exactly} $2$.

At each stage, the value of $\sqrt{2}$ is bracketed between one number which is smaller than $2$, and one which is greater.

To 15 places, 
\[ \sqrt{2} = 1.414213562373095 \dots \]

Here, the dots mean that the digits never stop, but the pattern appears to be random.

In order to actually calculate with $\sqrt{2}$ or $\pi$, we have to choose where to stop.  This is called rounding.

To 15 places, $\pi = 3.14159265358979 \dots$

Maybe $3.14$ is enough.  The error is $0.00159265358979 \dots$.

Or perhaps you need more accuracy and decide to use $3.141$.

There is a standard thing people do to reduce the error, called \emph{rounding up}.  If the digit in the next place is $9$, as in $3.14159 \dots$, the convention is to replace the $5$ with a $6$.  That is, the rule is to 

$\bullet$ \ round up if the next digit is $\ge 5$.  

To four places (after the decimal point)
\[ \pi = 3.1416 \]
The first time you round up for the previous example is
\[ \sqrt{2} = 1.4142136 \]
That $6$ on the end was rounded up from $5$, because the next digit after $5$ was $\ge 5$, namely $6$.

\subsection*{other bases}
Did you ever think, when we wrote
\[ 2042 = 2 \cdot 1000 + 0 \cdot 100 + 4 \cdot 10 + 2 \]
Why did we choose $10$?  After all, that's what decimal means.  Because we have $10$ fingers, maybe?  In any event, $10$ is a choice, and it's possible to make other choices.

One common choice is binary, where numbers are written with only $0$ and $1$.  For example
\[ 5 = 1 \cdot 2^2 + 0 \cdot 2^1 + 1 \cdot 2^0 \]
which is written as $101$ or sometimes $0b \ 101$ where the $0b$ means, what's next is a binary number.

Another common choice is hexadecimal, with 16 digits.  After $0..9$ we add $abcdef$.  

\begin{verbatim}
0x 1 = 1 
..
0x 9 = 9
0x a = 10
0x b = 11
0x c = 12
0x d = 13
0x e = 14
0x f = 15
\end{verbatim}

For example
\[ \\0x \ 32 = 3 \cdot 16^1 + 2 \cdot 16^0 = 50 \]
\[ \\0x \ af = 10 \cdot 16^1 + 15 \cdot 16^0 = 175 \]
Although everything is $0$ and $1$ inside the computer, to save space and aid recognition, binary numbers are usually translated into hexadecimal.  

\begin{verbatim}
68 65 6c 6c 6f 20 77 6f 72 6c 6
\end{verbatim}
means "hello world".  But that's another story and a good place to stop.


\end{document}