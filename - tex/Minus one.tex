\documentclass[11pt, oneside]{article} 
\usepackage{geometry}
\geometry{letterpaper} 
\usepackage{graphicx}
	
\usepackage{amssymb}
\usepackage{amsmath}
\usepackage{parskip}
\usepackage{color}
\usepackage{hyperref}

\graphicspath{{/Users/telliott/Dropbox/Github-Math/figures/}}
% \begin{center} \includegraphics [scale=0.4] {gauss3.png} \end{center}

\title{Minus one}
\date{}

\begin{document}
\maketitle
\Large

%[my-super-duper-separator]

You're familiar with the number line.

\begin{center} \includegraphics [scale=0.3] {number_line.png} \end{center}

Here we see the result of $5 + 4$.  Start on $0$, facing right.  First take $5$ steps forward, then $4$ more.  The result is $9$.

Negative numbers extend to the left of $0$ on the number line (original line on top and the extended line below).
\begin{center} \includegraphics [scale=0.3] {number_line2.png} \end{center}

The concept of negative numbers seems a bit strange at first.  How can you have $-3$ sheep?  

But if you first owe someone \$$3$ and then you earn \$$5$, after returning your friend's money you'll have $5-3 = 2$ dollars.  You could think of this as you had -\$$3$ (you had no money but also owed money) and then got paid \$$5$ so you have $-3 + 5 = 2$ dollars.

For multiplying with negative numbers we first have the rule that 
\[ -1 \cdot n = -n \]
for every positive number $n$.  For example $-1 \cdot 2 = -2$.

And since multiplication is \emph{commutative} (you can reverse the order of the terms without changing the result)
\[ n \cdot -1 = -n \]
for every positive number $n$.  For example $2 \cdot -1 = -2$.

The big question is what to do with 
\[ -1 \cdot -1 \stackrel{?}{=} \]

We're going to define this to be equal to something, and there are really only two choices we might think about for a possible answer:  either $1$ (that is, $+1$) or $-1$.  Let's think about the second possibility and see why that causes trouble.

\subsection*{what's wrong with this}

We're going to find that if we defined the above result as equal to $-1$, we run into a conflict with the fundamental properties of arithmetic.

One property is that $1$ is the multiplicative identity, meaning that for any number $n$

\[ 1 \cdot n = n \]

Also, when multiplying any equation by the same number on both sides, equality is maintained.  Multiplying the above by $1/n$ we get

\[ 1 = \frac{n}{n} \]

This second rule says that any number divided by itself is equal to $1$.

So let's look at

\[ -1 \cdot -1 = -1 \]

Using $1$ as the multiplicative identity, we have that $1 \cdot -1 = -1$ and equating that with our definition we have that

\[ -1 \cdot -1 = 1 \cdot -1 \]

Now multiply by $1/-1$, clearing one copy of $-1$ on each side.  (Here we are using another fundamental property, associativity).

This leaves us with
\[ -1 = 1 \]

That's a real problem, what mathematicians call a \emph{contradiction}.  Choosing $-1$ as the answer in the original equation messes up our mathematical universe.  Therefore, we are forced to define

\[ -1 \cdot -1 = 1 \]

[ Note:  the rule above was that any number divided by itself is equal to $1$.  The exception is that  $0/0$ is not defined.  It is not \emph{permitted}.  We'll explain why in a separate writeup. ]

Strogatz (in \emph{Joy of x}) shows this pattern

\[ 2 \cdot -1 = -2 \]
\[ 1 \cdot -1 = -1 \]
\[ 0 \cdot -1 = 0 \]
\[ -1 \cdot -1 = ? \]

The pattern is clear.  The answer is $+1$.

Here is a table
\begin{verbatim}
    -2 -1  0  1  2
    --------------
-2 |       0 -2 -4
-1 |    1  0 -1 -2
 0 | 0  0  0  0  0
 1 |-2 -1  0  1  2
 2 |-4 -2  0  2  4
 \end{verbatim}
 
Another explanation I like comes from Stewart (\emph{Letters to a Young Mathematician}).  Suppose we look at 
\[ (-1) \cdot (1 - 1) = 0 \]
Clearly, this is zero, because the sum inside the parentheses is 0.

But suppose we expand it using the distributive law.  The first term is $-1 \cdot 1 = -1$.  If we put that $-1$ on the right-hand side it becomes $+1$ and then we have:
\[ -1 \cdot -1 = 1 \]

\end{document}
