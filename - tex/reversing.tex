\documentclass[11pt, oneside]{article} 
\usepackage{geometry}
\geometry{letterpaper} 
\usepackage{graphicx}
	
\usepackage{amssymb}
\usepackage{amsmath}
\usepackage{parskip}
\usepackage{color}
\usepackage{hyperref}

\graphicspath{{/Users/telliott/Github-Math/figures/}}
% \begin{center} \includegraphics [scale=0.4] {gauss3.png} \end{center}

\title{Reversing series}
\date{}

\begin{document}
\maketitle
\Large

%[my-super-duper-separator]

Dunham (\emph{The Calculus Gallery}, chapter 1) has an introduction to Newton's binomial theorem plus some applications of series to approximation of the digits of $\pi$, to integration, and to writing series such as the ones for $\sin x$ and $\cos x$.

One of the tools discussed is that of inverting a series.  Wolfram calls this reversing a series.  

Consider one with $z$ as a function of $x$
\[ z = \frac{x}{1 + x} = x - x^2 + x^3 - x^4 \dots \]
and a second with $x$ as a function of $z$.
\[ x = \frac{z}{1 - z} = z + z^2 + z^3 + z^4 \dots \]

The second is obtained from the geometric series, multiplied by $x$.  The first is also the geometric series, with $-x = z$ and then multiplied by $z$.

Suppose we just knew the first one and we wanted to write its inverse.  Given the closed form, we can write:
\[ \frac{1}{z} = \frac{1}{x} + 1 \]
\[ \frac{1}{z} - 1 = \frac{1}{x} \]
\[ \frac{z}{1 - z} = x \]

Dunham uses this example to discuss how Newton found inverse functions for series that he knew.

Suppose we knew only that
\[ z =  x - x^2 + x^3 - x^4 \dots \]
Clearly, $z$ is approximately equal to $x$, especially since these series are only valid for $|x| < 1$.  There is a correction term.
\[ z + p = x \]
Substituting for $x$
\[ z = (z + p) - (z + p)^2 + (z + p)^3 - (z + p)^4 \dots \]
Grouping by powers of p we have
\[ z = z +  (-z^2 + z^3 - z^4 + \dots)p^0 + \]
\[ + \ (1 - 2z + 3z^2 - 4z^3 + \dots)p^1 +  \]
\[ + \ (-1 + 3z - 6z^2 + 10z^3 + \dots)p^2 + \dots \]

Now throw away everything except the cofactors of $p^0$ and $p^1$, which gives
\[ p = \frac{z^2 - z^3 + z^4 + \dots}{1 - 2z + 3z^2 - 4z^3 + \dots} \]

 Now, further, keep only the lowest powers of $z$ (i.e. $z^2$ in the numerator and $1$ in the denominator so that leaves just
 \[ p \approx z^2 \]

Write
\[ p = z^2 + q \]
\[ x = z + z^2 + q \]
where $q$ is a second error term.

If you substitute into the monster above and multiply out and then weed and cull, you will find that
\[ q = z^3 + r \]
where $r$ is a third error term.  Do this enough times and you can convince yourself that 
\[ x = z + z^2 + z^3 + \dots \]

Wolfram

\url{https://mathworld.wolfram.com/SeriesReversion.html}

gives these formulas (from references) for the cofactors of the two series.  If the cofactors of $x$ are $a$:
\[ a_1x + a_2x^2 + a_3x^3 + \dots \]
and those of $z$ are
\[ A_1z + A_2z^2 + A_3z^3 + \dots \]

Then
\[ A_1 = \frac{1}{a_1}  \ \ \ \ \ \  A_2 = - \frac{a_2}{a_1^3}  \ \ \ \ \ \  A_3 = \frac{2a_2^2 - a_1a_3}{a_1^5} \]
\[ A_4 = \frac{5 a_1 a_2 a_3 - a_1^2 a_4 - 5 a_2^3}{a_1^7} \]
\[ A_5 = \frac{6a_1^2 a_2 a^4 + 3a_1^2a_3^2 + 14a_2^4 - a_1^3a_5 - 21 a_1a_2^2a^3}{a_1^9} \]

They give two more terms, but I think this is enough.  In our example, all of the $a_n$ are just one, the even ones are minus.  So
\[ A_1 = 1 \ \ \ \ \ \ A_2 = -1 \ \ \ \ \ \ A_3 = 1 \]
\[ A_4 = -1 \ \ \ \ \ \ A_5 = 1 \]
Not much of a test, I suppose.  But it could have given wrong answers.

Recall the series for $\sin^{-1} x$:
\[ = 1 + \frac{1}{6}x^3 + \frac{3}{40} x^5 + \frac{5}{112} x^67 \]

\[ a_1 = 1 \ \ \ \ \ \ a_2 = 0 \ \ \ \ \ \ a_3 =  \frac{1}{6} \]
\[ a_4 =  0  \ \ \ \ \ \ a_5 =  \frac{3}{40} \]

so 

\[ A_1 = 1 \ \ \ \ \ \ A_2 = 0 \ \ \ \ \ \ A_3 = -a_1 a_3 =  -\frac{1}{6} = \frac{1}{3!} \]
\[ A_4 = 0 \ \ \ \ \ \ A_5 = 3 a_1^2 a_3^2 - a_1^3 a_5 =  \frac{1}{120} = \frac{1}{5!}  \]

We have
\[ z - \frac{1}{3!} z^3 + \frac{1}{5!} z^5 + \dots \]
which is indeed the series for $\sin z$.

\subsection*{Newton's method for $\sin z$}

Dunham says that Newton derived the famous series for $e^x$ (as well as sine and cosine) by "reversing" his series for the natural logarithm.  We apply what we saw above, first to sine and then to the logarithm.

The goal is to start with
\[ z = \sin^{-1} x = x + \frac{1}{6}x^3 + \frac{3}{40}x^5 + \frac{5}{112}x^7 + \dots \]
And obtain a series
\[ x = \sin z = z + \dots \]
Now, $z = x + $ stuff, so $x = z - $ stuff.  And therefore, although the series \emph{might} have started with $1$, we know it does not, so we start with $x \approx z$.  

There is an error term $p$
\[ x = z + p \]
Substitute into the series that we have
\[ z  = (z + p) +  \frac{1}{6}(z + p)^3 + \frac{3}{40}(z + p)^5 + \frac{5}{112}(z + p)^7 + \dots   \]

Remember that Newton's procedure is to keep only the very first term of the cofactors of $p$ (in the denominator), and the cofactors of $p^0 = 1$ (in the numerator).  

That lone $p$ will be followed by terms like $z^2p/6$, which will be discarded.

The terms with no $p$ start with $z^3/6$.  we have
\[ p + \frac{1}{6}z^3 \approx 0 \]
\[ p \approx - \frac{1}{6}z^3 \]
\[ x = z - \frac{1}{6}z^3 + q \]
where $q$ is another error term.

Let's try one more round.
\[ z  = (z - \frac{1}{6}z^3 + q) +  \frac{1}{6}(z  - \frac{1}{6}z^3 + q)^3 + \frac{3}{40}(z  - \frac{1}{6}z^3 + q)^5 + \dots   \]
We can see that there is a leading $q$ (with no $z$) so we can ignore all the other terms with $q$, and the leading $-z^3/6$ will cancel the first term from
\[  \frac{1}{6}(z  - \frac{1}{6}z^3 + q)^3 \]

Knowing the formula, we are looking for a factor of $z^5/5! = z^5/120$.  It comes partly from this term and partly from the next.  Ignore the leading $1/6$ and group:
\[  [ \ (z  - \frac{1}{6}z^3) + q \ ]^3  = (z  - \frac{1}{6}z^3)^3  + 3(z  - \frac{1}{6}z^3)^2q + \dots \]
\[ = z^3 - 3 z^2 (\frac{1}{6}z^3) + \frac{1}{2}z^7 + \dots \]
Bringing back the $1/6$ we can do our cancelation and we have $-z^5/12$.

The next power is 
\[ \frac{3}{40}(z  - \frac{1}{6}z^3 + q)^5 = \]
Of which only the first term matters:  $3/40 \cdot z^5$.

Remember that these will both change sign when we put them on the other side of the equation.  So then for cofactors of $z^5$ we have:
\[ \frac{1}{12}z^5 -  \frac{3}{40}z^5 = \frac{1}{120}z^5  \]

The first three terms in the series for $\sin z$ are
\[ \sin z = z - \frac{1}{6}z^3 + \frac{1}{120}z^5 + \dots  \]

\subsection*{Newton's method for the exponential}
Using calculus, the difference quotient for the logarithm leads to this relationship.
\[ \frac{d}{dt} \ln x = \frac{1}{x} \]

That derivation depends on knowing that 
\[ e = \lim_{m \rightarrow \infty} (1 + 1/m)^m \]

We set it up with the variable $x$ in the upper limit
\[ \int_1^x \frac{1}{t} \ dt = \ln x - \ln 1 = \ln x \]

The whole subject of logarithmic and exponential functions can start with this as the definition of the logarithm.

So the area under the curve $1/t$ between $1$ and $x$ is equal to the natural log of $x$.

Previously, we manipulated the geometric series to get a series for $1/(1 + x)$ and then integrated to obtain: 
\[ z = \ln 1 + x = x - \frac{x^2}{2} + \frac{x^3}{3} - \frac{x^4}{4} + \dots \]

Now, we want to ``reverse'' this series to get one for the exponential.  Before we get started, notice that we have
\[ z = \ln 1 + x \]
exponentiating
\[ e^z = e^{\ln 1 + x} = 1 + x \]
\[ x = e^z - 1 \]
Our series is for $x$.  So the series we're expecting to get is:
\[ e^z - 1 = z +  \frac{z^2}{2} + \frac{z^3}{6} + \dots \]

Start by saying that $x$ is equal to $z$ plus an error term.
\[ x = z + p \]
and substitute into the series we know for the logarithm
\[ z = (z + p) - \frac{(z + p)^2}{2} + \frac{(z + p)^3}{3} - \frac{(z + p)^4}{4} + \dots \]

Recalling Newton's procedure, since there is a leading $p$ with cofactor $1$ and no other $p^1$ without $z$ in it, we will have $p = \ \text{something}/1$, i.e. the denominator will just be $1$.  

The numerator will come from the lowest power of $z$ that does not multiply $p$, after a change of sign.
\[ 0 \approx p - \frac{z^2}{2} \]
This gives $p \approx z^2/2$ and then
\[ z = z +  \frac{z^2}{2} + q \]
where $q$ is a new error term.
\[ z = (z +  \frac{z^2}{2} + q) - \frac{1}{2} (z +  \frac{z^2}{2} + q)^2 + \frac{1}{3} (z +  \frac{z^2}{2} + q)^3 +  \dots \]

Again, we have a leading factor of $q$ with no $z$, so we can ignore anything that would have $q$ in it that comes after.  That leaves for the second term (without the cofactor):
\[  (z +  \frac{z^2}{2} + q)^2 =  \ [ \ (z + \frac{z^2}{2}) + \dots) \ ]^2 \]
\[ \approx (z + \frac{z^2}{2})^2 \approx z^2 + z^3 \]
bringing back the cofactor of $-1/2$ 
\[ -\frac{1}{2}( z^2 + z^3) \]

$-z^2/2$ cancels what was in the previous correction term, $p$.  What's left is $-z^3/2$, but that's not the whole of it.

From the cubic, we get 
\[ \frac{1}{3} (z +  \frac{z^2}{2} + q)^3 \approx \frac{1}{3}(z + \frac{z^2}{2})^3 \approx \frac{z^3}{3} \]
ignoring higher powers of $z$.

These two terms go on the other side of the equals sign with a change in sign, and then finally obtain
\[ q = \frac{z^3}{2} - \frac{z^3}{3} = \frac{z^3}{6} \]
so the series grows to
\[ x = z +  \frac{z^2}{2} + \frac{z^3}{6} + r \]
where $r$ is yet another error term.

Add $1$ to obtain
\[ e^z = 1 + z +  \frac{z^2}{2} + \frac{z^3}{6} + \dots \]

\subsection*{calculate $e$}

Among other things that gives us the ability to calculate 
\[ e = e^1 = 1 + 1 + \frac{1}{2} + \frac{1}{6} + \frac{1}{24} + \dots \]
With terms up to $1/6!$ I get
\[ = 2.718 \]

We can also verify term by term that
\[ \frac{d}{dz} e^z = e^z \]

\end{document}
