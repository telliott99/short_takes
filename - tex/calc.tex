\documentclass[11pt, oneside]{article} 
\usepackage{geometry}
\geometry{letterpaper} 
\usepackage{graphicx}
	
\usepackage{amssymb}
\usepackage{amsmath}
\usepackage{parskip}
\usepackage{color}
\usepackage{hyperref}

\graphicspath{{/Users/telliott/Github-Math/figures/}}
% \begin{center} \includegraphics [scale=0.4] {gauss3.png} \end{center}

\title{Secrets of calculus}
\date{}

\begin{document}
\maketitle
\Large

%[my-super-duper-separator]
The slope of the line drawn through two points is simply the change in $y$ divided by the change in $x$:
\[ \frac{\Delta y}{\Delta x} = \frac{y_2 - y_1}{x_2 - x_1} \]
It doesn't matter which point we call $(x_1,y_1)$ or $(x_2,y_2)$.

The idea here is to pick a point on a curve and then let the second point get closer and closer to it.
\begin{center} \includegraphics [scale=0.4] {diff_quotient_1.png} \end{center}

Suppose we have a function $f(x)$, let's say $f(x) = x^2$.

Then the point $(x,f(x))$ is on the curve.  The goal is to find the slope of the tangent line to the curve $f(x)$.  That slope is going to depend on $x$, because this is a curve, so the slope changes along the curve.

Move a little bit left or right on the curve by $\Delta x = h$, so we arrive at the point $((x + h), f(x + h))$.  It doesn't matter whether $h$ is positive or negative.  

Calculate the slope of the secant that connects the two points:
\[ \frac{\Delta y}{\Delta x} = \frac{f(x + h) - f(x)}{(x + h) - x} \]
\[ = \frac{f(x + h) - f(x)}{h} \]

The brilliant idea is to say:  what happens if we make $h$ small?  Really really small?  Then, the two points approach each other, and this line gets closer and closer being a tangent line.  

We know the earth is curved, but it is locally flat.  This is the same idea.

We call that slope of the tangent line $f'(x)$ and write:
\[ f'(x) = \lim_{h \rightarrow 0} \frac{f(x + h) - f(x)}{h} \]

This fancy-looking thing is \emph{just another function}.  It contains $h$ of course,  $f(x)$ and $f(x + h)$.  that's it.

At the end of the process, we will think about what happens when $h$ gets really small, $h \rightarrow 0$.  Now, we can't really let $h$ go all the way to zero (why not?), but we can let it get \emph{as small as we please}.  That's the secret.

\subsection*{first example}
The simplest function with variable slope is $x^2$.

Let's write the equation without stating the limit part.  We have
\[ f'(x) = \frac{1}{h} \cdot \ [ \ f(x + h) - f(x) \ ] \]
\[ = \frac{1}{h} \cdot \ [ \ (x + h)^2 - x^2 \ ] \]
\[ = \frac{1}{h} \cdot \ [ \ x^2 + 2xh + h^2 - x^2 \ ] \]
\[ = \frac{1}{h} \cdot \ [ \ 2xh + h^2  \ ] \]
It got simpler!  That's always nice.  Even better, now every term in the brackets has an $h$ in it so we can divide through by $h$, which leaves simply
\[ = 2x + h  \]

Let us think about the limit.  What happens when $h$ gets really really really small?  It becomes insignificant.  The slope will be just $2x$.  

If, instead of $x^2$, we had $f(x) = ax^2$, where $a$ is a constant, then every term above would have an extra $a$.  We can factor $a$ out (it can even come outside the limit) and then multiply the result by $a$ at the end.  It's good practice to do the calculation once yourself and see.

The answer is the slope of the tangent to the curve $y = ax^2$:
\[ f'(x) = 2ax \]

Finally, we need to choose a particular point at which to evaluate this thing.  Suppose $x = c$, then the slope is $2ac$.

\subsection*{integer powers}
$f'(x)$ is called the derivative of $f(x)$.  Let's do $x^3$.
\[ f'(x) = \frac{1}{h} \cdot \ [ \ (x + h)^3 - x^3 \ ] \]
\[ = \frac{1}{h} \cdot \ [ \ x^3 + 3x^2h + 3xh^2 + h^3 - x^3 \ ] \]
\[ = \frac{1}{h} \cdot \ [ \ 3x^2h + 3xh^2 + h^3 \ ] \]
\[ = 3x^2 + 3xh + h^2 \]
And then as $h \rightarrow 0$, we get that 
\[ f'(x) = 3x^2 \]
Again, the answer is surprisingly simple.  There's a pattern here.  Ignoring the possible constant $a$ in front:

$\circ$ \ If the function is $x^2$ the derivative is $2x$.

$\circ$ \ If the function is $x^3$ the derivative is $3x^2$.

It may not surprise you to learn that the derivative of $x^4$ is $4x^3$ and so on.  

The derivative of $f(x) = x$ is just 1, multiplied by $a$ if there was a constant.  We know that from algebra.

\subsection*{square root}

If you know this notation for square root:
\[ f(x) = x^{1/2} \]
you may not be surprised to learn that the rule we had previously still works.  Multiply by the original power out front and then take away 1 from the exponent.
\[ f'(x) = \frac{1}{2} \cdot x^{-1/2} = \frac{1}{2 \sqrt{x}} \]
That's the same thing.

\subsection*{area}
The first big idea of calculus is that with this function, called the \emph{difference quotient}, we can get the slope of the tangent, which is the rate at which the function is increasing at the point $x = c$.
\[ f'(x) = \lim_{h \rightarrow 0} \frac{f(x + h) - f(x)}{h} \]

The second big idea is that the function $f(x)$ and the area under the curve have the same relationship, but in reverse.  In other words, the area grows like $f(x)$.  The derivative of the area is the function $f(x)$.

\begin{center} \includegraphics [scale=0.4] {FTC_geometric2.png} \end{center}
I'm going to use the notation $A(x)$ for the area.  You can see that area is going to depend on where we start and stop, where we draw the verticals.

If $A(x)$ is the (as yet unknown) function that describes the area under the curve $f(x)$, then the derivative of $A(x)$ \emph{is} $f(x)$:
\[ A'(x) = f(x) \]
It's that simple.

If we know $f(x)$ it can be hard to find $A(x)$, sometimes it is impossible.

But what we will do is have a little black book where we write down a list of functions and their derivatives.  

Then when we want to know the area under some curve, we look to see if the function is one of the derivatives we know.

\subsection*{a first area problem}
I'll work an example of an area calculation.  We're going to find the area of the square between $x = 0$ and $x = 1$, and $y = 0$ and $y = 1$.  Of course, we know the answer, the area is just $1$.  But we are going to break it up into two parts.

For the first one, we find the area under the curve $y = x^2$ between $0$ and $1$.
\begin{center} \includegraphics [scale=0.5] {x_squared_plot.png} \end{center}

We write
\[ \int_0^1 x^2 \ dx \]

That's a new symbol, the integral sign $\int$, and also it has bounds on it, the $0$ and the $1$.  Further, there is that new notation $dx$.  I am not going to explain the $\int$ symbol nor $dx$ now.

However, the bounds should not be surprising, since the area accumulated will depend on where we start and stop.  

We will just say that
\[ \int f(x) \ dx = A(x) \]
if and only if $f(x)$ is the derivative of $A(x)$.

Integrating the function $f(x)$ means simply to find the function $A(x)$ such that the derivative of $A(x)$ is $f(x)$.

So we have the function $x^2$ and we want its integral.  We know the function $x^3$ and know that its derivative is $3x^2$.  Divide by 3:
\[ \ [ \ \frac{x^3}{3} \ ] ' = x^2 \]
We're writing brackets with a prime after, to indicate that we're talking the derivative $f'(x)$.

Because $\int f(x) = A(x)$ if and only if $A'(x) = f(x)$:  I assert that
\[ \int x^2 \ dx = \frac{x^3}{3} \]

which we must evaluate at the two bounds, subtracting the starting point from the ending point.  We write
\[ \int_0^1 x^2 \ dx = \frac{x^3}{3} \bigg|_0^1  \]
\[ = \frac{1}{3} - 0 = \frac{1}{3} \]
That's how you do an area problem.

\subsection*{the rest of the box}
For the last step, we need one more derivative, which is $x$ to the $3/2$ power.  This is simply
\[ f(x) = x \cdot \sqrt{x} \]
Finding $f'(x)$ for this will yield to exactly the same procedure as we used above for the simple square root.  By the power rule:
\[ [ \ x^{3/2} \ ] ' = \frac{3}{2} x^{1/2} \]
so
\[  [ \ \frac{2}{3} \ x^{3/2} \ ] ' = x^{1/2} \]
and then
\[ \int x^{1/2} \ dx = \frac{2}{3} \ x^{3/2}  \]

Here's why we did that.  If you take the function $x^2$, the inverse function of that is $\sqrt{x}$ or $x^{1/2}$.  The area of the other region in the box, the area above $x^2$ in this plot
\begin{center} \includegraphics [scale=0.5] {x_squared_plot.png} \end{center}

is exactly the area beneath the curve $\sqrt{x}$ (between 0 and 1) in this one.
\begin{center} \includegraphics [scale=0.3] {square_root_plot.png} \end{center}

We need, in other words, to calculate
\[  \int_0^1 x^{1/2} \ dx \]
which is why we just found the derivative of $x^{3/2}$.  Evaluate between the two bounds to obtain 
\[  \int_0^1 x^{1/2} \ dx = \frac{2}{3} \ x^{3/2}  \bigg|_0^1 = \frac{2}{3} \]

Add the results for the two regions
\[ \frac{1}{3} + \frac{2}{3} = 1 \]
as one would hope.

\subsection*{other calculations}
This is just put here for documentation.  Feel free to skip the rest of this write-up.

We will calculate the derivative of the square root function
\[ f'(x) = \frac{1}{h} \cdot \ [ \ \sqrt{x + h} - \sqrt{x} \ ] \]
I don't know how to do that subtraction.  But there's a trick.  Multiply by what is called the conjugate
\[ f'(x) = \frac{1}{h} \cdot \ [ \ \sqrt{x + h} - \sqrt{x} \ ] \cdot  \ \frac{\sqrt{x + h} + \sqrt{x}}{\sqrt{x + h} + \sqrt{x}}\]

That looks pretty wild.  But what we have in the numerator is basically $(p - q)(p + q) = p^2 - q^2$, which makes everything simple.
\[ f'(x) = \frac{1}{h} \cdot \ \frac{x + h - x}{\sqrt{x + h} + \sqrt{x}} \] 
\[ =  \frac{1}{h} \cdot \ \frac{h}{\sqrt{x + h} + \sqrt{x}} \] 
\[ =  \frac{1}{\sqrt{x + h} + \sqrt{x}} \] 
And then as $h \rightarrow 0$, we get that 
\[ f'(x) = \frac{1}{2 \sqrt{x}} \]
What a relief.

I decided to do the $3/2$ power as well:

\[ f'(x) = \frac{1}{h} \cdot \ [ \  (x + h) \sqrt{x + h} - x \sqrt{x} ) \ ] \] 
\[ = \frac{1}{h} \cdot \ [ \  (x + h) \sqrt{x + h} - x \sqrt{x}) \ ] \cdot \frac{ (x + h) \sqrt{x + h} + x \sqrt{x}}{ (x + h) \sqrt{x + h} + x \sqrt{x}}\] 
\[ = \frac{1}{h} \cdot \ \frac{(x+h)^3 - x^3}{(x + h) \sqrt{x + h} + x \sqrt{x}} \]
\[ = \frac{1}{h} \cdot \ \frac{3x^2h + 3xh^2 +h^3}{(x + h) \sqrt{x + h} + x \sqrt{x}} \]
\[ = \frac{3x^2 + 3xh + h^2}{(x + h) \sqrt{x + h} + x \sqrt{x}} \]
What happens when $h \rightarrow 0$?  Then
\[ f'(x) = \frac{3x^2}{2 x \sqrt{x }} \]
\[ = \frac{3}{2} \cdot \sqrt{x} \]


\end{document}
