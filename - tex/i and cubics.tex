\documentclass[11pt, oneside]{article} 
\usepackage{geometry}
\geometry{letterpaper} 
\usepackage{graphicx}
	
\usepackage{amssymb}
\usepackage{amsmath}
\usepackage{parskip}
\usepackage{color}
\usepackage{hyperref}

\graphicspath{{/Users/telliott/Dropbox/Github-math/figures/}}
% \begin{center} \includegraphics [scale=0.4] {gauss3.png} \end{center}

\title{Cubic Q}
\date{}

\begin{document}
\maketitle
\Large

The goal is to solve this famous cubic
\[ x^3 = 15x + 4 \]

Solutions are given as
\[ x = \sqrt[3]{2 + \sqrt{-121}} + \sqrt[3]{2 - \sqrt{-121}} \]

Which is, the story goes, the kind of thing that first made people take complex numbers seriously.  They can be intermediate results in computations that give real answers which are correct.

We can rewrite $\sqrt{-121}$ as $11 \sqrt{-1} = 11i$ so
\[ x = (2 + 11i)^{1/3} + (2 - 11i)^{1/3} \]

We also are given that $\sqrt[3]{2 + 11i}$ is $2 + i$, while the other term is equal to $2 - i$ and added together that's just $4$.

It is easy to confirm that $4$ does solve the cubic since $4^3 = 64$ and also that
\[ (2 + i)^2 = 4 + 4i - 1 = 3 + 4i \]
and
\[ (2 + i)^3 = (3 + 4i)(2 + i) \]
\[ = 6 - 4 + 11i = 2 + 11i \]

The question is, how would we take the cube root of $2 + 11i$ if we didn't know the answer?  

Normally, we use $r, \theta$ notation for multiplication, and powers too.  The complex number $2 + i$ has magnitude $r = \sqrt{5}$ and $\theta = \tan^{-1} 1/2$.  

This is mildly tricky, because it's not the sine but the tangent, so $\theta$ is \emph{not} $\pi/6$.

To take the cube root, $\tan^{-1} 11/2 = 1.391$ radians $ = 79.796^{\circ}$, and this is the phase (or argument) of $e$, so multiply by $1/3$ of that and then taking the tangent of the result does indeed give $0.5$.  

The magnitude of $2 + 11i$ is $\sqrt{125}$, so the magnitude we seek is the cube root of $\sqrt{125}$.  
\[ (n^{1/2})^{1/3} = n^{1/6} = (n^{1/3})^{1/2} \]

The roots can be taken in either order.  Hence we obtain the square root of the cube root of $125$, or $\sqrt{5}$.

A right triangle with sides of $2$ and $1$ has side length $\sqrt{5}$, so that's a match.



\end{document}