\documentclass[11pt, oneside]{article} 
\usepackage{geometry}
\geometry{letterpaper} 
\usepackage{graphicx}
	
\usepackage{amssymb}
\usepackage{amsmath}
\usepackage{parskip}
\usepackage{color}
\usepackage{hyperref}

\graphicspath{{/Users/telliott/Dropbox/Github-Math/figures/}}
% \begin{center} \includegraphics [scale=0.4] {gauss3.png} \end{center}

\title{Sum of squares}
\date{}

\begin{document}
\maketitle
\Large

%[my-super-duper-separator]

In this write-up we're going to look at a method to derive the formula for the sum of the squares of integers from $1 \dots n$.

The key idea is to first write
\[ (k + 1)^3 = k^3 + 3k^2 + 3k + 1 \]
(which is true for any particular $k$), and then sum all $n$ equations for $1 \dots n$.
\[ \sum_{k=1}^n (k+1)^3 = \sum_{k=1}^n k^3 + \sum_{k=1}^n 3k^2 + \sum_{k=1}^n 3k + \sum_{k=1}^n 1 \]

This looks a little frightening.  So the next thing is to suppress the indices in the sums.  They will always be $1 \dots n$.  Move one term to the left-hand side.

\[ \sum (k+1)^3 - \sum k^3 = 3 \sum k^2 + 3 \sum k + n \]

As before, the difference of sums on the left-hand side simplifies dramatically.  On the right, substitute the previous result.
\[ (n+1)^3 - 1 = 3 \sum k^2 + 3 \cdot \frac{n(n+1)}{2} + n \]
\[ n^3 + 3n^2 + 2n = 3 \sum k^2 + 3 \cdot \frac{n(n+1)}{2}  \]

The best thing to do here is to factor the left-hand side.  We see there is an $n$, there will also be a factor of ($n + 1)$:
\[ n^3 + 3n^2 + 2n = n(n^2 + 3n + 2) \]
\[ = n (n + 1)(n + 2) \]

So now, in one step, we will move the last term on the right-hand side to the left, and also factor out the $n(n+1)$ giving
\[ n(n+1)(n + 2 - \frac{3}{2}) = 3 \sum k^2 \]
\[ n (n+1)(n + \frac{1}{2}) = 3 \sum k^2 \]
\[ \sum k^2 = \frac{n (n+1)}{2} \cdot \frac{(2n + 1)}{3} \]

This can be re-written in various ways, for example
\[ S_{n^2} = \frac{1}{3} \cdot n (n+1)(n + \frac{1}{2}) \]

\subsection*{proof by induction}

We can check it by induction.  The base case is easy
\[ \frac{1(2)(3)}{6} = 1 \]  

Now for the induction step:
\[ \frac{n(n+1)(2n+1)}{6} + (n+1)^2 \]
\[ = \frac{n+1}{6}  \ [ \ (n)(2n+1) + 6(n+1) \ ] \]
Look at what's in the brackets
\[ (n)(2n+1) + 6(n+1) \]
\[ = 2n^2 + 7n + 6 \]
\[ = (n + 2)(2n + 3) \]
\[ = ((n + 1) + 1)(2(n + 1) + 1) \]
So altogether we have
\[ = \frac{(n+1)((n + 1) + 1)(2(n + 1) + 1)}{6} \]
which indeed, is the formula we had above, substituting $n+1$ for $n$.

$\square$

Here are a proof without words:
\begin{center} \includegraphics [scale=0.5] {sum_n2_2.png}\end{center}
The pieces have a base of $n(n+1)$ and there are three of them rising to a height of $n + \frac{1}{2}$.

\subsection*{Strang's proof}
Here is another approach, from Strang's \emph{Calculus}.  He says "the best place to start is a good guess".  So again, our goal is to find a formula for:

\[ S = \sum_{k=1}^{n} \ k^2 \]

Perhaps we visualize a pile of cannonballs
\begin{center} \includegraphics [scale=0.5] {cannonballs.png} \end{center}

Each layer contains a square number of cannonballs ($1$, then $4$, then $9$, etc.).  The shape is a pyramid with dimensions $n \times n \times n$.  We know the formula for the volume of a pyramid, and guess

\[ S_n = \frac{1}{3} n^3 \]

To test it, check whether this difference is $n^2$ (as it should be):

\[ S_{n} - S_{n-1} = \frac{1}{3} n^3 - \frac{1}{3} (n-1)^3 \]

Since
\[ (n-1)^3 = n^3 - 3 n^2 + 3 n - 1\]
then
\[ S_{n} - S_{n-1} = \frac{1}{3} (n^3 - n^3 + 3 n^2 - 3 n + 1) \]

We see that our guess is off by the residual terms

\[ \frac{1}{3} (3 n^2 - 3 n + 1) \]
\[ = n^2 - n + \frac{1}{3} \] 

Strang says:  the guess needs \emph{correction terms}.  
To cancel $1/3$ in the difference, subtract $n/3$ from the sum.  And to add back $n$ in the difference, add back $1 + 2 + \dots + n(n+1)/2$ to the sum.  Our new guess is

\[ S_n =  \frac{1}{3} n^3 + \frac{n(n+1)}{2} - \frac{n}{3} \]
\[ = \frac{n}{6} (2n^2 + 3(n+1) - 2) \]
\[ =  \frac{n}{6} (2n + 1)(n + 1) \]
\[ = \frac{n(n+1)(2n+1)}{6} \]

which may be easier to remember as

\[ S_n = \frac{n(n+1)}{2} \cdot \frac{2n + 1}{3} \]

\subsection*{undetermined coefficients}

We guess that the formula for sum of squares will have terms of degree 3:  powers of $n^3$.

\[ S_{n} = an^3 + bn^2 + cn + d \]
Since $S_0 = 0$ we conclude that $d = 0$.  The sum for $n-1$ would be
\[ S_{n-1} = a(n-1)^3 + b(n-1)^2 + c(n-1)  \]

If we add $n^2$ to this, we should have the previous expression:
\[ a(n-1)^3 + b(n-1)^2 + c(n-1) + n^2 = S_n =  an^3 + bn^2 + cn  \]

When we expand the parentheses, each term on the right is canceled, leaving 
\[ a(-3n^2 + 3n - 1) + b(- 2n + 1) - c + n^2 =  0  \]

And, as before, the coefficients of different powers of $n$ must all separately balance.
\[ -3a + 1 = 0 \]
\[ a = \frac{1}{3} \]

Then
\[ 3a - 2b = 0 \]
\[ b = \frac{3}{2} a = \frac{1}{2} \]

and $-a + b - c = 0$, which gives $c = b - a = 1/6$ so
\[ S_n = \frac{n^3}{3} + \frac{n^2}{2} + \frac{n}{6} \]

We have a factor of $n/6$ and the rest is
\[ 2n^2 + 3n + 1 = (2n + 1)(n + 1) \]
so
\[ S_n = \frac{n(n+1)(2n+1)}{6} \]

\end{document}