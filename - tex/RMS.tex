\documentclass[11pt, oneside]{article} 
\usepackage{geometry}
\geometry{letterpaper} 
\usepackage{graphicx}
	
\usepackage{amssymb}
\usepackage{amsmath}
\usepackage{parskip}
\usepackage{color}
\usepackage{hyperref}

\graphicspath{{/Users/telliott/Dropbox/Github-Math/figures/}}
% \begin{center} \includegraphics [scale=0.4] {gauss3.png} \end{center}

\title{RMS voltage}
\date{}

\begin{document}
\maketitle
\Large

%[my-super-duper-separator]

I got to thinking about how voltage is typically reported.  According to wikipedia

\url{https://en.wikipedia.org/wiki/Mains_electricity}

in the US and Canada the nominal voltage for a standard circuit should be 120V at source ($\pm \ 5$ \%).  They go on to say that  

\begin{quote}{Historically 110 V, 115 V and 117 V have been used at different times and places in North America.}\end{quote}

In a circuit run with alternating current (AC), the voltage varies as a sine wave.  So it's obvious that we need something a bit more sophisticated to talk about \emph{the} voltage.

One way would be to talk about the average voltage.  In calculus, we define the average of a function over an interval as
\[ \frac{\int_{x_1}^{x_2} f(x) \ dx}{x_2 - x_1} \]

This has a remarkably simple value for the sine and cosine functions.  Namely, since $\int \sin x \ dx = - \cos x$, the integral over the interval $[0 .. \pi/2]$ is
\[ I = - \cos x \bigg |_0^{\pi/2} = (- 0) - (- 1) = 1 \]

Because of symmetry we can pick any interval that is a multiple of $\pi/2$.

The average value is the maximum value ($1$) divided by $\pi/2$, that is to say, multiplied by $2/\pi$.  When thinking about this it is helpful to remember that the average is less than the maximum.

\subsection*{correct calculation}

The problem is that this is not what is done.  For historical reasons, we calculate the RMS, the \emph{root mean square}.  

This means exactly what it says, namely, square the values (for a discrete problem), calculate the mean, and then take the square root.

For a continuous function like the sine, we use integration to find the average.  The first part of the calculation is to integrate
\[ I = \int \sin^2 x \ dx \]
then divide the result $I$ by the interval to find the mean, and then take the square root.

This integral is probably the first non-trivial integral encountered in learning calculus, and can be solved in several ways.  We will do it by guessing.  With a prime to indicate the derivative, use the formula
\[ [uv ]' = u'v + uv' \]

Here what we want is to find the derivative of this product:
\[ [ \sin x \cos x ]' = - \cos^2 x + \sin^2 x \]
\[ = -1 + 2 \sin^2 x \]

Integrate both sides and all of a sudden, we have the integral we want
\[ \sin x \cos x = -x + 2 \int \sin^2 x \ dx \]
so
\[ I = \int \sin^2 x \ dx = \frac{1}{2} ( x + \sin x \cos x ) \]

Over the interval $[0 .. \pi/2]$ we have simply
\[ I = \frac{1}{2} \ [ \ (\pi/2 + 0) - (0 + 0) \ ] = \frac{\pi}{4} \]

Dividing by the length of the interval gives $1/2$ and then taking the square root we get $1/\sqrt{2}$

In other words the maximum and nominal voltage are simply related by this factor, $\sqrt{2}$.  The maximum voltage for a 120V circuit is about 170V.

Of course the actual voltage depends on a number of practical factors, which we ignore.


\end{document}