\documentclass[11pt, oneside]{article} 
\usepackage{geometry}
\geometry{letterpaper} 
\usepackage{graphicx}
	
\usepackage{amssymb}
\usepackage{amsmath}
\usepackage{parskip}
\usepackage{color}
\usepackage{hyperref}

\graphicspath{{/Users/telliott/Dropbox/Github-Math/figures/}}
% \begin{center} \includegraphics [scale=0.4] {gauss3.png} \end{center}

\title{Making change}
\date{}

\begin{document}
\maketitle
\Large

%[my-super-duper-separator]

You're working the cash register at the soup place and someone's grandad actually wants to use cash.  That's weird.

Their food costs \$$3.84$ and they hand you ten dollars.  You freeze ...

What to do?  Put away your calculator.

Turn this into a reverse addition problem.  What number, when added to $84$, gives $100$?
\[ 84 +  \text{??} = 100 \]

The rule is that in the ones place (the last place on the right), the digits must add to give $0$.  We get $6$, since $4 + 6 = 10$.
\[ 84 + \text{?}6 = 100 \]

In the next place, the digits add to give $9$ (one less than $10$).  Since $8 + 1 = 9$, we obtain
\[ 84 + 16 = 100 \]

Thus, we can solve the change problem for \$$1$ by saying, the digits in the tens place should add up to $9$, and the digits in the ones place should add to $10$.

\begin{verbatim}
25 + 75
34 + 66
51 + 49
72 + 28
\end{verbatim}

You get the idea.

Having solved the coins, count them out.  Now forget that part of the answer.

The paper money is as follows.  Do the same as before, but first, subtract \$$1$ from what the customer gives you.

Their food costs \$$3.84$ and they hand you \$$10$.  We already solved the coins as $0.16$ and now the dollars are $3 + 6 = 9$.  The change is $6.16$.

Suppose he gives you \$$20$, or \$$5$?

We have $3$ plus something equals $19$, one less than $20$?  The answer is $16$.

Or $3$ plus something equals $4$, one less than $5$?  The answer is $1$.

And for the person who gives you four pennies plus \$$20$, ask him:  what the heck are those things?

Or you can do the following:  subtract $4$ cents from what he owes.  So now make change for \$$3.80$ from \$$20.00$.  That's $0.20$ plus $16$ so \$$16.20$.

$\bullet$  total to $10$ in the ones position

$\bullet$  total to $9$ in the tens position

$\bullet$ add to $1$ less than what they gave for the dollars

$\bullet$ if they give you coins, credit the coins against the amount, then make change for that


\end{document}
