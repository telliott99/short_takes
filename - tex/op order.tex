\documentclass[11pt, oneside]{article} 
\usepackage{geometry}
\geometry{letterpaper} 
\usepackage{graphicx}
	
\usepackage{amssymb}
\usepackage{amsmath}
\usepackage{parskip}
\usepackage{color}
\usepackage{hyperref}

\graphicspath{{/Users/telliott/Dropbox/Github-Math/figures/}}
% \begin{center} \includegraphics [scale=0.4] {gauss3.png} \end{center}

\title{Order of operations}
\date{}

\begin{document}
\maketitle
\Large

%[my-super-duper-separator]

Order of operations is a topic in pre-algebra.  The acronym to remember is PEMDAS, which stands for

$\circ$ \ parentheses

$\circ$ \ exponents

$\circ$ \ multiplication and division

$\circ$ \ addition and subtraction

I'm not crazy about this subject.  For one thing, it usually employs the symbols $\times$ and $\div$, which we are going to do away with in algebra.  A few examples:

\subsection*{1}

\[ 42 \div 7 \times 3 \]

Since multiplication and division are at the same "level" in the hierarchy, and operations at the same level are carried out left-to-right: 
\[ 42 \div 7 \times 3 = 6 \times 3 = 18 \]
In algebra we would write
\[ \frac{42}{7} \cdot 3 \]
which is unambiguous.  And if we really wanted something different, we can use brackets
\[ 42 \div (7 \times 3) = \frac{42}{7 \cdot 3} \]

\subsection*{2}

Here's one with multiplication and addition:
\[ 6 + 7 \times 3 = 6 + 21 = 27 \]
Multiplication is at a higher level than addition, so we hold off on the addition and do the multiplication first.  To force the other evaluation just write brackets:
\[ (6 + 7) \cdot 3  = 39 \]

Remember the distributive law!
\[ 3 \cdot (6 + 7) = 3 \cdot 13 \]
And we can split up the terms any way that will make it easier.  For example
\[ 3 \cdot (6 + 7) = 3 \cdot (10 + 3) = 3 \cdot 10 + 3 \cdot 3 = 39 \]

\subsection*{brackets first}

\[ 27 \div (8 - 5)^2 = 27 \div 3^2 = 27 \div 9 = 3 \]
Here, the brackets direct us to do the subtraction first.  then E comes before D, so we do the square, and then the division last.  In algebra we would write
\[ \frac{27}{(8-5)^3} \]

Brackets top and bottom:
\[ (\frac{2 + 1}{3 - 1})^2 = \frac{(2 + 1)^2}{(3 - 1)^2} = \frac{3^2}{2^2} = \frac{9}{4} \]

\[ (18 \div 6 \times 5) - 14 \div 7 = (3 \cdot 5) - 2 = 13 \]


\end{document}
