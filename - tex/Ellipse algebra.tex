\documentclass[11pt, oneside]{article} 
\usepackage{geometry}
\geometry{letterpaper} 
\usepackage{graphicx}
	
\usepackage{amssymb}
\usepackage{amsmath}
\usepackage{parskip}
\usepackage{color}
\usepackage{hyperref}

\graphicspath{{figures}{/Users/telliott/Github-Math/figures/}}

% \begin{center} \includegraphics [scale=0.4] {gauss3.png} \end{center}

\title{Algebra of the ellipse}
\date{}

\begin{document}
\maketitle
\Large
Below is an ellipse which crosses the $x$-axis at $(\pm a,0)$ and the $y$-axis at $(0,\pm b)$.  By definition, the distances from every point on the ellipse to the two foci,  $(\pm c,0)$, combined, is a constant, $L$.
\begin{center} \includegraphics [scale=0.3] {ellipse.png} \end{center}

Note that if $(x,y) = (a,0)$ then the two distances are $a-c$ and $a+c$ which add up to $L = 2a$.  

If the point is $(x,y) = (0,b)$ then by the Pythagorean theorem we have:
\[ 2\sqrt{b^2 + c^2} = L = 2a \]
\[ a^2 = b^2 + c^2 \]
\[ b^2 = c^2 - a^2 \]
We'll come back to this.

This example has $a = 2.5$ and $b = 2$.  

$2.5^2 - 2^2 = 6.25 - 4 = 2.25 = 1.5^2$.  So even though the ratio is pretty moderate $a/b = 5/4$, the foci are still at $3/5 \cdot a$.

For any general point we can write another equation using Pythagoras twice:
\[ L = \sqrt{(x + c)^2 + y^2} + \sqrt{(x - c)^2 + y^2}  = 2a \]
Rearrange
\[ \sqrt{(x + c)^2 + y^2}  = 2a -  \sqrt{(x - c)^2 + y^2}  \]
Square
\[ (x + c)^2 + y^2 = 4a^2 - 4a  \sqrt{(x - c)^2 + y^2} + (x - c)^2 + y^2 \]
Cancel $y^2$, $x^2$ and $c^2$
\[ 2xc = 4a^2 - 4a  \sqrt{(x - c)^2 + y^2} - 2xc \]
Rearrange and cancel $4$
\[ a\sqrt{(x - c)^2 + y^2} = a^2 -xc \]
Square again
\[ a^2(x-c)^2 + a^2y^2 = a^4 - 2a^2xc + x^2c^2 \]
Expand
\[ a^2x^2 - 2a^2xc + a^2c^2 + a^2y^2 = a^4 - 2a^2xc + x^2c^2 \]
Cancel $ - 2a^2xc$
\[ a^2x^2 + a^2c^2 + a^2y^2 = a^4 + x^2c^2 \]
Rearrange and factor
\[ (a^2 - c^2)x^2 + a^2y^2 = a^2(a^2 - c^2) \] 
Substitute (from way back):
\[ b^2 x^2 + a^2y^2 = a^2 b^2 \]
Finally
\[ \frac{x^2}{a^2} + \frac{y^2}{b^2} = 1  \]

\end{document}
