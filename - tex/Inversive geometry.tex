\documentclass[14pt, oneside]{article} 
\usepackage{geometry}
\geometry{letterpaper} 
\usepackage{graphicx}
	
\usepackage{amssymb}
\usepackage{amsmath}
\usepackage{parskip}
\usepackage{color}
\usepackage{hyperref}

\graphicspath{{/Users/telliott/Github-math/figures/}}
% \begin{center} \includegraphics [scale=0.4] {gauss3.png} \end{center}

\title{Inversive Geometry}
\date{}

\begin{document}
\maketitle
\Large

%[my-super-duper-separator]

To carry out inversion, one starts by choosing a reference circle in the plane.  We will refer to $\omega$, the \emph{inversion circle}, with its \emph{inversion center} on $O$ and radius $k$.

Let $P$ be an arbitrary point, but not $O$.  The inverse of $P$ is called $P'$, computed according to the following rule:
\begin{center} \includegraphics [scale=0.40] {inversion1.png} \end{center}
$P'$ lies on $OP$ or its extension, such that
\[ OP \cdot OP' = k^2 \ \ \ \ \ \ \ \  \frac{OP}{k} = \frac{k}{OP'} \]

We can write $ I_{\omega} (P) = P'$.  In what follows we will show $\omega$ as a dotted line.  In some cases we only show a part of the circle.

A collection of transformed points is referred to as the image of the input.  In some cases the image of a point may be the same as the original.

You can see in the example above that $OP$ is about one-half of $k$, so then $OP'$ is twice $k$.  The image of points close to $O$ may be very far away.

$O$ is a special point in more than one way.  As $P$ approaches $O$, the inverse $P'$ moves farther and farther away.

To be complete, the theory invents a "point at infinity" which is the inverse of $O$.  We won't worry about that.  We'll just think of the set of possible inputs to $I$ as the "punctured plane", with $O$ missing.

When we say something like ``a circle through $O$'', we are not worrying about the point $O$ itself.
\begin{center} \includegraphics [scale=0.30] {inversion1a.png} \end{center}

Above we said ``$OP$ or its extension''.  The technical term here is ray, a half-line.  But we will usually just draw a line segment, with definite endpoints, and also just call it a line for brevity.

We call points that are invariant under inversion, fixed.  Let $R$ be a point on $\omega$, so the length of $OR$ is $k$.  Then we have that
\[ I_{\omega} (R) = R' = R \]

Points lying on $\omega$ itself invert into themselves, because then $OP = k = OP'$.

Much of our work with inversions involves similar triangles.  Coxeter introduces it this way:
\begin{center} \includegraphics [scale=0.30] {Coxeter_5_3A.png} \end{center}
Draw the chord $TU \perp OP$.  Then $P'$ is the point where the tangents from $T$ and $U$ meet.  From similar triangles $\triangle OTP' \sim \triangle OPT$, so we have:
\[ \frac{OP'}{k} = \frac{k}{OP} \]

An analogous transformation in three dimensions uses a sphere, in space.

\subsection*{inversion is an involution}

If $P'$ is the inverse of $P$, then the inverse of $P'$ is $P$.

\emph{Proof}.  Let $P''$ be the inverse of $P'$, with $OP' \cdot OP'' = k^2$.  But $k^2 = OP \cdot OP'$.  It follows that $OP'' = OP$. $\square$

If $P$ lies inside $\omega$, then $P'$ lies outside, and vice-versa.  There is a famous essay which analogizes how one would hunt a lion in the desert by various mathematical or physical techniques.  Here is the one using inversion:

\begin{quote}
We place a \emph{spherical} cage in the desert, enter it and lock it. We perform an inversion with respect to the cage. The lion is then inside the cage, and we are outside.  \emph{H. Petard}
\end{quote}

\subsection*{inversion of a line through $O$}

We now consider the results of inverting either a line or a circle.  There are two cases for each, depending on whether the line or circle goes through, $O$, the center of $\omega$.  They are all related, as we will see.

\begin{center} \includegraphics [scale=0.25] {inversion2.png} \end{center}
Let $P$ and $Q$ be any two points on a line through $O$.  The inverse points $P'$ and $Q'$ lie on the same line.

\textbf{The inverse of a line through $O$ is the same line}.

\subsection*{inversion of a line not through $O$}

\begin{center} \includegraphics [scale=0.3] {inversion3a.png} \end{center}

Let $L$ be a line, not passing through $O$, and let $P$ and $A$ be two points on the line.

\textbf{The inverse of a line not through $O$ is a circle through $O$.}

This is the first result that is a little surprising.

\emph{Proof}.

Let $L$ be external to $\omega$.  Pick $A$ such that $OA'A \perp PA$.  We have that 
\[ OP \cdot OP' = k^2 = OA \cdot OA' \]
\[ \frac{OP}{OA} = \frac{OP'}{OA'} \]

It follows that $\triangle PAO \sim \triangle A'P'O$ and in particular, $\angle OP'A'$ is a right angle because $\angle OAP$ is a right angle.

But this is true for \emph{every} point on $L$ other than $A$.

By the inverse of Thales' circle theorem, it follows that $A'$ and $P'$ lie on a circle.  This circle passes through $O$.

$\square$

The diameter of this circle is perpendicular to the line $L$, so $L$ is parallel to the tangent to the circle at $O$.

\begin{center} \includegraphics [scale=0.3] {inversion4.png} \end{center}

The same proof still works for the case where $L$ passes through $\omega$, or even when it is tangent (below).  This is because, in both cases, $\triangle OAP \sim \triangle OP'A'$.

\begin{center} \includegraphics [scale=0.3] {inversion4b.png} \end{center}

\subsection*{note on similar triangles}

For any points $A$ and $P$, we can form similar triangles by using the fundamental equations
\[ OP \cdot OP' = k^2 = OA \cdot OA' \]
\[ \frac{OP}{OA} = \frac{OP'}{OA'} \]

\begin{center} \includegraphics [scale=0.38] {inversion3b.png} \end{center}

This works even when one of $A$ or $P$ lies inside, and one outside $\omega$.

\subsection*{inversion of a circle through $O$}

Let $K$ be a circle, passing through $O$.  

\begin{center} \includegraphics [scale=0.3] {inversion3a.png} \end{center}

\textbf{The inverse of a circle through $O$ is a line not through $O$.}

This follows directly from the previous theorem, since the inverse of the inverse of any point is the same point.

As we saw from the previous result, it does not matter whether the circle is entirely internal to $\omega$, is internally tangent, or extends outside.

\subsection*{Ptolemy's Theorem}

With only what we have established so far, we can prove an important theorem.

Let $A$, $B$ and $C$ lie on a circle through $O$.  Then, as we showed above, the inverse of this circle is a line not through $O$ and here, entirely external to the inversion circle.

Ptolemy says that 
\[ AB \cdot OC + BC \cdot OA = AC \cdot OB \]

\emph{Proof}.

\begin{center} \includegraphics [scale=0.35] {inversion9.png} \end{center}

$A'B' + B'C' = A'C'$.  We will leverage this identity to find the result.

We have three pairs of similar triangles.  To help stay organized, we write out the ratio boxes.
\begin{center} \includegraphics [scale=0.18] {ratios11.png} \end{center}

The symmetry is pretty clear.  We have that
\[ \frac{A'B'}{AB} = \frac{OA'}{OB} \Rightarrow A'B' =  \frac{AB \cdot OA'}{OB} \]

Also
\[ \frac{B'C'}{BC} = \frac{OC'}{OB} \Rightarrow B'C' = \frac{BC \cdot OC'}{OB} \]

Then
\[ \frac{A'C'}{OC'} = \frac{AC}{OA} \Rightarrow A'C' = \frac{AC \cdot OC'}{OA} \]

Form the sum and clear the denominators:
\[ AB \cdot OA' \cdot OA + BC \cdot OC' \cdot OA = AC \cdot OC' \cdot OB \]

Divide by $OC'$
\[ AB \cdot \frac{OA'}{OC'} \cdot OA + BC \cdot OA = AC \cdot OB \]

Can you find the appropriate entry for the last step?  I obtain
\[ AB \cdot OC + BC \cdot OA = AC \cdot OB \]

$\square$

This is Ptolemy's theorem.

We pause to reflect on one of the most useful and important theorems in all of geometry.

We now consider circles not through $O$.  This is the point where the theory gets a little more sophisticated.

\subsection*{inversion of a circle not through $O$}

\textbf{The inverse of a circle not through $O$ is another circle not through $O$.}

We can again draw different diagrams depending on where the circles lie with respect to $\omega$.  We need to make sure that the proof or proofs work for all of them.

Consider first the case where $K$ is entirely external to $\omega$.

\emph{Proof}.

Let $K$ be a circle, not passing through $O$, and not even intersecting with $\omega$.  Draw the diameter $AB$ of $K$, such that the extension passes through $O$.

\begin{center} \includegraphics [scale=0.35] {inversion5.png} \end{center}

Let $C$ be an otherwise arbitrary point on $K$, not on $AB$.  Find the inverses of $A$, $B$ and $C$.  Draw $\triangle A'B'C'$.

As before
\[ \frac{OA'}{OC'} = \frac{OA}{OC} \]
so $\triangle OB'C' \sim \triangle OCB$ and $\triangle OA'C' \sim \triangle OCA$.

From similar triangles, we have that the angles marked $\alpha$ and $\beta$ are equal.

By Thales' circle theorem, $\angle ACB$ is a right angle.  So from $\triangle BOC$:
\[ \angle BOC + \alpha + \beta + \angle ACB = 180 \]

But from $\triangle B'OC'$ (with $\angle B'OC' = \angle BOC$):
\[ \angle B'OC'  + \alpha + \beta + \angle AC'B' = 180 \]

It follows that $\angle A'C'B' = \angle ACB$, so $\angle A'C'B'$  is also a right angle, and thus it lies on the circle whose diameter is $A'B'$.  This is true for every point $C$ on circle $K$, other than $A$ and $B$.  

$\square$

Second case: let the circle $K$ intersect $\omega$, with the inversion center $O$ inside $K$.

Draw the diameter $AB$ of circle $K$ which goes through $O$.  Let $C$ be any other point on circle $K$.

\begin{center} \includegraphics [scale=0.36] {inversion6.png} \end{center}

\emph{Proof}.

We have $\triangle OAC \sim \triangle OC'A'$ and $\triangle OBC \sim \triangle OC'B'$.  From similar triangles, the angles marked as $\alpha$ are equal, as are those marked as $\beta$.

$\angle ACB$ is right, but $\alpha + \beta$ is supplementary to $\angle ACB$ to $\alpha + \beta$ so the sum is $90$.

It follows that $\angle A'C'B' = \alpha + \beta$ is also right.  Therefore $C'$ lies on a circle whose diameter is $A'B'$.  

$\square$.

Third case: the circle $K$ intersects $\omega$ but the inversion center $O$ is external to $K$.

\begin{center} \includegraphics [scale=0.36] {inversion7.png} \end{center}

Draw the diameter $AB$ of circle $K$ which, extended, goes through $O$.  Let $C$ be any other point on circle $K$.

\emph{Proof}.

The angles marked $\alpha$ and $\beta$ are equal (in pairs) by similar triangles.

$\angle ACB$ is right.  So $\angle BOC + \alpha + \beta$ is equal to a right angle.

But $\angle BOC + \alpha + \beta + \angle A'B'C'$ is equal to two right angles.

Thus, $\angle A'C'B'$ is also right.

$\square$

The fourth and last case is where the circle $K$ and $\omega$ are concentric.  But then the inverse of $K$ is also a concentric circle.

\subsection*{Invariants}

Either a given point is on a circle (or line) or it is not.  If it is on the locus, then it must also lie on the image of the locus under inversion.

This seems quite obvious, yet subtle.  We will not prove it here.

Then if a circle or a line cuts a circle at two points, then after inversion the resulting figures will still have two points of intersection.

And if a line is tangent to a circle or if two circles are tangent, or a line cuts a line, after inversion the resulting figures will still be tangent (in the first two cases) and in all cases, share one point of intersection.

The next figure shows a circle $K$ not through $O$ and its inverse $K'$.  We notice that lines tangent to the original circle $K$ are also tangent to the inverse circle $K'$

However, although the center of $K$ and the center of $K'$ are on the same line through $O$, they are \emph{not} inverses.

\begin{center} \includegraphics [scale=0.35] {inversion10.png} \end{center} 

We'll see why in just a bit.

\subsection*{Inversion preserves angles}.

As a special case (and the only one we will use for the Feuerbach theorem, below), if two circles are orthogonal, then their inverses are also orthogonal.

Two circles are orthogonal if, at the points where they cross, the two tangents are perpendicular.  Equivalently, the two radii to that point must be perpendicular.

Then, considering a circle orthogonal to $\omega$, the claim is that inversion in $\omega$ leaves such a circle invariant, or fixed.

\begin{center} \includegraphics [scale=0.45] {FB4.png} \end{center}

\emph{Proof}.

Let the inversion circle $\omega$ have its center at $O$ and have radius of length $k$.

Another circle $K$ with radius of length $R$ is positioned so that at the points where the two circles cross, the tangents (and radii) are perpendicular.

The first thing we notice is that the points $S$ and $T$ lie on $\omega$, so they are fixed, their inverses are the same as the original point.

Next consider a pair of points lying $P$ and $P'$ on a diameter of $K$, $P'$ on a distance $x$ from the inversion center, while $P$ lies a distance $2R+x$ from the inversion center.  The product of the distances is
\[ (2R + x)x = 2Rx + x^2 \]

But we also have from the geometry that
\[ k^2 + R^2 = (R+x)^2 \]
\[ k^2 = 2Rx + x^2 \]

The product of the two distances is just $k^2$, so the inverse of $P$ is $P'$ and vice-versa.  We now know that four points of the inverted circle are also on the original circle.  The result follows.

Even more generally, we may consider an arbitrary secant $QQ'$.  Recall from the secant tangent theorem that $OQ \cdot OQ' = OT^2 = k^2$.

Any pair of points on any secant of circle $K$ are their own inverses, so they both lie on the inverted circle.  Thus, the circle is fixed under inversion.

$\square$

A relatively simple proof for circles that are not necessarily orthogonal comes from the web.  I have lost the author's name.

We start with the special case where one circle goes through $O$:

\begin{center} \includegraphics [scale=0.30] {inversion11.png} \end{center}

The line $m$ through $O$ transforms into itself.  The circle is also through $O$, and is the inverse of the line $l$.  The angle between $l$ and $m$ is easily shown to be equal to the angle between $m$ and the image of the line $l$, namely the circle, at $A'$.

The shaded gray angle at $A$ is complementary to $\angle AOP$.  But $\angle AOP = \angle QOA' = \angle QA'O$, and $\angle QA'O$ is complementary to the gray shaded angle at $A'$.  $\square$ 

In the general case, suppose the circles on centers $B$ and $B'$ are inverses through $I_{\omega}$.  Let $A$ and $A'$ be inverse points as usual.

\begin{center} \includegraphics [scale=0.30] {inversion12.png} \end{center}

This does \emph{not} mean that $I_{\omega} (B') = B$!  If that were true, then $\angle OBA$ would be equal to the angles marked in green, which would mean that $OA \parallel OB$, which is clearly not true.

However, the green angles at $A'$ and $A_1$ \emph{are} equal by similar triangles.  This requires a bit of work.  We simplify the diagram a bit and change the notation as far as the center of the circle, and also introduce tangent points $T$.

\begin{center} \includegraphics [scale=0.25] {inversion14.png} \end{center}

By the secant-tangent theorem, $OA_1 \cdot OA' = OT_1^2$ and this is true wherever $A'$ and $A_1$ lie.

By the definition of inversion $OA \cdot OA' = k^2$.  So

\[ \frac{OA_1}{OA} = \frac{OT_1^2}{r^2} \]
which is a constant.  Call it $c$.  So then $OA_1 = c \cdot OA$ for every points $A_1$ and $A$ on the same line circles $Q$ and $Q_1$, \emph{at the second point of intersection}.

\begin{center} \includegraphics [scale=0.25] {inversion14.png} \end{center}

Since $\triangle OQT \sim \triangle OQ_1T_1$, that includes $Q$ and $Q_1$.  This relationship is called a homothety.  The entire circle on $Q_1$ is a bigger version of the one on $Q$.

It follows that $\triangle OAQ \sim \triangle OA_1 Q_1$.  We now switch back to the original notation:  $\triangle OB'A' \sim \triangle OBA_1$, which gives two of the equal angles colored green in the figure below.  The third one (at $A$) is equal by similar right triangles.

\begin{center} \includegraphics [scale=0.25] {inversion13.png} \end{center}

The gray angles are both complementary to the green angles.  The second one in circle $B$ (at $A$) is equal by reflection across a line of symmetry through $B$ and $\perp m$.

$\square$

\emph{Proof}.  (Alternate).

This is section 5.5 of Coxeter.

\begin{center} \includegraphics [scale=0.25] {Coxeter_5_5A.png} \end{center}

Consider two circles intersecting at $P$ (not drawn).  The supplementary angles formed between the circles are naturally defined in terms of the tangents to the two circles at $P$.  Suppose the angles are $\theta$ and $\phi$.  Now consider just the two tangent lines $a$ and $b$, which form $\angle \theta$ in the figure above.

We perform an inversion with respect to circle $\omega$ on center $O$.  The inverse of a line not through $O$ is a circle through $O$, where the tangent to the new circle at $O$ is parallel to the original line.  So the inverse of line $a$ is circle $\alpha$ with its tangent parallel to $a$, and similarly for $b$.  We can see that the angle between these two tangents at $O$ is also equal to $\theta$, since they are parallel to the original lines $a$ and $b$.

As we said above, if $P$ lies on both circles before inversion, it lies on both circles (or lines) after inversion.  Thus the tangents at $P'$ are equal to those at $P$ and form the same angle.

We can also see that for intersecting circles, the angles at the two points of intersection are the same although the tangents have a mirror image symmetry.
\begin{center} \includegraphics [scale=0.45] {FB5a.png} \end{center}

\rule{\linewidth}{1pt}

\subsection*{Feuerbach's Theorem}

Finally, we reach this famous theorem.

\begin{center} \includegraphics [scale=0.12] {Feuerbach.png} \end{center} 

Recall that the nine point circle, passing through the midpoints of the sides, is the circumcircle of the medial triangle, i.e., it is formed with these points as the vertices.

Feuerbach's theorem says that this circle is tangent (internally) to the incircle and (externally) to each of the excircles of a triangle.

\begin{center} \includegraphics [scale=0.36] {FB8.png} \end{center}

We previously looked at the incircle and three excircles of a triangle.  Recall that the nine point circle also goes through the feet of the altitudes and the bisectors of that part of each altitude between the orthocenter and each vertex.

Let $\triangle ABC$ have the sides bisected at $A'$, $B'$ and $C'$.  The incircle is drawn on center $I$, and one excircle is drawn tangent to side $a = BC$, on center $O_a$.  $\triangle A'B'C'$ is also drawn.

\begin{center} \includegraphics [scale=0.35] {FB1.png} \end{center}

\textbf{I}

We know from previous work that $BX = BY = s-b$, $AY = AZ = s-a$ and $CX = CZ = s-c$.  As well, the entire tangent from $A$ to the excircle has length $s$, the semi-perimeter.

Radii are drawn from the incenter $I$ and the excenter $O_a$, perpendicular to side $BC$.  $BC$ is tangent to the two circles at these points.  It follows that
\[ IX \parallel O_a X_a \]

We also know from previous work with excircles that $BX = X_a C = s-b$.  So
\[ X X_a = a - 2(s-b) = a - (a - b + c) = b - c  \]

We also have that $XA' = X_a A'$.  That's because $A'$ is the midpoint of side $BC = a$, and we've subtracted equal lengths from both ends with $BX = X_a C$.

Therefore, we can draw the circle as shown, with $A'$ as the center and radius $A'X = A'X_a = (b-c)/2$.

We will refer to that circle as $\omega$.  It will be the inversion center for the transformation.

We also draw another line tangent to both incircle and excircle.  Label the endpoints as $B_1$ and $C_1$.

Also mark three other points:   where $A'B'$ crosses $B_1 C_1$ at $B''$, where $A'C'$ crosses $B_1 C_1$ at $C''$, and finally, the point where $BC$ crosses $B_1 C_1$, at $S$.

\begin{center} \includegraphics [scale=0.35] {FB2.png} \end{center}

\textbf{II}

The center of the inversion circle $\omega$ is at $A'$, which is also on the nine-point circle.  The inverse of a circle through the inversion center is a line.

The claim is that $B''$ is the inverse of $B'$, and $C''$ is the inverse of $C'$ so the straight line $B_1 C_1$ which goes through those two points is the inverse of the nine point circle containing points $A'$, $B'$ and $C'$.

To begin with we obtain some more lengths.  Use the angle bisector theorem to get an expression for the length of $BS$ and $CS$, the components of side $a$ produced by the bisector.  The claim is that
\[ BS = \frac{ac}{b+c} \ \ \ \ \ \ CS = \frac{ab}{b + c} \]

This looks plausible because the sum is just $a = BC$, which checks.  

Coxeter explain it thus:  \begin{quote}Since $S$ (like $I$ and $I_a$) lies on the bisector of the angle $A$, $S$ divides the segment $BC$, of length $a$, in the ratio $b:c$. \end{quote}

Thus the two parts of $a$ are the fractions $b/(b+c)$ and $c/(b+c)$.

\begin{center} \includegraphics [scale=0.35] {FB2.png} \end{center}

Working through the algebra just to confirm this, let $BS = u$ and $CS = v$.  Then $u/c = v/b$ so $u/v = c/b$.  Add one to both sides
\[ \frac{u+v}{v} = \frac{b + c}{b} \]
but $u+v = a$ so we have $v = ab/(b+c)$.  Alternatively add one to both sides of the inverse of what's above.  Then
\[ \frac{u+v}{u} = \frac{b + c}{c} \]
and the second result follows easily as well.

Another claim is that
\[ SA' = \frac{a(b-c)}{2(b+c)} \]
We get this from $BC = a$, so $BA' = a/2$ and then
\[ SA' = \frac{a}{2} - \frac{ac}{b+c} \]
Placed over a common denominator and subtracting, that checks.

The last claim, which I missed in a youtube video of this proof but is in the source (Coxeter), is that $BC_1 = B_1C = b-c$.  This is easier to see for $B_1 C$ so let us start there.  The whole of side $AC$ has length $b$. so somehow it must be the case that $AB_1$ has length $c$ and then $B_1 C$ is the difference.

\begin{center} \includegraphics [scale=0.35] {FB3.png} \end{center}

We must show that $B_1 C = BC_1 = b-c$.  There is a lot of symmetry in the figure.  Both circles have their centers on the bisector of $\angle A$.

We have that $\triangle SIX$ and $\triangle SIX_1$ have three sides equal (shared side, radius, and two tangents from a point), so they are congruent, which gives $\angle ASX = \angle ASX_1$.

It follows that $\triangle ASB$ and $\triangle ASB_1$ are equiangular and share side $AS$, so they are congruent.  Thus $AB = AB_1 = c$.

So $B_1C = b - c$, by subtraction, and 
\[ B_1Z = b - (s-a) - (b-c) = a + c - s = s - b \]
One can also get this from $SB = SB_1$ (congruent triangles, above) and $SX = SX_1$, so by subtraction, $BX = B_1X_1 = B_1Z$.

The whole $\angle ASC = \angle ASC_1$ because the parts are equal (from congruent triangles and then vertical angles).  It follows that $\triangle ASC \cong \triangle ASC_1$.  So $AC = AC_1$ and $AB = AB_1$, and then finally $B_1 C = BC_1$, also by subtraction.

We do not need it but we can also show that $BC = B_1C_1$.  $\triangle ABC \cong \triangle AB_1C_1$.

\textbf{III}

\begin{center} \includegraphics [scale=0.35] {FB2.png} \end{center}
Next, we must find two pairs of similar triangles.  These are
\[ \triangle SA'B'' \sim \triangle SBC_1 \ \ \ \ \ \ \triangle SA'C'' \sim \triangle SCB_1 \]

These are easy because we have parallel lines and either vertical angles in the first case, or shared $\angle CSB_1$ in the second.

We have 
\[ \frac{SA'}{A'B''} = \frac{SB}{BC_1} \]
\[ A'B'' = SA' \cdot BC_1 \cdot \frac{1}{SB} \]
Plugging in the results from above
\[ = \frac{a(b-c)}{2(b+c)} \cdot (b-c) \cdot \frac{b+c}{ac} \]
and since $A'B' = c/2$:
\[ A'B' \cdot A'B'' = (\frac{b-c}{2})^2 \]

For the second triangle we obtain:
\[ \frac{SA'}{A'C''} = \frac{SC}{B_1C} \]
\[ A'C'' = SA' \cdot B_1C \cdot \frac{1}{SC} \]
And again, plugging in:
\[ = \frac{a(b-c)}{2(b+c)} \cdot (b-c) \cdot \frac{b+c}{ab} \]
and since $A'C' = b/2$:
\[ A'C' \cdot A'C'' = (\frac{b-c}{2})^2 \]

But this is (the same for both), namely $r^2$.

\begin{center} \includegraphics [scale=0.35] {FB2.png} \end{center}

So $B''$ and $C''$ are the inverses of $B'$ and $C'$ respectively, with respect to $\omega$.

\textbf{IV}

We have an inversion circle centered on $A'$ with diameter $XX_a$ that we have labeled $\omega$.

We have shown that the inverse of the nine-point circle is the line $B_1 C_1$, which passes through $BC$ at $S$ and is tangent to both the incircle and to the excircle (these points are not labeled).

Since the inverses of the incircle and excircle are orthogonal with respect to $\omega$, they are fixed (they invert into themselves).

So then, finally, in the inverted system we have line $B_1 C_1$ and the incircle and excircle, with the line tangent to both of these circles.

Points where two curves cross or touch are still points where the inverted figures cross or touch, after inversion.

Since $B_1 C_1$ is tangent to both circles, its inverse is still tangent to both circles.  The inverse of $B_1 C_1$ is the nine point circle, so that circle is tangent to both the other circles, i.e. the incircle and the excircle, and to the other two excircles, by extension.

\begin{center} \includegraphics [scale=0.36] {FB8.png} \end{center}

\subsection*{other ideas}

Finally, a note about other approaches.  One can show that two circles are tangent externally (such as the excircle and nine-point circle), by showing that the distance between the two centers is equal to the sum of the radii.

For internal tangency, the distance between centers must be the difference of the two radii.

Thus, if one can calculate \emph{where} the centers lie, then this requirement can be checked.  A judicious choice of coordinate system will help, and you can find such proofs on the internet.  However, I think the inversion proof presented in detail here is a lot more fun, and quite simple, once we understand what inversion is all about.


\end{document}