\documentclass[11pt, oneside]{article} 
\usepackage{geometry}
\geometry{letterpaper} 
\usepackage{graphicx}
	
\usepackage{amssymb}
\usepackage{amsmath}
\usepackage{parskip}
\usepackage{color}
\usepackage{hyperref}

\graphicspath{{/Users/telliott/Dropbox/Github-math/figures/}}
% \begin{center} \includegraphics [scale=0.4] {gauss3.png} \end{center}

\title{B\'ezout's Identity}
\date{}

\begin{document}
\maketitle
\Large

%[my-super-duper-separator]

\subsection*{lemma}
Let $a,b \in \mathbb{N}$, i.e. $\{ 1, 2, 3, \dots \}$.  There exist $x,y \in \mathbb{Z}$ (integers) such that
\[ \text{gcd} \ (a,b) = xa + yb \]
B\'ezout's lemma (or identity) is a statement about the \emph{greatest common divisor} of two numbers $a,b \in \mathbb{N}$.  It says that we can always find two integers $x$ and $y$, such that the resulting linear combination of $a$ and $b$ is equal to the gcd$(a,b)$.

\subsection*{examples}
Two primes:
\[ \text{gcd} \ (3,2) = x \cdot 3 + y \cdot 2 = 3 \cdot 3 + (-4) \cdot 2 = 1 \]
One of $x,y$ is either negative or zero, since the gcd is less than or equal to the smaller of $a,b$.  Next, a multiple.  Here one of $x,y$ is zero:
\[ \text{gcd} \ (4,2) = x \cdot 4 + y \cdot 2 = 0 \cdot 4 + 1 \cdot 2 = 2 \]
Finally:
\[ \text{gcd} \ (81,45) = x \cdot 81 + y \cdot 45 = (-1) \cdot 81 + 2 \cdot 45 = 9 \]
The lemma does not say how to find the gcd.  But we know a good method for that:  Euclid's algorithm.

\subsection*{preliminary}
Let $d = $ gcd($a,b$).

We use the symbol $|$ to mean "divides", or is a factor of, leaving no remainder.  If $n$ is any even number, then $2|n$, since $n = 2 \cdot q$ + r where $r$ is zero.

We know that $d|a$ and $d|b$ so $d$ divides \emph{every} linear combination $xa + yb$ for integer $x,y$.

\begin{quote}  \emph{Proof}: If $p | m$ then $m = jp$ for some (integer) $j$, similarly if $p | n$ then $n = kp$ for some $k$.  Thereefore, $m + n = (j + k)p$, which means that $p | (m + n)$.\end{quote}

We consider only positive combinations:  $xa + by > 0$.  Since $d$ divides all of them, $d$ must be smaller than or equal to every such combination.  So we expect it will be equal to the least of them.

If $S$ is the set of such combinations, we know that $S$ is not empty (clearly, $a \in S$), and also $S \subset \mathbb{N}$.  As a consequence, the well-ordering principle applies, and we know there is a least element.

\subsection*{outline}

Let the least element of $S$ be $m$, and as we said, let $d = $ gcd($a,b$).  

We will show that $d = m$, and since $m \in S$ whose elements are all linear combinations $xa + yb$, that will complete the proof that there is a linear combination of $a$ and $b$ that is equal to $d$.

We will do this by showing first that $d|m$ which implies that $d \le m$.  

And then second, $m$ is a common divisor of $(a,b)$.  But $d$ is the \emph{greatest} common divisor of the same two numbers, so $m \le d$.  

Since $d \le m$ and $m \le d$, therefore $m = d$.

\subsection*{d divides m}
Again, $d = $ gcd($a,b$) so $d|a$ and $d|b$.  It follows that $d | (xa + yb)$ for any integer $x,y$.  

That is, $d$ divides every element of $S$ so it must be that $d | m$.

Therefore, $d \le m$.

\subsection*{m divides a and b} 
We claim that $m|a$, in other words $a = qm + r$ with $r = 0$.  

\emph{Proof}.  (By contradiction).  

In the expression $a = qm + r$, suppose $r$ is not zero, that is, suppose $0 \le r < m$.  Recall that $m = xa + yb$:
\[ r = a - qm = a - q(xa + yb) \]
\[ = (1 - qx)a + (-qy)b \]
But $1 - qx$ and $-qy$ are both $\in \mathbb{Z}$.  It follows that $r$ is a linear combination of $a$ and $b$ and $r \in S$, since $r > 0$.

We have that $m$ is the smallest element in $S$, $r \in S$ and $r < m$.  

This is a contradiction.

Therefore, $r = 0$ and thus, $m|a$.

The same reasoning will show that $m|b$.  Since $m|a$ and $m|b$ so $m \le d$.

This completes the proof.

$\square$

\url{http://ramanujan.math.trinity.edu/rdaileda/teach/s20/m3326/lectures/bezout_handout.pdf}





\end{document}